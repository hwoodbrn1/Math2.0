\documentclass[11pt, reqno]{article}

\usepackage{amsmath, amsthm, amssymb}
\usepackage{enumitem}
\usepackage{tcolorbox}
\usepackage{hyperref}
\usepackage{tikz}
\usepackage{tikz-cd}
\usetikzlibrary{arrows.meta}
\usepackage{mathrsfs}
\usepackage{fancyhdr}
\usepackage[bottom=0.75in, top=1in, left=0.5in, right=0.5in]{geometry}
\usepackage{array}   % for \newcolumntype macro
\newcolumntype{L}{>{$}l<{$}}

\theoremstyle{plain}
\newtheorem*{theorem}{Theorem}
\newtheorem*{proposition}{Proposition}
\newtheorem{exercise}{Exercise}
\newtheorem*{lemma}{Lemma}
\newtheorem*{corollary}{Corollary}

\theoremstyle{definition}
\newtheorem*{definition}{Definition}
\newtheorem*{example}{Example}

\theoremstyle{remark}
\newtheorem*{remark}{Remark}

\renewcommand{\phi}{\varphi}
\renewcommand{\epsilon}{\varepsilon}
\renewcommand{\emptyset}{\varnothing}

\newcommand{\RR}{\mathbb{R}}
\newcommand{\ZZ}{\mathbb{Z}}
\newcommand{\NN}{\mathbb{N}}
\newcommand{\CC}{\mathbb{C}}
\newcommand{\QQ}{\mathbb{Q}}

\DeclareMathOperator{\ima}{\text{im}}

\begin{document}

\topmargin=-40pt
\rhead{Henry Woodburn}
\lhead{Math 636}
\renewcommand{\headrulewidth}{1pt}
\renewcommand{\headsep}{20pt}
\thispagestyle{fancy}

{\Huge \bfseries \noindent Homework 7}

\subsection*{Section 30}

\begin{enumerate}
    \item[1.] (a.) We show every point in a first countable T1 space $X$ is a $G_\delta$ set. 
    
    Let $\mathcal{B}_x$ be a basis at $x \in X$. Let $U = \cap_{B \in \mathcal{B}_x} B$. Then we need to show $U = \{x\}$. 
    Choose any $y \neq x$. Then there is an open set $V \ni x$ with $y \notin V$. Then 
    there is some basis element $B \in \mathcal{B}_x$ such that $B \subset V$, and thus $y \notin B$. 
    Then $y \notin U$. Then $U$ contains only $x$.

    (b.) Take $\RR$ with the discrete topology. Then every point is a countable intersection of 
    the intervals $(x-1/n, x+1/n)$, but $\RR$ with this topology is clearly not first countable.
    If there were any countable basis at $x$, we could construct another open set which 
    contains $x$ but no basis element by removing one point from each basis element and taking their 
    union. 

    \item[4.] We show every compact metric space $X$ has a countable basis. Cover 
    $X$ in balls of radius $1/n$ and take a finite subcover. Call this set $\mathcal{B}_n$. 
    Let the countable basis be $\mathcal{B} = \cup_{n = 1}^\infty \mathcal{B}_n$. To verify this,
    choose any $x \in X$ and any open set $U$ containing $x$. Then choose an epsilon 
    ball $x \in B_\epsilon(x) \subset U$. Then choose $n$ such that $1/n < \epsilon$. We know
    that there is some ball $B \in \mathcal{B}$ which contains $x$. Then every point in $B$
    is less than $\epsilon$ away from $x$, so $B \subset B_\epsilon(x) \subset U$.

    \item[11.] Let $f: X \rightarrow Y$ be continuous. We show that if $X$ is lindelof or has a countable
    dense subset, then so does $f(X)$. 
    
    First suppose $X$ is lindelof. Let $\{U_\alpha\}_{\alpha \in A}$ be a cover of $f(X)$. 
    Then $\{f^{-1}(U_\alpha)\}_{\alpha \in A}$ is a cover of $X$. Take a countable 
    subcover $\{U_n\}_1^\infty$. Then 
    \[
        f(X) = f(\cup(f^{-1}(U_n)) = \cup_1^\infty U_n
    \]
    and thus $U_n$ is a countable cover of $f(X)$.
\end{enumerate}

\subsection*{Section 31}

\begin{enumerate}
    \item[5.] Let $f, g: X \rightarrow Y$ be continuous and suppose $Y$ is hausdorff. We have shown
    that $Y$ is hausdorff iff the diagonal in $Y \times Y$ is closed. We have $(f,g): X \rightarrow Y \times Y$
    continuous, so the preimage of the diagonal is a closed set in $X$. This is exactly the set 
    we desired to show was closed.

    \item[6.] Let $p:X \rightarrow Y$ be a closed surjective map. Let $A, B$ be closed disjoint sets in $Y$. 
    Then $p^{-1}(A)$ and $p^{-1}(B)$ are closed in $X$ and thus we can choose disjoint open 
    $U, V$ containing $p^{-1}(A)$ and $p^{-1}(B)$. Then $U^c$ and $V^c$ are both closed,
    so $E := p(U^c)^c$ and $F := p(V^c)^c$ are open. These sets are disjoint since if $x$ is in both,
    then it is not in the image of either $U^c$ or $V^c$. But every point is in the 
    image of one of these sets, since $U^c \cup V^c = (U \cap V)^c = X$. 

    Let $x \in A$ and suppose $x \notin E$. Then $x \in p(U^c)$, but this is impossible 
    since $p^{-1}(A) \subset U$.
\end{enumerate}

\subsection*{Section 32}
    
\begin{enumerate}
    \item[6.] We show a space $X$ is completely normal iff there exist disjoint 
    open neighborhoods of every separated pair of sets.
    
    First suppose every subspace of $X$ is normal. Let $A, B$ be separated sets. Then 
    $E := X \setminus(\overline{A} \cap \overline{B})$ is an open set. Also,
    $E$ contains both $A$ and $B$ by the separation condition. Then $\overline{A} \cap E$
    and $\overline{B} \cap E$ are closed disjoint sets in $E$ and there exist open (in $E$) disjoint sets 
    $U$ and $V$ containing them respectively. But an open subset of an open set is open in the original
    space. Then we are done, since clearly $U$ and $V$ contain $A$ and $B$ as their closures 
    in $E$ contain them. 

    Now suppose the other condition holds. Let $Y$ be a subspace of $X$ and suppose $A$ and $B$
    are closed and disjoint in $Y$. Then $A$ and $B$ are separated subsets of $X$ and there exist 
    $U$ and $V$ open disjoint neighborhoods of $A$ and $B$. Then we can intersect these with $Y$ 
    to show that $Y$ is normal. 
\end{enumerate}

\subsection*{Section 33}

\begin{enumerate}
    \item[2.] (a.) Let $X$ be a connected normal space with more than one point, say $x_1$ and $x_2$. Then
    by Urysohn's lemma there is a continuous function such that $f(x_1) = 0$ and $f(x_2) = 1$. Since the 
    image of a connected space is connected, $f(X)$ is a connected subset of $\RR$ and thus contains 
    uncountably many points. Then $X$ must be uncountable as well.

    (b.) Let $X$ be a regular and lindelof space. We know that every such space is normal. To see this one 
    can replicate the proof that every regular second countable space is normal. Then we can apply (a.) 
    to finish.
\end{enumerate}

\end{document}