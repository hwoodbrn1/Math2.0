\documentclass[11pt, reqno]{article}

\usepackage{amsmath, amsthm, amssymb}
\usepackage{enumitem}
\usepackage{tcolorbox}
\usepackage{hyperref}
\usepackage{tikz}
\usetikzlibrary{arrows.meta}
\usepackage{mathrsfs}
\usepackage{fancyhdr}
\usepackage[bottom=0.75in, top=1in, left=0.5in, right=0.5in]{geometry}
\usepackage{array}   % for \newcolumntype macro
\newcolumntype{L}{>{$}l<{$}}

\theoremstyle{plain}
\newtheorem*{theorem}{Theorem}
\newtheorem*{proposition}{Proposition}
\newtheorem{exercise}{Exercise}
\newtheorem*{lemma}{Lemma}
\newtheorem*{corollary}{Corollary}

\theoremstyle{definition}
\newtheorem*{definition}{Definition}
\newtheorem*{example}{Example}

\theoremstyle{remark}
\newtheorem*{remark}{Remark}

\renewcommand{\phi}{\varphi}
\renewcommand{\epsilon}{\varepsilon}
\renewcommand{\emptyset}{\varnothing}

\newcommand{\RR}{\mathbb{R}}
\newcommand{\ZZ}{\mathbb{Z}}
\newcommand{\NN}{\mathbb{N}}
\newcommand{\CC}{\mathbb{C}}
\newcommand{\QQ}{\mathbb{Q}}

\DeclareMathOperator{\ima}{\text{im}}

\begin{document}

\topmargin=-40pt
\rhead{Henry Woodburn}
\lhead{Math 636}
\renewcommand{\headrulewidth}{1pt}
\renewcommand{\headsep}{20pt}
\thispagestyle{fancy}

{\Huge \bfseries \noindent Homework 5}

\subsection*{Topological Groups}

\begin{enumerate}
    \item[4.] Since $G$ is a topological group, the multiplication operation is a continuous
    map $G\times G \rightarrow G$. Then the map sending $x \mapsto \alpha\cdot x$ is just the 
    composition of the projection in the first coordinate onto $\alpha$ with the multiplication
    map. Both are continuous, so the map $x \mapsto \alpha \cdot x$ is also. The same is true 
    for right multiplication by $\alpha$.

    Also, it is well known that left (right) multiplication by an element is a bijection $G \rightarrow G$.

    The inverse map $f_{\alpha^{-1}}$ is also continuous, thus $f_{\alpha}$ is a homeomorphism. 

    Then for any pair $x, y \in G$, the map $f_{yx^{-1}}$ is a homeomorphism sending $x$ to $y$.

    \item[5.] Let $H$ be a subgroup of $G$ and give $G/H$ the quotient topology. 
    \begin{enumerate}
        \item Let $f_\alpha$ be the map from question $4$ and let $f_\alpha'$ be the induced map on
        $G/H$ the set of cosets of $H$ in $G$. Specifically, $f_\alpha'$ sends an element $x \in G/H$
        corresponding to a coset $xH$ into the coset $\alpha xH$, and returns the corresponding
        element $\alpha x$ in $G/H$. It is well known that this map is a 
        bijection on $G/H$ the set of cosets of $H$.

        Let $U$ be an open set in $G/H$. We will show $f_\alpha'^{-1}(U)$ is open. 
        $U$ corresponds to a set of cosets we will denote by $U\cdot H$. This 
        set is open in $G$ by the definition of the quotient topology on $G/H$. The map 
        $f_\alpha'^{-1}$ sends $U\cdot H$ to $\alpha^{-1}\cdot (U\cdot H)$. Of course, $\alpha^{-1}\cdot(U\cdot H)$ is a subset of $G$ which is the image of 
        the set $U\cdot H$ under the map $f_\alpha^{-1}$. Thus $\alpha^{-1}\cdot U\cdot H$ is open in $G$ by problem (4.). On the other hand, 
        we know that $f_\alpha^{-1}$ sends entire cosets to entire cosets, so that $\alpha^{-1}\cdot U\cdot H$ is saturated in the 
        projection map onto $G/H$. Then by the definition of the quotient map, $\alpha^{-1} U$ is in fact an open set in 
        $G/H$.

        The inverse map $f_\alpha'^{-1} = f_{\alpha^{-1}}'$ is continuous by the same reasoning, and we have already mentioned that 
        $f_\alpha'$ is a bijection. Thus it is a homemomorphism of $G/H$. If $xH$ and $yH$ are two cosets in $G/H$,
        the homeomorphism $f_{yx^{-1}}'$ sends $xH$ to $yH$.
    
        \item Let $H$ be a closed set in $G$. Then a single point set $\{xH\} \in G/H$ is closed if and only if the 
        set $xH$ is closed as an element of $G$ by the definition of a quotient map. By problem (4.) we know that this is true.
        
        \item Let $U \subset G$ be an open set. We cannot say whether $p(U)$ is open in $G$ because $U$ may not be saturated. 
        However, the image of $U$ under $p$ is the same as the image of its "saturation", $p^{-1}(p(U))$, which shows 
        $p(U)$ is open
        in the quotient topology on $G/H$.

        \item We will use the map $p$ from the previous part. If $H$ is closed and normal, $G/H$ forms a group with 
        multiplication $(xH)\cdot(yH) = (xy)H$. It satisfies the T1 property by part (c.). Let $m: G \times G \rightarrow G$
        be the multiplication map, and $m'$ the multiplication on $G/H$.
        \bigbreak
        If $UH \subset G/H$ is open, then $m'^{-1}(U) = \{(xH,yH) \in G/H \times G/H: xyH \in U\}$.
        This is equivalent to the set $A = \{(x,y) \in G \times G: xy \in UH\}$, since $xyH = zH$ for $zH \in U$ if and only if 
        $xy \in zH$. But $A$ is the preimage $m^{-1}(UH)$, and we know $UH$ is open in $G$. Then $A$ is open in $G$ as well.
        
        Then $m'^{-1}(U) = (p \times p)(A)$. We know $p$ is an open map, and for a basic open set $E\times F$ in $G/H \times G/H$,
        $(p \times p)(E \times F) = (p(E),p(F))$, an open set in $G/H \times G/H$. Since open sets are unions of basic open 
        sets and $(p \times p)(A \cup B) = (p\times p)(A) \cup (p \times p)(B)$, we know that $(p \times p)$ is an open 
        map, and thus $m'^{-1}(U) = (p \times p)(A)$ is open in $G/H \times G/H$. 

        To show the map $j: xH \rightarrow x^{-1}H$ is continuous is the same as part (a.), where 
        we need to show that for some open $U \in G/H$, $U^{-1}\cdot H$ is open and saturated in $G$. This is again true
        by the well-definedness of the group operation on $G/H$ when $H$ is normal. 
    \end{enumerate}
\end{enumerate}

\subsection*{Section 23}

\begin{enumerate}
    \item[3.] Let $A_\alpha$ and $A$ be connected subspaces of $X$, and suppose $A \cap A_\alpha \neq \emptyset$ for all $\alpha$.
    By Theorem 23.3, for each $\alpha$, $A \cup A_\alpha$ is connected since they have nonempty intersection. 
    Then 
    \[
        A \cup (\bigcup A_\alpha) = \bigcup (A \cup A_\alpha)
    \]
    is a union of sets each sharing a common point, namely any point in $A$, and is connected by 23.3.

    \item[8.] Give $\RR^\omega$ the uniform topology. Let $A$ be the set of bounded sequences in $\RR^\omega$. I claim that 
    $A$ is both open and closed.
    \bigbreak
    To show $A$ is closed, let $(x_n)$ be a limit point of $A$. Then by the definition of the uniform metric, for every
    $\epsilon > 0$ there is some $(y_n) \in A$ which is not equal to $(x_n)$, such that $|x_n - y_n| < \epsilon$ for 
    all $n$. Thus if $|y_n| < M$ for all $n$, then $|x_n| < M + \epsilon$ for all $n$, and thus $(x_n) \in A$. Then
    $A$ contains all of its limit points and is closed. 

    Now we show $A$ is open. Let $(x_n) \in A$ with $|x_n| < M$ for all $n$. 
    Then for any $0 < \epsilon < 1$, the $\epsilon$-ball centered at $(x_n)$
    contains only sequences which are bounded by $M + \epsilon$, and thus is contained in $A$. Then $A$ is open.

    Since $A$ is not the entire space, $\RR^\omega$ is disconnected by the alternate formulation of connectedness.

    \item[11.] Let $p: X \rightarrow Y$ be a quotient map, with $Y$ connected and $p^{-1}(y)$ connected for each $y \in Y$. 
    Suppose $X = A \cup B$ for open and disjoint sets $A$ and $B$. Since $p^{-1}(y)$ is connected, we must have 
    $p^{-1}(y)$ entirely contained in either $A$ or $B$ for all $y \in Y$. Then the sets $A$ and $B$ are saturated open sets,
    so $p(A)$ and $p(B)$ are open and disjoint, with $Y = p(A) \cup p(B)$. Since $Y$ is connected, either set must be empty
    and thus one of $A$ and $B$ must be empty. Then $X$ has no disconnection.
\end{enumerate}

\subsection*{Section 24}

\begin{enumerate}
    \item[4.] Let $X$ be an ordered set equipped with the order topology and suppose $X$ is connected. To show that $X$ is a 
    linear continuum, we must show that for all $x < y$, there is some $z \in X$ with $x < z < y$, and that 
    every set has a least upper bound. 
    \bigbreak
    The first condition is easy. If this is not true, then $(-\infty, y)\cup(x, \infty)$ is a separation of $X$ which is a 
    contradiction. 

    For the least upper bound property, let $A$ be a set in $X$. Let 
    \[
        F := \bigcap_{y \in A}\{x \in X: x \geq y\}
    \]
    be the intersection of all closed rays going to infinity starting from elements of $A$. Then $F$ is closed 
    since it is an intersection of closed sets in the order topology. 
    Similarly let 
    \[
        E = \bigcup\limits_{y \in X\\ y > A} \{x \in X: x > y\}
    \]
    be the union of open rays to infinity starting from points greater than every element in $X$. It is open.

    It is clear that $E \subset F$. But if $E = F$, then $F$ is a set which is both open and closed. Moreover,
    $F$ is not $X$ since it does not contain $A$. Then this contradicts that $X$ is connected. 

    Then there must be some point $x \in F \setminus E$. Since $x \in F$, we know that $x$ is an upper bound. 
    Since $x \notin E$, we have $x \leq y$ for all upper bounds $y$ of $A$. Then $x$ is a least upper bound for $A$.

    \item[8.] 
    \begin{enumerate}
        \item[a.] The product of path connected spaces is path connected. Let $X$ and $Y$ be path connected and 
        choose points $(x_1, y_1), (x_2, y_2) \in X \times Y$. Let $f: \RR \rightarrow X$ be a path from $x_1$
        to $x_2$, and $g: \RR \rightarrow Y$ be a path from $y_1$ to $y_1$. Then the function
        \[
            g := \begin{cases} (f(2t), y_1) & 0 \leq t < \frac{1}{2}\\ (x_2, g(2t-1)) & \frac{1}{2} \leq t \leq 1\end{cases}
        \]
        is a path from $(x_1, y_1)$ to $(x_2, y_2)$.

        \item[b.] The closure of a path connected space is not path connected. The set $\{(x, sin(\frac{1}{x})): x > 0\}$
        is path connected, but its closure is not.

        \item[c.] Let $f: X \rightarrow Y$ be continuous and suppose $X$ is path connected. Let $f(x), f(y) \in f(X)$.
        If $g$ is a path from $x$ to $y$ in $X$, then $f\circ g$ is a path from $f(x)$ to $f(y)$ in $f(X)$. Hence,
        $f(X)$ is path connected.

        \item[d.] Let $A_\alpha$ each be path connected subspaces of $X$ and suppose there is some $a \in \cap A_\alpha$.
        Then for any $x, y \in \bigcup A_\alpha$, we can make a path from $x$ to $y$ by joining the path from $x$ to $a$
        with the path from $a$ to $y$, which both exist by the connectedness of the $A_\alpha$s. This process is the same
        as in part (a.).
    \end{enumerate}
\end{enumerate}

\end{document}