\documentclass[11pt, reqno]{article}

\usepackage{amsmath, amsthm, amssymb}
\usepackage{enumitem}
\usepackage{tcolorbox}
\usepackage{hyperref}
\usepackage{tikz}
\usetikzlibrary{arrows.meta}
\usepackage{mathrsfs}
\usepackage{fancyhdr}
\usepackage[bottom=0.75in, top=1in, left=0.5in, right=0.5in]{geometry}
\usepackage{array}   % for \newcolumntype macro
\newcolumntype{L}{>{$}l<{$}}

\theoremstyle{plain}
\newtheorem*{theorem}{Theorem}
\newtheorem*{proposition}{Proposition}
\newtheorem{exercise}{Exercise}
\newtheorem*{lemma}{Lemma}
\newtheorem*{corollary}{Corollary}

\theoremstyle{definition}
\newtheorem*{definition}{Definition}
\newtheorem*{example}{Example}

\theoremstyle{remark}
\newtheorem*{remark}{Remark}

\renewcommand{\phi}{\varphi}
\renewcommand{\epsilon}{\varepsilon}
\renewcommand{\emptyset}{\varnothing}

\newcommand{\RR}{\mathbb{R}}
\newcommand{\ZZ}{\mathbb{Z}}
\newcommand{\NN}{\mathbb{N}}
\newcommand{\CC}{\mathbb{C}}
\newcommand{\QQ}{\mathbb{Q}}

\DeclareMathOperator{\ima}{\text{im}}

\begin{document}

\topmargin=-40pt
\rhead{Henry Woodburn}
\lhead{Math 636}
\renewcommand{\headrulewidth}{1pt}
\renewcommand{\headsep}{20pt}
\thispagestyle{fancy}

{\Huge \bfseries \noindent Homework 4}

\subsection*{Section 20}
\begin{enumerate}
    \item[5.] Let $\RR^\infty$ be the subset of $\RR^\omega$ consisting of sequences which
    are eventually $0$. Equip $\RR^\omega$ with the uniform topology induced by the 
    uniform metric $D$.

    \textbf{Claim:} The uniform closure of $\RR^\infty$ is $c_0$, the space of convergent 
    sequences in $\RR$.
    \bigbreak
    Because the closure of $\RR^\infty$ is the smallest closed set containing $\RR^\infty$, 
    to show that $\overline{\RR^\infty} \subset c_0$, we just need to show that $c_0$ is 
    closed in the uniform topology. To do this, we will show $c_0$ contains all of its 
    limit points. 

    Let $(x_n)$ be a limit point of $c_0$ and suppose $x_n \rightarrow a > 0$. Then choose $\epsilon = a/2 > 0$, and 
    we are guaranteed some $(y_n) \in c_0$ such that $D((x_n), (y_n)) < a/2$. But this implies $|y_i| > a/2 > 0$ for 
    all $i$, contradicting that $|y_i| \rightarrow 0$. Then we must have $(x_n) \in c_0$ and $c_0$ is closed.

    Conversely, in order to show $c_0 \subset \overline{\RR^\infty}$, it is enough to show that every point 
    in $c_0$ is a limit point of $\RR^\infty$. Take any $(y_n) \in c_0$, and let $\epsilon > 0$. Choose 
    $N > 0$ such that $|y_i| < \epsilon$ whenever $i \geq N$. Define a sequence in $\RR^\infty$ by
    \[
        (x_n) = \begin{cases} y_i & n < N\\ 0 & n \geq N\end{cases}.
    \]
    Clearly $(x_n)$ is eventually zero. 
    Then we have 
    \[
        D(x_n, y_n) \leq \sup\limits_i |x_i - y_i| = \sup\limits_{i \geq N} |y_i| \leq \epsilon,
    \]
    and thus every $\epsilon$-ball about $(x_n)$ intersects $\RR^\infty$ at a point other than $(x_n)$.

    Then $\overline{\RR^\infty} = c_0$.

    \item[7.] Let $(a_n)_1^\infty$ and $(b_n)_1^\infty$ be sequences of real numbers. Consider the map
    \begin{align*}
        h: & \RR^\omega \rightarrow \RR^\omega\\
        & (x_n) \mapsto (a_1 x_1 + b_1, a_2 x_2 + b_2, \dots)
    \end{align*}
    Let $f$ be the map $(x_n) \mapsto (a_1 x_1, a_2 x_2, \dots)$, and $g$ the map $(x_n) \mapsto (x_1 + b_1, x_2 + b_2,\dots)$.
    Then $h = g \circ f$, and in order to show $h$ is continuous we may consider $f$ and $g$ separately.
    \bigbreak
    For $g$, i claim there is no restriction of the values of $b_i$ in order for $g$ to be continuous. This 
    is because for any $g(x)$ in the image of $g$, the preimage of any open ball about $f(x)$ is just the open ball
    of the same radius about $x$. Then it is clear that $g$ is continuous.

    In order for $f$ to be continuous, I claim that the sequence of $|a_i|$'s must be bounded. First suppose this is 
    the case, and we will show that $f$ is indeed continuous. Let $a = \sup_i |a_i|$ and choose $\epsilon > 0$. 
    Let $(y_n) \in \RR^\omega$ such that $d((x_n), (y_n)) < \frac{\epsilon}{a}$. Then 
    \[
        |x_i - y_i| < \frac{\epsilon}{a}
    \]
    assuming $\frac{\epsilon}{a} < 1$, and thus 
    \[
        |f(x_i) - f(y_i)| = |a_i||x_i - y_i| \leq a|x_i - y_i| < \epsilon.
    \]
    Then $f$ is continuous, with $\delta = \frac{\epsilon}{a}$.

    Now suppose $|a_i|$ is unbounded. For any $\delta > 0$, choose some $(y_n) \in \RR^\omega$
    such that $|x_i - y_i| = \frac{\delta}{2}$. We see that 
    \[
        |f(x_i) - f(y_i)| = |a_i||x_i - y_i| = |a_i|\delta,
    \]
    and for any $N > 0$ there is some $i$ such that $|a_i| > N$ and $|f(x_i) - f(y_i)|$ is unbounded. Then 
    this implies $f$ cannot be continuous if we choose $0 < \epsilon < 1$.

    Together, as long as $|a_i|$ is bounded, the function $h$ will be continuous in the uniform metric. 

    \item[8.] (a.) Let $X \subset \RR^\omega$ be the space of square summable sequences. We will show that 
    on $X$ there are inclusions 
    \[
        \text{box topology}\ \supset\ \ell^2\ \text{topology}\ \supset\ \text{uniform topology}.
    \]
    Let $D$ be the uniform metric and $d$ be the $\ell^2$ metric.
    \bigbreak
    We will show the second inclusion first. Choose $\epsilon > 0$. Without loss of generality, consider
    a uniform $\epsilon$ ball $B_{u, \epsilon}$ about $0$, and say $\epsilon < 1$. This works because for any uniform
    open set $U$ and a point $x \in U$, we can translate $x$ to the origin. If $(y_n) \in X$ with $d((x_n), (y_n)) < \epsilon$,
    then 
    \[
        \sum |x_i|^2 < \epsilon^2,\ \text{so that}\ |x_i|^2 < \epsilon^2\ \text{and}\ |x_i| < \epsilon\ \text{for all}\ i.
    \]
    Then $D((x_n), (y_n)) < \epsilon$ and the $\ell^2$ epsilon ball of radius $\epsilon$ is contained in $B_{u, \epsilon}$.
    Then the uniform topology is contained in the $\ell^2$ topology.
    \bigbreak
    Now consider an $\ell^2$ ball $B_{2, \epsilon}$ of radius $\epsilon$ for some $\epsilon > 0$. Define an open 
    set $U$ containing $0$ in the box topology by 
    \[
        U = \prod\limits_{i = 1}^\infty (-\epsilon 2^{i - 1}, \epsilon 2^{i - 1}).
    \]
    Then if $(x_i) \in U$, we have 
    \[
        d((x_i),0)^2 = \sum_1^\infty |x_i|^2 \leq \sum_1^\infty \epsilon^2 2^i = \epsilon^2,
    \]
    and thus $d((x_i), 0) < \epsilon$ and $(x_n) \in B_{2, \epsilon}$. Then the $\ell^2$ topology is contained 
    in the box topology.

    (b.) We will show that the uniform, box, product, and $\ell^2$ topologies are all different on $\RR^\infty$ as a subspace
    of $X$. 

    \textbf{Box topology is distinct:}
    First, for $\epsilon > 0$,
    \[
        \prod\limits_{i = 1}^\infty (\epsilon, \epsilon)
    \]
    is an open set in the box topology which is not open in the product or $\ell^2$ topologies.

    The set 
    \[
        \prod\limits_{i = 1}^\infty U_i
    \]
    where $U_i = X_i$ for all $i$ except some $j$, where $U_j = (-1, 1)$ is not open in the uniform topology.

    \textbf{Product topology $\neq$ $\ell^2$ topology $\neq$ uniform topology:}

    The set $\{x \in \RR^\infty: d(x, 0) < \epsilon\}$ is open in the $\ell^2$ topology but not in the product topology,
    since an open set in the product topology cannot have infinitely many of its projections not all of $\RR$, but for every
    $i$, we must have $d(x_i) < \epsilon$.
    
    Likewise it is not open in the uniform topology, since every uniform open ball contains sequences of arbitrarily 
    large $\ell^2$ norm.

    \textbf{Product topology $\neq$ uniform topology:}
    The set $\{x \in \RR^\infty: D(x,0) < \epsilon\}$ is open in the uniform topology but is not open in the product topology,
    for the same reason as above. 

    Then all four topologies are distinct.
    
\end{enumerate}

\subsection*{Section 21}

\begin{enumerate}
    \item[1.] Let $d$ be a metric on $X$ and let $A \subset X$ be a subspace. We will show that the restriction of $d$
    to $A \times A$ induces the subspace topology that $A$ inherets from $X$. Let $d'$ be the restricted metric on $A$.
    \bigbreak
    First note that the collection of intersection of open balls in $X$ with $A$ forms a basis for the subspace topology
    on $A$. Then if $U$ is a basic open set in $A$, we have $U = B_\epsilon(x) \cap A$ for some epsilon ball 
    $B_\epsilon(x)$ centered at $x \in X$. For any $y \in U$, we have $d(x,y) < \epsilon$, and thus 
    the $d'$ ball about $y$ of radius $\epsilon - d(x,y)$ is contained in $U$. Then every set from the subspace 
    topology is open in the $d'$ metric topology. 

    Conversely let $B'_\epsilon(x)$ be a $d'$ ball of radius $\epsilon$ at some $x \in A$. Then
    $B'_\epsilon(x) = B_\epsilon(x) \cap A$, where $B_\epsilon(x)$ is a ball in the original metric. Thus $d'$ balls
    are open in the subspace topology and we are done. 

    \item[2.] First we show that $f$ is continuous. Choose $\epsilon > 0$. Then if $d_X(x,y) < \epsilon$, we have
    $d_Y(f(x), f(y)) = d_X(x,y) < \epsilon$, and we are done. 

    Now we show $f^{-1}$ is continuous as a map $f(X) \rightarrow X$. Suppose $d_Y(f(x), f(y)) < \epsilon$.
    Then $d_X(x,y) = d_Y(f(x), f(y)) < \epsilon$ and we are done. 

    Finally, we show $f$ is injective. This follows from $f$ being an isometry. If $f(x) = f(y)$, then 
    \[
        0 = d_Y(f(x), f(y)) = d_X(x,y),
    \]
    so $x = y$ since $d_X$ is a metric. 

    \item[7.] Let $X$ be a set and $f_n: X \rightarrow \RR$ a sequence of functions. We will show that uniform convergence
    of $f_n$ to $f$ is equivalent to convergence of $f_n$ to $f$ as elements of $\RR^X$ in the uniform topology. We can
    suppose $f_n$ converges to $0$ without loss of generality, otherwise take $f_n - f$.

    First suppose $f_n \rightarrow 0$ uniformly on $X$. Then for all $\epsilon > 0$ there exists $N > 0$
    such that $|f_n(x)| < \epsilon$ for all $x \in X$ and $n > N$. Then with $d$ the uniform metric,
    \[
        d(f_n, 0) \leq \sup\limits_{x \in X} |f_n(x)| < \epsilon
    \]
    and thus $f_n$ converges to $0$ in the uniform topology as an element of $\RR^X$.

    Conversely consider $f_n$ as an element of $\RR^X$ and suppose it converges in the uniform topology to $0$. 
    Choose $0 < \epsilon < 1$ and $N > 0$ such that $d(f_n(x), 0) < \epsilon$ for $n > N$. Then
    \[
        |f_n(x)| < \epsilon < 1
    \]
    for all $x \in X$ and $n > N$, so $f_n$ converges uniformly as well.
\end{enumerate}

\subsection*{Section 22}

\begin{enumerate}
    \item[2.] (a.) Let $p: X \rightarrow Y$ be a continuous map and suppose there is a continuous map $f: Y \rightarrow X$ such that 
    $p\circ f$ is the identity on $Y$. We will show $p$ is a quotient map.

    For surjectivity of $p$, for any $y \in Y$, since $p(f(y)) = y$, the point $f(y)$ maps to $y$ under $p$.

    Let $U \subset Y$. We already have that if $U$ is open in $Y$, $p^{-1}(U)$ must be open in $X$, since $p$ is continuous. Now suppose 
    $p^{-1}(U)$ is open in $X$. Then $f^{-1}(p^{-1}(U))$ is open in $Y$. But this is just $U$, since the inverse 
    of $p \circ f$, $f^{-1} \circ p^{-1}$, must also be the identity (as maps of sets).

    (b.) Let $A \subset X$ and let $r: X \rightarrow A$ be a retraction. $r$ is clearly surjective since 
    it is the identity on $A$. 

    Let $U \subset A$. We only need to show $r^{-1}(U)$ open implies $U$ open since $r$ is continuous. In this case, 
    we have $U = r^{-1}(U) \cap A$, and thus $U$ is open in the subspace topology on $A$.

    \item[4.] (a.) Define an equivalence relation on $\RR^2$ by 
    \[
        x_0 \times y_0 \sim x_1 \times y_1\ \text{if}\ x_0+y_0^2 = x_1 + y_1^2,
    \]
    and let $X^*$ be the quotient space. Then $X^*$ is homeomorphic to the real line $\RR$. To see this define 
    a map 
    \begin{align*}
        g: & \RR^2 \rightarrow \RR\\
        & x\times y \mapsto x + y^2.
    \end{align*}
    Then the fibers of $g$ are exactly the equivalence classes under 
    the relation above. Then by $22.3$, $X^*$ is homeomorphic to $\RR$ if $g$ is a quotient map. We can already see 
    that $g$ is continuous and surjective. Also, $g$ is an open map because is it the composition of the map $x\times y 
    \mapsto x - y^2 \times y$ and projection onto the $y$ axis. Then it is a quotient map. 

    (b.) Define $g: \RR^2 \rightarrow \RR_{\geq 0}$ by $x\times y \mapsto x^2 + y^2$. Then $g$ is continuous and surjective,
    and is an open map as it maps open sets to the segment of $\RR_{\geq 0}$ obtained by collecting all the 
    points intersected by sweeping a ray from $0$ around $360$ degrees and then squaring.

\end{enumerate}

\end{document}