\documentclass[11pt, reqno]{article}

\usepackage{amsmath, amsthm, amssymb}
\usepackage{enumitem}
\usepackage{tcolorbox}
\usepackage{hyperref}
\usepackage{tikz}
\usepackage{tikz-cd}
\usetikzlibrary{arrows.meta}
\usepackage{mathrsfs}
\usepackage{fancyhdr}
\usepackage[bottom=0.75in, top=1in, left=0.5in, right=0.5in]{geometry}
\usepackage{array}   % for \newcolumntype macro
\newcolumntype{L}{>{$}l<{$}}

\theoremstyle{plain}
\newtheorem*{theorem}{Theorem}
\newtheorem*{proposition}{Proposition}
\newtheorem{exercise}{Exercise}
\newtheorem*{lemma}{Lemma}
\newtheorem*{corollary}{Corollary}

\theoremstyle{definition}
\newtheorem*{definition}{Definition}
\newtheorem*{example}{Example}

\theoremstyle{remark}
\newtheorem*{remark}{Remark}

\renewcommand{\phi}{\varphi}
\renewcommand{\epsilon}{\varepsilon}
\renewcommand{\emptyset}{\varnothing}

\newcommand{\RR}{\mathbb{R}}
\newcommand{\ZZ}{\mathbb{Z}}
\newcommand{\NN}{\mathbb{N}}
\newcommand{\CC}{\mathbb{C}}
\newcommand{\QQ}{\mathbb{Q}}

\DeclareMathOperator{\ima}{\text{im}}

\begin{document}

\topmargin=-40pt
\rhead{Henry Woodburn}
\lhead{Math 636}
\renewcommand{\headrulewidth}{1pt}
\renewcommand{\headsep}{20pt}
\thispagestyle{fancy}

{\Huge \bfseries \noindent Homework 9}

\subsection*{Section 51}

\begin{enumerate}
    \item[1.] Let $h, h': X \rightarrow Y$ and $k, k': Y \rightarrow Z$ each be homotopic
    pairs of maps. Let $F: X \times I \rightarrow Y$ and $G: Y \times I \rightarrow Z$ be
    homotopies from $h$ to $h'$ and $k$ to $k'$ respectively. 

    Then define a function
    \begin{align}
        H: & X \times I \rightarrow Z\\
        &(x,t) \mapsto G(H(x,t),t).
    \end{align}
    $H$ is continuous since it is a composition of continuous functions. It is a path 
    homotopy from $k \circ h$ to $k' \circ h'$ since we have $H(x,0) = G(H(x,0),0) = G(h(x), 0) = (k \circ h)(x)$,
    and $H(x, 1) = G(H(x,1),1) = k'\circ h'(x)$. 

    \item[2.] (a.) Let $[X,I]$ be the set of homotopy classes of maps $X \rightarrow I$. 
    Let $f: X \rightarrow I$ be a continuous map. Then define 
    \[
        F(x,t) = (1-t)f(x).
    \]
    This is a path homotopy from $f$ to the constant path at $0$: We have $F(x,0) = f(x)$, $F(x,1) = 0$.

    Then the only element of $[X,I]$ is the class $[e]$, where $e(x) = 0$. 

    (b.) Let $f$ and $g$ be two paths in $Y$, a path connected space. Then connect $f(1)$ with $g(0)$ by a 
    path $h$. Then $f*h*g$ is a path containing $f$ and $g$, and there is a clear homotopy between 
    the two by moving the time interval $[0,1]$ along the path from $f$ to $g$. 
\end{enumerate}

\subsection*{Section 52}

\begin{enumerate}
    \item[2.] Let $\alpha$ be a path from $x_0$ to $x_1$ and let $\beta$ be a path from $x_1$ to $x_2$. 
    Let $\gamma = \alpha * \beta$. Then we have 
    \[
        \hat{\gamma}([f]) = [\overline{\gamma}]*[f]*[\gamma] = [\overline{\beta}]*[\overline{\alpha}]*
        [f]*[\alpha]*[\beta] = \hat{\beta}([\overline{\alpha}]*
        [f]*[\alpha]) = \hat{\beta}\circ\hat{\alpha}([f]),
    \] 
    verifying that $\hat{\gamma} = \hat{\beta}\circ\hat{\alpha}$

    \item[3.] Let $X$ be a path connected space with $x_0, x_1 \in X$. We will show
    that $\pi_1(X, x_0)$ is abelian if and only if for every pair of paths $\alpha, \beta$ from $x_0$ to $x_1$,
    we have $\hat{\alpha} = \hat{\beta}$. 

    First suppose the second condition holds. Then 
    \[
        \hat{\alpha}([\beta]) = [\hat{\alpha}]*[\beta]*[\alpha] = \hat{\beta}([\beta]) = \beta,
    \]
    and hence $[\alpha]*[\beta] = [\beta]*[\alpha]$.

    Conversely suppose $\pi_1(X, x_0)$ is abelian. Then 
    \[
        \hat{\alpha}([f]) = [\overline{\alpha}]*[f]*[\alpha] = [\overline{\alpha}]
        *[f]*[\alpha]*[\overline{\beta}]*[\beta] = [\overline{\alpha}]
        *[\alpha]*[\overline{\beta}]*[f]*[\beta] = [\overline{\beta}]*[f]*[\beta]
        = \hat{\beta}([f])
    \]
    since $[\alpha]*[\overline{\beta}]$ is a loop at $x_0$ and thus commutes with $[f]$. 
    
    \item[4.] Let $r$ be a retraction of $X$ onto a subspace $A$. Let $a_0 \in A$. Then the map
    \[
        r_*: \pi_1(X, a_0) \rightarrow \pi_1(A, a_0)
    \]
    is a surjection. 

    Take a loop $\tilde{f}$ in $\pi_1(A, a_0)$. Then consider this as a map $f$ from $I$ to $X$. Then
    we have $r \circ f = f$, since $f(t)$ is in $A$ for all $t \in [0,1]$. 

    \item[5.] Let $A$ be a subspace of $\RR^n$ and let $h: (A, a_0) \rightarrow (Y, y_0)$. Suppose $h$ can be extended 
    to a map $\tilde{h}$ with domain all of $\RR^n$. 

    Let $f \in \pi_1(A, a_0)$. Then the function
    \[
        \tilde{h}((1-t)f(s) + ta_0)
    \]
    is a path homotopy from $h\circ f$ to $h(a_0) = y_0$, and hence the image of every element of $\pi_1(A, a_0)$
    under $h$ is homotopic to the identity element, and $h$ is a trivial homomorphism. 
\end{enumerate}

\end{document}