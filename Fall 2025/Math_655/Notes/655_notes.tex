\documentclass[11pt, reqno]{article}

\usepackage{amsmath, amsthm, amssymb}
\usepackage{enumitem}
\usepackage{tcolorbox}
\usepackage{hyperref}
\usepackage{tikz}
\usetikzlibrary{arrows.meta}
\usepackage{mathrsfs}
\usepackage{fancyhdr}
\usepackage[bottom=0.75in, top=1in, left=0.5in, right=0.5in]{geometry}
\usepackage{array}   % for \newcolumntype macro
\newcolumntype{L}{>{$}l<{$}}

\theoremstyle{plain}
\newtheorem*{theorem}{Theorem}
\newtheorem*{proposition}{Proposition}
\newtheorem{exercise}{Exercise}
\newtheorem*{lemma}{Lemma}
\newtheorem*{corollary}{Corollary}

\theoremstyle{definition}
\newtheorem*{definition}{Definition}
\newtheorem*{example}{Example}

\theoremstyle{remark}
\newtheorem*{remark}{Remark}

\renewcommand{\phi}{\varphi}
\renewcommand{\epsilon}{\varepsilon}
\renewcommand{\emptyset}{\varnothing}

\newcommand{\RR}{\mathbb{R}}

\begin{document}

\topmargin=-40pt
\rhead{Henry Woodburn}
\lhead{Math 655}
\renewcommand{\headrulewidth}{1pt}
\renewcommand{\headsep}{20pt}
\thispagestyle{fancy}

{\Huge \bfseries \noindent 655 Notes}

\subsection*{Product Topology}

Let $\Gamma$ be a set and $(X_\gamma, \tau_\gamma)_{\gamma \in \Gamma}$ a collection of topological
spaces. The Product topology on $\prod_{\gamma \in \Gamma}X_\gamma$ is defined as the weakest topology on 
$\prod_{\gamma \in \Gamma}X_\gamma$ which makes the projection maps $\pi_\gamma: \prod_{\gamma \in \Gamma} X_\gamma$
continuous.

\begin{example}
    On $\mathbb{R}^\Gamma$, the product topology is given by the following neighborhood basis:
    \[
    \{U(x; \gamma_1, \dots, \gamma_n; \epsilon): \gamma_1, \dots,\gamma_n \in \Gamma, \epsilon > 0, n \geq 1, x \in \mathbb{R}^\gamma\},
    \]
    where $U(x; \gamma_1, \dots, \gamma_n; \epsilon) := \{z \in \mathbb{R}^\Gamma: |z_{\gamma_i} - x_{\gamma_i}| < \epsilon,
    1 \leq i \leq n\}$.

    $\mathbb{R}^\Gamma$ with the product topology is hausdorff.
\end{example}

\subsection*{Locally Convex Topological Vector Spaces}

\begin{definition}
    A \textbf{topological vector space} is a vector space $X$ equipped with a topology $\tau$ such that the maps 
    \[
    \begin{split}
        A: & X\times X \rightarrow X\\
        & (x_1, x_2) \mapsto x_1 + x_2
    \end{split} \qquad 
    \begin{split}
        \Omega: & \mathbb{R}\times X \rightarrow X\\
        & (a, x) \mapsto ax
    \end{split}
    \]
    are both continuous. 

    A TVS is locally convex if every point has a local base consisting of convex sets.
\end{definition}

\begin{example}
    An arbitrary product of LCTVS's is an LCTVS with the product Topology. A vector subspace of an LCTVS is 
    an LCTVS when given the relative topology.
\end{example}

\subsection*{Dual Pairs}

Let $E$ be a vector space and let $E^{\#} := \{f: E \rightarrow \RR: f\ \textrm{is linear}\}$ be the algebraic dual space.

Let $E$ and $F$ be vector spaces. Then a bilinear form $\langle \cdot, \cdot \rangle: E \times F \rightarrow \RR$ induces 
two maps:
\begin{equation*}
    \begin{split}
        \varphi: &E \rightarrow F^\#\\
            &e \mapsto f \mapsto \langle e, f\rangle
    \end{split} \qquad 
    \begin{split}
        \psi: & F \rightarrow E^\#\\
            & f \mapsto e \mapsto \langle e,f \rangle.
    \end{split}
\end{equation*}

\begin{definition}
    A dual pair is a pair of vector spaces $E, F$ and a bilinear map $\langle \cdot, \cdot \rangle: E \times F \rightarrow \RR$
    such that 
    \begin{enumerate}
        \item[a.)] $E$ separates points in $F$, meaning for all $f_1, f_2 \in E$, $f_1 \neq f_2$, there is an $e \in E$
        such that $\langle e, f_1\rangle \neq \langle e, f_2\rangle$.
        \item[b.)] $F$ separates points in $E$.
    \end{enumerate}
    We write $\langle E, F\rangle$ is a dual pair. 
\end{definition}

\begin{remark}
    The statement that $E$ separates points in $F$ is equivalent to the statement that for $f \in F$, if for all $e \in E$, 
    $\langle e, f \rangle = 0$, then $f = 0$. Then $\psi$ is an injection, and we can identify $F$ with its image 
    in under $\psi$ in $E^\#$

    The dual statement is that if $F$ separates points in $E$, we can identify $E$ with its image under $\varphi$ in $F^\#$.
\end{remark}

\begin{example}
    Given a vector space $E$, $\langle E, E^\# \rangle$ is a dual pair for $\langle \cdot, \cdot \rangle: 
    E\times E^\# \rightarrow \RR$ given by $(e, e^\#) \mapsto e^\#(e)$.
\end{example}

\begin{example}
    Given a normed vector space $X$, $\langle X, X^*\rangle$ is a dual pair for $\langle \cdot, \cdot, \rangle: 
    X\times X^* \rightarrow \RR$ given by $(x,x^*) \mapsto x^*(x)$.
\end{example}

\begin{definition}
    Let $\langle E,F\rangle$ be a dual pair. The weak topology associated to the dual pair, denoted by $\sigma(E,F)$, is 
    defined as the restriction to $E$ of the product topology on $\RR^F$. 
\end{definition}

\begin{remark}
    We showed that we can view $E$ as a subset of $F^\#$ by the injection $\varphi$. $F^\#$ is a subset of $\RR^F$, the space
    of all maps $F \rightarrow \RR$, consisting of those maps which are linear. Then we can view $E$ as a subset of $\RR^F$. 
\end{remark}

\begin{example}
    Let $X$ be a normed vector space and consider the dual pair $\langle X, X^*\rangle$, with $\langle e, e^*\rangle = e^*(e)$.
    The topology $\sigma(X,X^*)$ on $X$ is called the weak topology. The topology $\sigma(X^*, X)$ on $X^*$ is called
    the weak$^*$ topology. 
\end{example}

We now give some equivalent definitions for the weak topology in the case that $X$ is a normed vector space
and $\langle X, X^*\rangle$ is our dual pair. 

\subsubsection*{Weak Topology} 

The weak topology on $X$ is given by: 
\begin{itemize}
    \item The topology generated by the sets 
    \begin{align*}
    U(x_0; x_1^*, \dots, x_n^*; \epsilon) &= \{x \in X: |\langle x_0, x_i^*\rangle - \langle x, x_i^*\rangle| < \epsilon, 1 \leq i \leq n\}\\
    & = \{x \in X: |x_i^*(x_0) - x_i^*(x)| < \epsilon, 1 \leq i \leq n\}
    \end{align*}
    
    \item If $\{x_\alpha\}_{\alpha}$ is a net in $X$ and $x \in X$, then $x_\alpha \rightarrow x$ weakly if and only if 
    for all $x^* \in X^*$, $x^*(x_\alpha) \rightarrow x^*(x)$

    \item the weakest topology on $X$ which makes all of the bounded linear functionals on $X$ continuous.
\end{itemize}

\subsubsection*{Weak$^*$ Topology}

The weak$^*$ topology on $X^*$ is given by 
\begin{itemize}
    \item the topology generated by sets 
    \[
    U(x_0^*; x_1, \dots, x_n; \epsilon) = \{x^* \in X^*: |x_0^*(x_i) - x^*(x_i)| < \epsilon, 1 \leq i \leq n\}
    \]

    \item $x_\alpha^* \rightarrow x^*$ in the weak$^*$ topology if and only if $x_\alpha^*(x) \rightarrow x^*(x)$ for 
    all $x \in X$
    
    \item the weakest topology on $X^*$ for which the maps $x^* \rightarrow x^*(x)$ are continuous for every $x \in X$.
\end{itemize}

\begin{remark}
    The map $i: (X^*, \sigma(X^*, X)) \rightarrow \RR^X, x^* \mapsto {(x^*(x))}_{x \in X}$ is a homeomorphism from $(X^*, \sigma(X^*, X))$ onto its
    image in $\RR^X$ with the product topology.

    We have $x_\alpha^* \rightarrow x^*$ in the weak$^*$ topology if and only if for all $x \in X$, $x_\alpha^*(x) \rightarrow x^*(x)$,
    if and only if $i(x_\alpha^*) \rightarrow i(x^*)$ in the product topology. 
\end{remark}

\begin{remark}
    The map $j: (X, \sigma(X, X^*)) \rightarrow X^{**} \subset \RR^{X^*}, x \mapsto (x^*(x))_{x^* \in X^*}$ is a homeomorphism
    from $(X, \sigma(X, X^*))$ onto its image in $(X^{**}, \sigma(X^{**}, X^*))$. 

    We have $x_\alpha \rightarrow x$ weakly if and only if for all $x^* \in X^*$, $x^*(x_\alpha) \rightarrow x^*(x)$
    if and only if $j(x_\alpha) \rightarrow j(x)$ in the weak$^*$ topology on $X^{**}$. 
\end{remark}

\begin{proposition}
    Let $X$ be a normed space. 
    \begin{enumerate}
        \item $(X, \sigma(X, X^*))^* = X^*$
        \item $(X^*, \sigma(X^*, x))^* = j(X)$
    \end{enumerate}
\end{proposition}


\textit{Proof.} (1.) We have $(X, \sigma(X, X^*))^* \subset X^*$ because $\sigma(X, X^*)$ is weaker than the norm 
topology, thus every functional which is weak-continuous is also norm-continuous. That $X^* \subset (X, \sigma(X, X^*))$
follows by construction, since $\sigma(X, X^*)$ ensures that each functional which is norm-continuous is also
$\sigma(X, X^*)$ continuous.

(2.) We have $j(X) \subset (X^*, \sigma(X^*, X))^*$ by construction, since $\sigma(X^*, X)$ is a topology 
such that the maps $j(x)$ are continuous. 

To show the other direction, let $\phi: (X^*, \sigma(X^*, X)) \rightarrow \RR$ be a weak$^*$ continuous functional
on $X^*$. Since $\phi$ is continuous, there is a weak$^*$ neighborhood $U \ni 0$ in $X^*$ such that 
$U \subset \phi^{-1}(-1, 1)$.

From one of the above characterizations of the weak$^*$ topology, we know that there must be elements $x_1, \dots, x_n$ 
such that $U = \{x^*: |x^*(x_i)| < \epsilon\ \textrm{for} 1 \leq i \leq n\}$. Now suppose $f^* \in \bigcap_1^n \ker x_i$.
In particular, we have $|f^*(x_i)| = 0 < \epsilon$ for $i = 1, \dots, n$, thus $f^* \in U$. Then for any $\lambda > 0$, 
$|\lambda f(x_i)| = \lambda 0 = 0 < \epsilon$ for $i = 1, \dots, n$, thus $\lambda f^* \in U$, and we have
$|\phi(\lambda f^*)| < 1$ and thus $|\phi(f^*)| < 1/\lambda$. 

Since this holds for all $\lambda > 0$, it must be that $\phi(f^*) = 0$ and $f^* \in \ker\phi$. We have therefore shown
that $\ker\phi \subset \bigcap_1^n \ker x_i$. Then linear algebra tells us that $\phi$ must be a linear combination of
the functionals $x_i$, $\phi = \sum_1^n a_i x_i := x$. Then $j(x) = \phi$ \hfill $\qed$

\begin{theorem}[Banach-Alaoglu Theorem]
    Let $X$ be a normed vector space. Then $(B_{X^*}, \sigma(X^*, X))$ is a compact topological space.
\end{theorem}

\textit{Proof (outline)} Observe that for all $x \in X, x^* \in X^*$, $\|x^*(x)\| \leq \|x^*\|\|x\|$. Then $B_{X^*}$ embeds
in $\RR^X$ by the map 

\begin{align*}
    i: & B_{X^*} \rightarrow \prod_{x \in X} [-\|x\|, \|x\|] \subset \RR^X\\
    & x^* \mapsto (x^*)_{x \in X}.
\end{align*}

$K := \prod_{x \in X} [-\|x\|, \|x\|]$ is compact by Tychonoff's theorem. $i(B_{X^*})$ consists of only the elements
of $K$ that are linear. To finish show nets in $i(B_{X^*})$ converge to linear elements of $K$.\hfill $\qed$

\begin{theorem}
    If $X$ is reflexive, then $(B_X, \sigma(X, X^*))$ is compact. 
\end{theorem}

\subsection*{Hahn-Banach Theorems}

\begin{definition}
    Let $E$ be a vector space over $\RR$. A subset $A \subset E$ is called absorbing if for all $x \in E$, there
    exists $\lambda > 0$ such that $x \in \lambda A$.
\end{definition}

A neighborhood of $0$ in a topological vector space is absorbing: For all $x \in E$, the map
$\mu_\lambda: \RR \rightarrow E$ sending $\lambda$ to $\lambda x$ is continuous
and sends $0$ to $0$. Then if $V$ is a neighborhood of $0$ in $E$, there exists
$r > 0$ such that $(-r, r) \subset \mu_x^{-1}(V)$, and thus for all $|\lambda| < r$,
$\mu_x(\lambda) = \lambda x \in V$. 

\begin{definition}
    Let $A$ be an absorbing set in a topological vector space $E$. We define the 
    gauge, or Minkowski Functional, of $A$, denoted $\mu_A$, as follows:
    \begin{align*}
        \mu_A: & X \rightarrow \left[0,\infty\right) \\
        & x \mapsto \inf\{\lambda > 0: x \in \lambda A\}
    \end{align*}
\end{definition}

Notice that $\mu_A(0) = 0$. 

\begin{lemma}
    If $C$ is a convex absorbing subset, then 
    \begin{enumerate}
        \item[i.] $\mu_C$ is a sublinear functional 
        \item[ii.] $\{x \in E: \mu_C(x) < 1\} \subset C \subset \{x \in E: \mu_C(x) \leq 1\}$.
        \item[iii.] If $E$ is an LCTVS and $0 \in C^\circ$, then $\mu_C$ is continuous at $0$. 
    \end{enumerate}
\end{lemma}

\textit{Proof.} (i.) Let $x, y \in E$ and $\epsilon > 0$. By definition, there are $\lambda, \mu > 0$ 
such that $\lambda < \mu_C(x) + \epsilon$, $\mu < \mu_C(y) + \epsilon$ and $x \in \lambda C$, $y \in \mu C$.
Then 
\[
    \frac{x + y}{\lambda + \mu} = \frac{\lambda}{\lambda + \mu}\frac{x}{\lambda} + \frac{\mu}{\lambda + \mu}\frac{y}{\mu} \in C,
\]

so that $x + y \in (\lambda + \mu)C$ and $\mu_C(x + y) \leq \lambda + \mu \leq \mu_C(x) + \mu_C(y) + 2\epsilon$.
This shows subadditivity. Positive homogeneity is obvious after expanding the definition of $\mu_C$. 

(ii.) If $x \in C$, then $x = \frac{x}{1}$, which proves the second inclusion. For the first, if $ \mu_C(x) < 1$,
then for some $\lambda < 1$, we have $x \in \lambda C$. Since $C$ is convex, writing $x = \lambda \frac{x}{\lambda} + (1-\lambda ) 0$
shows that $x \in C$. 

(iii.) Since $x \in C^\circ$, there is a convex open neighborhood $U \ni 0$ in $C$. Let $\epsilon > 0$, then 
$\epsilon U$ is also an open neighborhood of $0$, and if $x_\alpha$ is a net in $E$ converging to $0$, 
then there exists $\alpha_0$ such that $x_\alpha \in \epsilon U$ for all $\alpha > \alpha_0$. 
Then $\mu_C(x_\alpha) \leq \mu_U(x_\alpha) \leq \epsilon$.

\subsubsection*{Geometric Hahn-Banach Separation Theorem for LCTVS}

\begin{theorem}
    Let $(X, \tau)$ be an LCTVS, $C$ a nonempty closed convex subset, and $x_0 \in X \setminus C$. Then
    there exists $x^* \in (X,\tau)^*$ such that 
    \[
        x^*(x_0) > \sup_{x \in C} x^*(x)
    \]
\end{theorem}

\textit{Proof.} WLOG, suppose $0 \in C$. Since $C$ is closed, $X \setminus C$ is open and there exists 
a convex neighborhood $U$ of $0$ such that $x_0 + U \subset X \setminus C$. Then take a convex neighborhood 
$V$ of $0$ such that $V - V \subset U$ by continuity of operations in a TVS. 

Let $D = C + V$ and observe that $(x_0 + V) \cap D = \emptyset$, and $D$ is convex and $0 \in D^\circ$. Need 
to write this step out to see how $V - V \subset U$ is used. 

Let $\mu_D$ be the gauge of $D$. Then for all $z \in x_0 + V$, $\mu_D(z) \geq 1$. Since $V$ is open, there 
is a $\lambda > 1$ such that $\lambda x_0 \in x_0 + V$ and in fact $\mu_D(x_0) > 1$. 

Now define 
\begin{align*}
    f: & \RR x_0 \rightarrow \RR\\
    &\alpha x_0 \mapsto \alpha \mu_D(x_0)
\end{align*}

and observe that $f$ is linear. Then for any $\alpha \geq 0$, we have 
\[
    f(\alpha x_0) = \alpha \mu_D(x_0) = \mu_D(\alpha x_0).
\]
Likewise if $\alpha < 0$ we have 
\[
    f(\alpha x_0) = \alpha \mu_D(x_0) \leq \mu_D(\alpha x_0),
\]
so that $f \leq \mu_D$ on $\RR x_0$. By the algebraic Hahn-Banach theorem, we can extend $f$ to a function 
$F: X \rightarrow \RR$ such that $F$ equals $f$ on the subspace $\RR x_0$, and $F \leq \mu_D$ on $X$. 
In particular, $x \in D$ implies $\mu_D(x) \leq 1$ and thus $F(x) \leq 1$ on $D$ and $F(x) \geq -1$ on $-D$.
Then we have $|F(x)| \leq 1$ on $D \cap (-D)$ and $F$ is continuous at $0$. 

The inequality holds since $F(x_0) \geq 1$ but $F(x) < 1$ for all $x \in D$. 

\subsubsection*{Applications}

\begin{theorem}[Goldstine's Theorem]
    Let $X$ be a normed space. Then 
    \[
        \overline{j(B_x)}^{\sigma(X^{**}, X^*)} = B_{X^{**}}.
    \]
    In particular, 
    \[
        \overline{j(X)}^{\sigma(X^{**}, X^*)} = X^{**}.
    \]
\end{theorem}

\textit{Proof.} First notice that 
\[
    \overline{j(B_x)}^{\sigma(X^{**}, X^*)} \subset B_{X^{**}}
\]  
since $B_{X^{**}}$ is weak$^*$ compact and hence closed. 

Next suppose $x_0 \in B_{X^{**}} \setminus \overline{j(B_x)}^{\sigma(X^{**}, X^*)}$. $\sigma(X^{**}, X^*)$ is a 
hausdorff LCVT, so we can apply geometric Hahn-Banach theorem to obtain $\phi \in (X, \sigma(X^{**}, X^*)) = j(X^*)$
such that $\phi(x_0) > \sup_{x \in \overline{j(B_x)}^{\sigma(X^{**}, X^*)}} \phi(x)$. 

Then since $\phi = j(x_0^*)$ for some $x_0^* \in X^*$, we have 
\[
\phi(x_0) > \sup_{x \in \overline{j(B_x)}^{\sigma(X^{**}, X^*)}} x(x_0^*) \geq \sup_{x \in j(B_X)} x(x_0) =
\sup_{x \in B_x} x_0^*(x) = \|x_0^*\|.
\]
However, $j(x_0^*)(x_0) = x_0(x_0^*) \leq \|x_0\|_{\sigma(X^{**}, X^*)} \|x_0^*\|_{X^*} \leq \|x_0\|_{X^*}$,
which shows $\|x_0^*\| < \|x_0^*\|$, a contradiction.

\begin{theorem}[Mazur's Theorem]
    Let $C$ be a convex subset of a normed space $X$. Then $\overline{C}^{\|\cdot\|} = \overline{C}^w$.
\end{theorem}

\textit{Proof.} We have $\overline{C}^{\|\cdot\|} \subset \overline{C}^w$ by definition. The intuition is that
since the weak topology is less restrictive, it allows more into the closure. 

Then suppose $x_0 \in \overline{C}^w \setminus \overline{C}^{\|\cdot\|}$.

By the Geometric Hahn-Banach theorem, there exists $x_0^* \in (X, \|\cdot\|)^*$ such that 
$x_0^*(x_0) > \sup_{x \in C} x_0^*(x)$. Now let $x_\alpha$ be a net in $C$ converging weakly to $x_0$. 
Then for all $x^* \in X^*$, $x^*(x_\alpha) \rightarrow x^*(x_0)$. In particular, $x_0^*(x_\alpha)
\rightarrow x_0^*(x_0)$. However, we have $x_0^*(x_\alpha) \leq \sup_{x \in C} x_0^*(x)$,
implying $x_0^*(x_0) \leq \sup_{x \in C} x_0^*(x) < x_0^*(x_0)$, a contradiction. 

\begin{theorem}[Eberlein-Smulian]
    Let $(X, \|\cdot\|)$ be a normed vector space. Then $A \subset X$ is (relatively) weakly compact if and only if 
    $A$ is (relatively) weakly sequentially compact. 
\end{theorem}

\begin{remark}
    \begin{enumerate}
        \item The weak topology on $X$ is metrizable iff $X$ is finite dimensional
        \item The weak topology on $X$ is not $1st$ countable
        \item $(B_X, \sigma(X, X^*))$ is metrizable iff $X^*$ is separable
        \item $(B_{X^*}, \sigma(X^*, X))$ is metrizable iff $X$ is separable.
    \end{enumerate}
\end{remark}

\begin{lemma}
    Let $(X, \|\cdot\|)$ be a normed space. If $X$ is separable, then there exists a norm on $X$ that induces
    a topology that is weaker than the weak topology on the unit ball.
\end{lemma}

\textit{Proof of Lemma.} Let $\{x_n\}$ be a dense sequence in $B_X$. Choose $x_n^* \in B_X$ such that 
$x_n^*(x_n) = \|x_n\|$ using algebraic Hahn-Banach theorem. Let $p(x) = \sum_1^\infty \frac{1}{2^n}|x_n^*(x)|$,
taking values in $\left[0,\infty\right)$. Check that $p$ is a sublinear functional. Assume that $p(x) = 0$ 
and $\|x\| \leq 1$. Let $i \geq 1$ such that $\|x - x_i\| < \epsilon$. Then 

\[
    \|x_i\| = |x_i^*(x_i)| = |x_i^*(x - x_i)| \leq \|x_i - x\| < \epsilon
\]

Now let $r > 0$ and consier $\{x \in B_X: p(x) < r\}$. Let $V = \{x \in B_X: |x_i^*(x)| < \epsilon, 1 \leq i \leq N\}$.
We can choose $\epsilon$ small so that the first $N$ terms of $p(x)$ sum to less than $r/2$ and $N$ large 
so that the remaining terms sum to less than $r/2$.\hfill\qed

\bigbreak
\textit{Proof of Eberlein-Smulian ($\Rightarrow$)} Since $X$ is a normed vector space and $A$ relatively weakly compact, 
every sequence in $A$ has a subsequence which is convergent in $X$. 

Let $K = \overline{A}^{\sigma(X,X^*)}$. Then $K$ is weakly compact. Let $a_n \in A$ and define
$Z := \overline{\text{span}\{a_n\}}^{\|\cdot\|} \subset X$. $Z$ is a separable subspace of $X$. 

Let $K_0 = \overline{\{a_n\}}^{\sigma(X, X^*)}$. Note that $K_0 \subset Z$, since $Z$ is a convex
set and is thus also weakly closed by Mazur's theorem. Also $K_0$ is a wealky closed subset of $K$, which is 
compact, thus $K_0$ is weakly compact. 

In fact $K_0$ is $\sigma(Z, Z^*)$ compact by Hahn-Banach extension theorem, since every linear functional on 
$Z$ extends to one on $X$. 

Note that $K_0$ is weakly compact and hence bounded in $Z$. 
By the previous lemma, there is a norm $\rho$ on $Z$ which induces a topology on $K_0$ which is 
weaker than the weak topology. 

The $\rho$ topology actually coincides with $\sigma(Z, Z^*)$ on $K_0$. This is because if 
$\tau_1 \subset \tau_2$ are both topologies, with $\tau_1$ hausdorff and $\tau_2$ compact, then 
$\tau_1 = \tau_2$. Then $K_0$ is metrizable, so $a_n$ has a subsequence which is weakly convergent
in $Z$. \hfill\qed

\begin{definition}
    Let $A \subset (X, \|\cdot\|)$. We say that $A$ is weakly bounded if for all $x^* \in X^*$, 
    the set $x^*(A)\subset \RR$ is bounded.
\end{definition}

\begin{remark}
    Every originally bounded subset is also weakly bounded.
\end{remark}

\begin{lemma}
    If $A$ is weakly bounded, then $A$ is norm bounded.
\end{lemma}

\textit{Proof.} Consider linear maps $T_a: X^* \rightarrow \RR$, where $x^* \mapsto x^*(a)$ for $a \in A$. 
Then $\|T_a\| = \|a\|$. Since $A$ is weakly bounded, for each $x^* \in X*$ we have 
$\sup_{a \in A} |T_a(x^*)| < \infty$. Then the Uniform Boundedness Principle implies that 
$\sup_{a \in A} \|T_a\| < \infty$ and thus $\sup_{a \in A}\|a\| < \infty$. \hfill\qed

\begin{corollary}
    If $A \subset (X, \|\cdot\|)$ is (relatively) weakly compact OR (relatively) weakly
    sequentially compact, then $A$ is norm-bounded.
\end{corollary}

\textit{Proof.} Prof. only sketched. Prove by contradiction.

\begin{lemma}
    Let $(X, \|\cdot\|)$ be a normed space and $E \subset X^*$ a finite dimensional subspace. Then there 
    exists a finite subset $F \subset X$ such that for all $x^* \in E$, we have 
    \[
        \frac{\|x^*\|}{2}\leq \max\limits_{x \in F}|x^*(x)| \leq \|x^*\|
    \]
\end{lemma}

\textit{Proof.} Since $E$ is finite dimensional, the unit sphere $S_E$ is compact. Then we can 
choose a finite $\eta$-net $\{x_1^*, \dots, x_N^*\}$ such that for all $x^* \in S_E$, there
is some $i \in \{1, \dots, N\}$ such that $\|x^* - x_i^*\| < \eta$. For each $i$, choose
$x_i \in B_X$ such that $|x_i^*(x_i)| > 1 - \eta$.

Then for any $x^* \in E$, choose $i \in \{1, \dots, N\}$ such that $\left\|\frac{x^*}{\|x^*\|} - x_i^*\right\| < \eta$.
Then we have 
\[
    \left|\frac{x^*}{\|x^*\|}(x_i)\right| = \left|\left(\frac{x^*}{\|x^*\|} - x_i^*\right)(x_i) + x_i^*(x_i)\right|
    \geq |x_i^*(x_i)| - \left|\left(\frac{x^*}{\|x^*\|} - x_i^*\right)(x_i)\right| \geq 1 - \eta - \eta,
\]
using the reverse triangle inequality. Then take $\eta = 1/4$.

\bigbreak
\textit{Proof of Eberlein-Smulian ($\Leftarrow$)} Our first observation is that $A$ is bounded by the above corollary. 
The second and main observation is that if $A \subset X$ is bounded, then $\overline{A}^{\sigma(X, X^*)}$ is compact
if and only if $\overline{J(A)}^{\sigma(X^{**}, X^*)} \subset J(X)$.

To prove the only if, first we have that $j(\overline{A}^{\sigma(X,X^*)})$ is $\sigma(X^{**}, X^*)$ 
compact since $J$ is weak to weak$^*$ continuous. Then $j(\overline{A}^{\sigma(X,X^*)})$ is closed
since the weak$^*$ topology is hausdorff. Then since $A \subset \overline{A}^{\sigma(X, X^*)}$, we have 
$\overline{J(A)}^{\sigma(X^{**}, X^*)} \subset j(\overline{A}^{\sigma(X, X^*)})$.

For the other direction, $A$ bounded implies $j(A)$ bounded, so $\overline{j(A)}^{\sigma(X^{**}, X^*)}$ 
is $\sigma(X^{**}, X^*)$-compact by Banach-Alaoglu. Now if $\smash{\overline{j(A)}^{\sigma(X^{**}, X^*)}}
\subset J(X)$, the $\sigma(X^{**}, X^*)$ topology restricted to $J(X)$ coincides with the weak topology on $X$
and thus $\overline{A}^{\sigma(X, X^*)}$ is weakly compact. 

Now we begin the proof. Let $x_0^{**} \in \overline{J(A)}^{\sigma(X^{**}, X^*)}$. Our goal will be to show
that there is some $x_0 \in X$ such that $x_0^{**} = J(x)$. We will construct a sequence $\{a_n\} \subset A$ and 
$\{x_n^*\} \subset B_{X^*}$ inductively. 

Begin by taking $x_1^* \in S_{X^*}$ and consider the $\sigma(X^{**}, X^*)$ neighborhood 
$V = \{x^{**} \subset X^{**}: |x^{**}(x_1^*) - x_0^{**}(x_1^*)| < 1\}$ of $x_0^{**}$. 
Since $x_0^{**} \in \overline{J(A)}^{\sigma(X^{**}, X^*)}$, there is $a_1 \in A$ such that $J(a_1) 
\in V$ and hence $|J(a_1)(x_1^*) - x_0^{**}(x_1^*)| < 1$.

Now $E_1 := \operatorname{span}\{x_0^{**}, x_0^{**} - J(a_1)\}$ is a finite dimensional subspace of $X^*$, so by the lemma
there is a finite sequence $x_2^*, \dots, x_{n_2}^* \in B_{X^*}$ such that for all $x^{**} \in E_1$, 
\[
    \frac{\|x^{**}}{2} \leq \max\limits_{2 \leq i \leq n_2} |x^{**}(x_i^*)| \leq \|x^{**}\|
\]

Then in a similar fashion to above, there is some $a_2 \in A$ such that 
\[
    \left|J(a_2)(x_i^*) - x_0^{**}(x_i^*)\right| < \frac{1}{2}
\]
for all $1 \leq i \leq n_2$. By the lemma there exist $x_{n_2 + 1}^*, \dots, x_{n_3}^* \in B_{X^*}$ such that 
for all $x^{**} \in \operatorname{span}\{x_0^{**}, x_0^{**} - j(a_1), x_0^{**} - j(a_2)\}$, we have 
\[
    \frac{\|x^{**}\|}{2} \leq \max\limits_{n_2 + 1 \leq i \leq n_3} |x^{**}(x_i^*)| \leq \|x^{**}\|.
\]

Continue inductively to obtain sequences $\{a_n\} \subset A$ and $\{x_n^*\} \subset B_{X^*}$, such that 
\begin{enumerate}
    \item for all $x^{**} \in \operatorname{span}\{x_0^{**}, x_0^{**} - J(a_1), x_0^{**} - J(a_2), \dots\}$,
    \[
        \frac{\|x^{**}\|}{2} \leq \sup\limits_{i \geq 1} |x^{**}(x_i)| \leq \|x^{**}\|
    \]

    \item $|J(a_k)(x_i^*) - x_0^{**}(x_i^*)| < \frac{1}{k}$ for all $1 \leq i \leq n_k$.
\end{enumerate}

Since $A$ is relatively weakly sequentially compact, there is some $x \in X$ and a subsequence $\{a_{n_k}\}$ 
converging to $x$ in the $\sigma(X, X^*)$ topology.

Note that by Mazur's theorem, $x \in \overline{\operatorname{span}\{a_n: n \geq 1\}}$. Hence 
$x_0^{**} - j(x) \in \overline{\operatorname{span}\{x_0^{**} - J(a_n): n \geq 1\}} =: Z$. 
This needs to be verified. Then for any $z^{**} \in Z$, we have 
\[
    \frac{\|z^{**}\|}{2} \leq \sup\limits_{i \geq 1} |z^{**}(x_i^*)|
\]
by a continuity argument.

In particular, 
\[
    \frac{\|x_0^{**} - J(x)\|}{2} \leq \sup\limits_{i \geq 1} |(x_0^{**} - J(x))(x_i^*)|.
\]
Finally we will show this last term must be zero. Let $i \geq 1$. Then 
\begin{align*}
    |(x_0^{**} - J(x))(x_i^*)| \leq |(x_0^{**} - J(a_k))(x_i^*)| + |(J(a_k) - J(x))(x_i^*)| \leq \epsilon/2 + \epsilon/2
\end{align*}
by choosing $k$ large enough that the second term is small by weak convergence, and the first is small by (2.) above,
such that $a_k > i$.

\subsection*{Reflexive Spaces}

\begin{definition}
    A normed space is called reflexive if the canonical map 
    \begin{align*}
        J: & X \rightarrow X^{**}\\
        & x \mapsto (x^* \mapsto x^*(x)) =: \langle J(x), x^*\rangle
    \end{align*}
\end{definition}

\begin{remark}
    A reflexive space is always a Banach space.
\end{remark}

The obvious examples are the spaces $\ell_p$ and $L_p([0,1])$ for $1 < p < \infty$.

\subsubsection*{Topological Characterization of Reflexivity}

\begin{theorem}
    Let $X$ be a Banach space. $X$ is reflexive if and only if $B_X$ is $\sigma(X, X^*)$ compact.
\end{theorem}

\textit{Proof.} The forward direction is immediate by Banach-Alaoglu theorem. 

For the other direction, if $(B_X, \sigma(X, X^*))$ is compact, then $J(B_X)$ is $\sigma(X^{**}, X^*)$ compact.
Then $J(B_X)$ is closed since $\sigma(X^{**}, X^*)$ is a hausdorff topology. But by Goldstine's theorem,
$J(B_X) = \overline{J(B_X)}^{\sigma(X^{**}, X^*)} = B_{X^{**}}$. Then $J(B_X) = B_{X^{**}}$, implying
that $J(X) = X^{**}$.

\begin{corollary}
    Let $X$ be a Banach space. If $X$ is reflexive, then 
    \begin{enumerate}
        \item $X^*$ is reflexive
        \item Every closed subspace of $X$ is reflexive
        \item Every $x^* \in X^*$ attains its norm
        \item $Y$ is reflexive whenever $Y$ is isomorphic to $X$
        \item Every bounded sequence in $X$ has a weakly convergent subsequence.
    \end{enumerate}
\end{corollary}

\textit{Proof.} (1.) Assume $X$ is reflexive. Then $(B_{X^*}, \sigma(X^*, X^{**})) \simeq (B_{X^*}, \sigma(X^*, X))$,
and since the second space is compact, the unit ball in $X^*$ is weakly compact and thus $X^*$ is reflexive.

(2.) Let $X$ be reflexive and $Y$ be a closed subspace. By assumption, $(B_X, \sigma(X, X^*))$ is compact. 
The restriction of $\sigma(X, X^*)$ to $Y$ is $\sigma(Y, Y^*)$. Therefore, $(B_Y, \sigma(Y, Y^*))$ is compact because it is 
a $\sigma(X, X^*)$ closed subset of $B_X$. 

(3.) Compactness argument.

(4.) Assume there exists $T: X \rightarrow Y$ such that $1/C\|x\| \leq \|Tx\| \leq C\|x\|$ for some $C> 0$. 
Then $\frac{1}{C} B_Y \subset T(B_X) \subset C B_Y$. We have that $(B_X, \sigma(X, X^*))$ is compact. Since $T$ is 
weak to weak continuous, $T(B_X)$ is $\sigma(Y, Y^*)$ compact. Finally, since $\frac{1}{C}B_Y$ is a $\sigma(Y,Y^*)$ 
closed subset of a $\sigma(Y, Y^*)$ compact set, it is also $\sigma(Y, Y^*)$ compact. 

(5.) $x_n$ bounded implies $x_n \subset c B_X$ for some $c$. Since the unit ball is weakly compact and thus weakly 
sequentially compact by Eberlein-Smulian, there is a weakly convergent subsequence. \hfill \qed

\begin{proposition}
    If $X^*$ is reflexive, then $X$ is reflexive. 
\end{proposition}

\textit{Proof.} The above corollary implies that $X^{**}$ is reflexive in this case. Then $J(X)$ is a closed subspace
of $X^{**}$ and thus $J(X)$ and $X$ are reflexive.

\subsection*{Sequential/Geometric Characterization of Reflexivity}

\begin{theorem}
    Let $X$ be a Banach space. The following are equivalent:
    \begin{enumerate}
        \item $X$ is not reflexive.
        \item For all $\theta \in (0,1)$, there exists a sequence $\{x_n\} \subset B_X$ and $\{x_n^*\} \subset B_{X^*}$ 
        such that $x_n^*(x_k) = 0$ if $k < n$ and $\theta$ if $k \geq n$.
        \item For all $\theta \in (0,1)$, there exists a sequence $\{x_n\} \subset B_X$ such that for all $k > 1$,
        \[d(\operatorname{conv}\{x_1, \dots, x_k\}, \operatorname{conv}\{x_{n+1}, \dots\}) \geq \theta\]
    \end{enumerate}
\end{theorem}

\subsubsection*{Moment Problem}

Let $(X, \|\cdot\|)$ be a normed vector space. Let $x_1^*, \dots, x_n^* \in X^*$ and $c_1, \dots, c_n \in \RR$. 
Does there exist $x \in X$ such that $x_i^*(x) = c_i$ for all $1 \leq i \leq n$. 

\begin{theorem}[Helly's Theorem]
    Let $x_1^*,\dots, x_n^* \in X^*$ , $c_1,\dots, c_n \in \RR$, and $k > 0$. Then the following are equivalent:
    \begin{enumerate}
        \item For all $\epsilon > 0$, there exists $x_\epsilon \in X$ such that $\|x_\epsilon\| \leq k + \epsilon$
        and $x_i^*(x_\epsilon) = c_i$ for $1 \leq i \leq n$.
        \item For all $a_1,\dots, a_n \in \RR$, 
        \[
            \left|\sum_1^N a_i c_i\right| < k\left\| \sum_1^N a_i x_i^*\right\|
        \]
    \end{enumerate}
\end{theorem}

\textit{Proof.} $(1 \Rightarrow 2)$
\[
    \left|\sum a_i c_i\right| = \left| \sum a_i x_i^*(x_\epsilon)\right| = \|x_\epsilon\|\left\|\sum_1^N a_i x_i^*\right\|
    \leq (x + \epsilon)\left\|\sum a_i x_i^*\right\|
\]

$(2 \Rightarrow 1)$ Without loss of generality, suppose not all $c_i = 0$. Say $c_{i_0} \neq 0$. Also suppose 
not all $x_i^*$ are zero. 

Therefore we can assume $x_1^*, \dots, x_k^*$ are linearly independent for $k \leq n$. Thus for all $1 \leq i \leq n$,
$x_i^* = \sum_1^k \alpha_j^{(i)} x_j^*$.

Given this assumption, if we show $2$ holds for the linearly independent elements, there is an argument to show
that it holds for the rest of the elements. See notes.

Then we can assume $x_1, \dots, x_n$ are linearly independent. 

Consider $T: X \rightarrow \RR^n$, $x \mapsto (x_1^*(x), \dots, x_n^*(x))$. $T$ is linear. Because the $x_i^*$ are linearly
independent, for all $1 \leq k \leq n$ we have $\bigcap_{i \neq k} \ker x_i^* \subset \ker(x_k^*)$. 

For all $1 \leq k \leq n$, there exists $y_k \in \bigcap \ker x_i^* \setminus \ker x_k^*$ such that $x_k^*(y_k) = 1$,
and $x_j^*(y_k) = 0$ for all $j \neq k$. Let $y = \sum_1^n c_j y_j$. Then $x_i^*(y) = \sum_1^n c_j x_j^*(y_j) = c_i$.

Let $Z := \bigcap_1^n \ker x_i^*$, a set closed in $X$. By Hahn-Banach, there exists $x^* \in X^*$ such that 
$x^*(y) = d(y,z)$. Since $\ker(x^*) \supset Z$, we have $x^* = \sum_1^n \alpha_i x_i^*$. 

$d(y,Z) = x^*(y) = \sum_1^n \alpha_i x_i^*(y) = \sum \alpha_i c_i \leq k\left\|\sum_1^n \alpha_i x_i^*\right\|$.

Fix $\epsilon > 0$. There exists $z \in Z$ such that $\|y - z\| \leq (k + \epsilon)\left\|\sum_1^n \alpha_i x_i^*\right\| = \|x^*\| = 1$.

Then $x_\epsilon = y - z$ satisfies $\|x_\epsilon\| \leq k + \epsilon$.

See notes for proof of sequential characterization of Reflexivity.

\begin{theorem}
    Let $X$ be a Banach space. The following are equivalent:
    \begin{enumerate}
        \item X is reflexive
        \item Every bounded sequence in $X$ has a weakly convergent subsequence 
        \item If $(C_n)$ is a nonincreasing sequence of nonempty, bounded, closed, convex sets, 
        then $\cap C_n \neq \emptyset$
    \end{enumerate}
\end{theorem}

\textit{Proof.} $(1 \rightarrow 2)$ follows by Eberlein-Smulian. 

$(2 \rightarrow 3)$: Let $x_n \in C_n$ for all $n \geq 1$. $(x_n)$ is bounded and hence there exists 
$x \in X$ such that $x_{n_k} \rightarrow x$ weakly for some $n_k$. 

Claim: $x \in \cap_1^\infty C_n$. Assume otherwise. Then there is some $n_0$ such that $x \notin C_{n_0}$.
By geoemtric Hahn-Banach, there exists $x^* \in S_{X^*}$ such that $x^*(x) > \sup_{C_{n_0}} x^*(z)$. 
Note that $x^*(x) = \lim x^*(x_{n_k})$. But there exists $K \geq 1$ such that for all $k \geq K$, 
$x_{n_k} \in C_{n_0}$ and hence $x^*(x_{n_k}) \leq \sup_{C_{n_0}} x^*(z)$ for all $k \geq K$. 
Then we have a contradiction.

$(3 \rightarrow 1)$: Assume $X$ is not reflexive. Then apply the sequential characterization
to obtain some $\theta \in (0,1), x_n \in B_X, and x_n^* \in B_{X^*}$. Consider 
$C_n = \overline{\operatorname{conv}\{x_k: k \geq n\}}$. Then $C_n$ is nonincreasing, nonempty,
closed, and bounded. We claim $\cap C_n = \emptyset$. Suppose not. Then let $x \in \cap C_n$ and 
observe that for all $\epsilon > 0$ and all $k \geq 1$, there exists $y \in C_n$ such that 
$\|x - y\| < \epsilon$, where $y = \sum_1^{n_\epsilon} \lambda_i x_i$ for $\lambda_1 > 0$ 
and where $y$ is a convex combination. 

Then for all $n > n_\epsilon$, we have $|x_n^*(x - y)| = |x_n^*(x)| < \epsilon$
and thus $\lim_{n \rightarrow \infty} x_n^*(x) = 0$. But because $x \in C_k$ for 
$k \geq 1$, we have $x_k^*(x) \geq \theta/2$,a contradiction. 

\subsection*{Finite Representability}

\begin{definition}
    Let $X, Y$ be Banach spaces and $\lambda \geq 1$. 
    \begin{enumerate}
        \item $Y$ is $\lambda$-finitely representable in $X$ if for every 
        finite dimensional subspace $E \subset Y$, there exists an isomorphism 
        $T: E \rightarrow X$ such that $\|T\|\|T^{-1}\| \leq \lambda$. In other words,
        there exists $k \geq 0$ with $k^2 \leq \lambda$ such that for all $e \in E$, 
        \[
            \frac{\|e\|}{k} \leq \|Te\| \leq k\|e\|.
        \]

        \item $Y$ is finitely representable in $X$ if it is $(1 + \epsilon)$-finitely representable 
        in $X$ for all $\epsilon > 0$. 
    \end{enumerate}
\end{definition}

\begin{example}
    \begin{enumerate}
        \item $L_p([0,1])$ is finitely representable in $\ell_p$ for $1 \leq p < \infty$.
        \item Every Banach space is finitely representable in any Banach space which contains 
        the $\ell_\infty^n$'s, such as $c_0, \ell_\infty, C([0,1])$, etc.
    \end{enumerate}
\end{example}

\begin{lemma}
    \begin{enumerate}
        \item If $X_1$ is $\lambda_1$ finite representable in $X_2$ and $X_2$ is $\lambda_2$ finite 
        representable in $X_3$, then $X_1$ is $\lambda_1 \lambda_2$ finite representable in $X_3$
        
        \item If $X_1$ is finitely representable in $X_2$ and $X_2$ is finitely representable 
        in $X_3$, then $X_1$ is finitely representable in $X_3$
    \end{enumerate}
\end{lemma}

\begin{definition}
    Let $P$ be a property of Banach spaces. We say that a Banach space $X$ has super-$P$ if 
    every Banach space that is finitely representable in $X$ has $P$.
\end{definition}

\begin{remark}
    \begin{enumerate}
        \item Super-$P$ implies $P$
        \item Super-super-$P$ is equal to super-$P$
    \end{enumerate}
\end{remark}

\begin{theorem}
    If $X$ is a Banach space, then $X^{**}$ is finitely representable in $X$.
\end{theorem}

The proof of this theorem is a consequence of the principle of local reflexivity.

\subsubsection*{Ultraproducts}

\begin{definition}
    Let $I$ be a set, $\mathcal{U} \in \beta I$ an untrafilter on $I$, and $(X_i)_{i \in I}$ a collection
    of Banach spaces. The Ultraproduct of $(X_i)$ with respect to $\mathcal{U}$ is 
    $(\prod_{i \in I} X_i)_{\mathcal{U}} := \ell_\infty(I; (X_i)_i)/N_\mathcal{U}$, 

    where $\ell_\infty(I; (X_i)i) := \{(x_i)_{i \in I}: \forall i \in I, x_i \in X_i, \sup \|x_i\|_{X_i} < \infty\}$
    equipped with the sup norm, and $N_\mathcal{U} := \{(x_i)_{i \in I} \subset \ell_\infty(I, (x_i)_i):
    \lim_{i, \mathcal{U}} \|x_i\|_{X_i} = 0\}$.
\end{definition}

One important notion used here is that of a limit along an ultrafilter. If $f:I \rightarrow (X,\tau)$,
we say $\lim_{i, \mathcal{U}} f(i) = x$ iff for all neighborhoods $V$ of $x$, we have $f^{-1}(V) \in \mathcal{U}$.

\begin{lemma}
    If $(x_i)_\mathcal{U} \in (\prod_{i \in I} X_i)_\mathcal{U}$, then 
    \[
        \|(x_i)_\mathcal{U}\|_\mathcal{U} = \lim_{i, \mathcal{U}}\|x_i\|_{X_i}
    \]
    where the first norm is the quotient norm.
\end{lemma}

\begin{lemma}
    If $\mathcal{U} \in \beta I$ is non-principle, then $\RR^{\mathcal{U}}$ is linearly isomorphic
    to $\RR$. 
\end{lemma}

\begin{theorem}
    Let $X$ be a Banach space, $\mathcal{U} \in \beta I$ non-principle. Then $X^{\mathcal{U}}$ is finitely 
    representable in $X$.
\end{theorem}

See notes for proof.

\begin{theorem}
    Let $X$ be a Banach space, $E \subset X^{**}$, $F \subset X^*$ both finite dimensional subspaces. For all $\epsilon > 0$,
    there exists an injective operator $T: X \rightarrow X$ such that 
    \begin{enumerate}
        \item $Tx = x$ for all $x \in E \cap X$
        \item $\|T\|\|T^{-1}\| \leq 1 + \epsilon$
        \item For all $x^{**} \in E$ and $x^* \in F$, $x^{**}(x^{*}) = x^*(Tx^{**})$.
    \end{enumerate}
\end{theorem}

\begin{corollary}
    $X^{**}$ is finitely representable in $X$.
\end{corollary}

\begin{lemma}[Helly's Lemma]
    Let $X$ be a Banach space and $G \subset X^*$ finite dimensional. For all $x^{**} \in X^*$ and $\epsilon > 0$,
    there exists $x \in X$ such that 
    \begin{enumerate}
        \item $\|x\| \leq (1 + \epsilon)\|x^{**}\|$
        \item For all $x^* \in G$, $x^{**}(x^*) = x^*(x) = J(x)(x^{**})$.
    \end{enumerate}
\end{lemma}

\textit{Proof of Helly's lemma} 

Let $G = \operatorname{span}\{x^*_i: 1 \leq i \leq n\}$ and let $c_i = x^{**}(x_i^*)$ for $1 \leq i \leq n$.

Choose any $(\alpha_i)_1^\infty \subset \RR$. Then we have 
\[
    \left|\sum_1^n \alpha_i c_i\right| = \left|\sum_1^n \alpha_i x^{**}(x_i)\right| \leq \|x^{**}\|\left\|\sum_1^n \alpha_i x^*\right\|
\]
Then the conditions for Helly's theorem are satisfied, and there exists $x \in X$ such that $\|x\| \leq (1 + \epsilon)\|x^{**}\|$
and $x_i^*(x) = c_i = x^{**}(x_i^*)$ for $1 \leq i \leq n$. Then by linearity, we can extend this to 
\[
    \text{for all}\ x^* \in G, x^*(x) = x^{**}(x^*)
\]
\hfill\qed

\begin{remark}
    Let $X$ be a real vector space. Then there is a canonical identification of $L(\RR, X)$ with $X$, where 
    $f \mapsto f(1)$. Then we have $L(\RR,X)^{**} \equiv X^{**} \equiv L(\RR, X^{**})$. 
\end{remark}

Thus, we can restate Helly's Lemma as follows:
\begin{lemma}[Generalized Helly's Lemma]
    Let $X$ be a Banach space, $E \subset X$ and $F \subset X^*$ finite dimensional subspaces. For all $S \in L(E, X^{**})$
    and all $\epsilon > 0$, there exists $T \in L(E, X)$ such that 
    \[
        \|T\| \leq (1 + \epsilon)\|S\|
    \]
    and for all $x^* \in F$ and $e \in E$, $(Se)(x^*) = x^*(Te)$.
\end{lemma}

\textit{Proof of GHL}
For all $x \in E$ and $x^* \in F$, define 
\begin{align*}
    x \otimes x^*: L(E, X)& \rightarrow \RR\\
    A &\rightarrow x^*(Ax)
\end{align*}

Then $x \otimes x^*$ is linear, and we have $\|x\otimes x^*\| \leq \|x\|\|x^*\|$. Then
$x\otimes x^* \in L(E, X)^*$ for all $x \in E, x^* \in F$.
\bigbreak
Define $G:= \operatorname{span}\{x\otimes x^*: x \in E, x^* \in F\}$, a finite dimensional subspace of $L(E, X)^*$.
By Helly's lemma on $G \subset L(E, X)^*$, for all $S \in L(E, X)^{**}$ and $\epsilon > 0$, there exists 
$T \in L(E, X)$ such that $\|T\|\leq (1 + \epsilon)\|S\|$, and 
\[
    \text{for all}\ R \in G, S(R) = R(T),
\]
i.e. for all $x \in E$ and $x^* \in F$, $S(x\otimes x^*) = (x \otimes x^*)(T) = x^*(Tx)$.
\bigbreak
Then if for all $\tilde{S} \in L(E, X)^{**}$ we can find an $S \in L(E, X^{**})$ such that 
\[
    (\tilde{S}e)(x^*) = S(e\otimes x^*)\ \text{for all}\ e \in E, x^* \in F,
\]
we will be done. We will save the proof of this fact until after proving PLR.

\textit{Proof of PLR}

Let $E \subset X^{**}, F \subset X^*$ be finite dimensional subspaces. Consider 
\begin{align*}
    S: E & \rightarrow X^{**}\\
    x^{**} & \mapsto x^{**}.
\end{align*}
By Generalized Helly's Lemma, for all $\epsilon > 0$, there exists some bounded operator $T: E \rightarrow X$ 
such that 
\begin{enumerate}
    \item $\|T\| \leq (1 + \epsilon)\|S\|\leq (1 + \epsilon)$
    \item For all $x^* \in F$ and $e \in E$, $(Se)(x^*) = x^*(Te)$.
\end{enumerate}
The problem is that we do not know if $T$ is injective or if $\|T\| \leq 1$.
\bigbreak
We need to enlarge $F$ in order to get these results. Let $\delta > 0$ and choose a $\delta/2$ net 
$\{x_1^{**}, \dots, x_n^{**}\} \subset S_E$, and let $x_i^* \in S_{X^*}$ such that $x_i^{**}(x_i^*) \geq 1 - \delta$
for $1 \leq i \leq n$. Let $\tilde{F} = \operatorname{span}\{F \cup \{x_i^*: 1 \leq i \leq n\}\}$, and observe that 
$\tilde{F}$ is also a finite dimensional subspace of $X^*$. Now apply GHL to $\tilde{F}$. Then for all $\epsilon > 0$,
there exists $T \in L(E, X)$ such that
\begin{enumerate}
    \item $\|T\| \leq (1 + \epsilon)$
    \item For all $x^* \in \tilde{F}$ and all $x^{**} \in E$, $x^{**}(x^*) = x^*(Tx^{**})$
\end{enumerate}
\bigbreak
Next we will prove that $T$ is the identity on $E$. Let $x \in E$. Then for all $x^* \in F$, 
$x^*(Tx - x) = 0$. Assume by contradiction that $Tx \neq x$. Pick some $x_{i_0}^{**} \in S_E$ such that 
\[
    \left\|\frac{Tx - x}{\|Tx - x\|} - x_{i_0}^{**}\right\| \leq \delta.
\]
But then 
\begin{align*}
    1 - \delta \leq |x^{**}_{i_0}(x_{i_0}^*)| = \left|\left(x_{i_0}^* - \frac{Tx - x}{\|Tx-x\|}\right)(x_{i_0}^*)\right|
    \leq \left\|x_{i_0}^* - \frac{Tx - x}{\|Tx - x}\right\|\|x_{i_0}^*\| < \delta
\end{align*}

\end{document}