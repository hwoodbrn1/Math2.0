\documentclass[11pt, reqno]{article}

\usepackage{amsmath, amsthm, amssymb}
\usepackage{enumitem}
\usepackage{tcolorbox}
\usepackage{hyperref}
\usepackage{tikz}
\usepackage{tikz-cd}
\usetikzlibrary{arrows.meta}
\usepackage{mathrsfs}
\usepackage{fancyhdr}
\usepackage[bottom=0.75in, top=1in, left=0.5in, right=0.5in]{geometry}
\usepackage{array}   % for \newcolumntype macro
\newcolumntype{L}{>{$}l<{$}}

\theoremstyle{plain}
\newtheorem*{theorem}{Theorem}
\newtheorem*{proposition}{Proposition}
\newtheorem{exercise}{Exercise}
\newtheorem*{lemma}{Lemma}
\newtheorem*{corollary}{Corollary}

\theoremstyle{definition}
\newtheorem*{definition}{Definition}
\newtheorem*{example}{Example}

\theoremstyle{remark}
\newtheorem*{remark}{Remark}

\renewcommand{\phi}{\varphi}
\renewcommand{\epsilon}{\varepsilon}
\renewcommand{\emptyset}{\varnothing}

\newcommand{\RR}{\mathbb{R}}
\newcommand{\ZZ}{\mathbb{Z}}
\newcommand{\NN}{\mathbb{N}}
\newcommand{\CC}{\mathbb{C}}
\newcommand{\QQ}{\mathbb{Q}}

\DeclareMathOperator{\ima}{\text{im}}

\begin{document}

\topmargin=-40pt
\rhead{Henry Woodburn}
\lhead{Math 655}
\renewcommand{\headrulewidth}{1pt}
\renewcommand{\headsep}{20pt}
\thispagestyle{fancy}

{\Huge \bfseries \noindent Homework 8}

\begin{enumerate}
    \item[1.] Suppose $\ell_1$ does not have the Schur property: then without loss of generality, there is a sequence $\{x_n\}$ which converges 
    weakly to $0$ but
    not in norm. Then by the Bessage-pelczynski selection principle, there is a subsequence $\{x_{n_k}\}$ which
    is equivalent to a block-basic sequence of the standard basis of $\ell_1$, which in turn is equivalent
    to the standard basis of $\ell_1$. Then $[\{x_{n_k}\}]$ is isomorphic to $\ell_1$.

    We want to show there is a bounded linear functional $\phi$ given by
    \[
        x = \sum_1^\infty a_k x_{n_k} \mapsto_\phi \sum_1^\infty a_k.
    \]

    Then we need to show this map is well defined, i.e. that the series converges.

    Note that 
    \[
        \frac{1}{C}\left\|\sum_1^\infty a_k e_k\right\| \leq \left\|\sum_1^\infty a_k x_{n_k}\right\|
        \leq C \left\|\sum_1^\infty a_k e_k\right\|
    \]

    where $\{e_k\}$ is the standard basis of $\ell_1$. Then 

    \[
        \left|\ \phi\left(\sum_1^\infty a_k x_{n_k}\right)\right| \leq \sum_1^\infty |a_k| = \left\|\sum_1^\infty a_k e_k\right\|
        \leq C \left\|\sum_1^\infty a_k x_{n_k}\right\|,
    \]

    which shows $\phi$ is well defined and bounded, and linearity is obvious. 

    Then we can extend this to a functional $\phi \in \ell^\infty$ by Hahn-Banach. But then we have 
    \[
        \phi(x_{n_k}) = 1
    \]
    for all $k$, contradicting the weak convergence of $\{x_{n_k}\}$ to zero. So $\ell_1$ must have the schur property.

    \item[2.] Let $X$ be a Banach space with the Schur property and suppose $A$ is a weakly compact subset. 
    By Eberlein-Smulian, $A$ is also sequentially weakly compact, and thus norm sequentially compact, 
    since every weak convergent subsequence is also norm convergent. Then since $X$ is a metric space,
    sequential norm compactness is equivalent to norm compactness, and thus $A$ is compact.

    Conversely suppose $A$ is a norm compact subset of $X$. Since every weak open cover is also a norm open
    cover, we can always reduce to a finite subcover. Thus $A$ is weakly compact.

    \item[3.] Let $X$ be a reflexive Banach space with the Schur property. By the topological characterization of 
    reflexivity, the unit ball in $X$ is weakly compact, and thus norm compact by problem 2. Then $X$ must
    be a finite dimensional space, since the unit ball in a normed vector space is compact if and only if 
    the space is finite dimensional. 

    \item[4.] Let $X$ be a reflexive space and suppose $T: X \rightarrow \ell_1$ is a bounded operator. 
    Since $X$ is reflexive, the unit ball is weakly compact. Note that $T$ is also weakly bounded. Then 
    the image of the unit ball $T(B_X)$ is weakly compact in $\ell_1$, and thus norm compact by problem 
    $2$. Since $X$ is hausdorff, this is a stronger condition than the closure $\overline{T(B_X)}$ being 
    compact, and thus $T$ is a compact operator. 

\end{enumerate}

\end{document}