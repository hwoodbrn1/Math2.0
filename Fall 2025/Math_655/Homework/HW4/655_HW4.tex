\documentclass[11pt, reqno]{article}

\usepackage{amsmath, amsthm, amssymb}
\usepackage{enumitem}
\usepackage{tcolorbox}
\usepackage{hyperref}
\usepackage{tikz}
\usetikzlibrary{arrows.meta}
\usepackage{mathrsfs}
\usepackage{fancyhdr}
\usepackage[bottom=0.75in, top=1in, left=0.5in, right=0.5in]{geometry}
\usepackage{array}   % for \newcolumntype macro
\newcolumntype{L}{>{$}l<{$}}

\theoremstyle{plain}
\newtheorem*{theorem}{Theorem}
\newtheorem*{proposition}{Proposition}
\newtheorem{exercise}{Exercise}
\newtheorem*{lemma}{Lemma}
\newtheorem*{corollary}{Corollary}

\theoremstyle{definition}
\newtheorem*{definition}{Definition}
\newtheorem*{example}{Example}

\theoremstyle{remark}
\newtheorem*{remark}{Remark}

\renewcommand{\phi}{\varphi}
\renewcommand{\epsilon}{\varepsilon}
\renewcommand{\emptyset}{\varnothing}

\newcommand{\RR}{\mathbb{R}}
\newcommand{\ZZ}{\mathbb{Z}}
\newcommand{\NN}{\mathbb{N}}
\newcommand{\CC}{\mathbb{C}}
\newcommand{\QQ}{\mathbb{Q}}

\DeclareMathOperator{\ima}{\text{im}}

\begin{document}

\topmargin=-40pt
\rhead{Henry Woodburn}
\lhead{Math 655}
\renewcommand{\headrulewidth}{1pt}
\renewcommand{\headsep}{20pt}
\thispagestyle{fancy}

{\Huge \bfseries \noindent Homework 4}

\begin{enumerate}
    \item Let $(x_n)_1^\infty \in \ell^\infty$. We will show that 
    \[
        d((x_n)_1^\infty, c_0) = \limsup\limits_n |x_n|.
    \]

    By definition, 
    \[
        d((x_n)_1^\infty, c_0) = \inf\limits_{(y_n) \in\ c_0} d((x_n), (y_n)),
    \]
    where $d\left((x_n), (y_n)\right) = \sup\limits_n |x_n - y_n|$. 
    
    To show that the distance $d((x_n), c_0)$ is at most $\liminf\limits_n |x_n|$, define a family of 
    sequences in $c_0$ as follows: Let 
    \[
        (y_n^k) := \begin{cases}
            x_n & n < k\\
            0   & n \geq k
        \end{cases},
    \]
    so that 
    \[
        d((x_n), (y_n^k)) = \sup\limits_n |x_n - y_n^k| = \sup\limits_{n \geq k} |x_n|.
    \]
    Then we have
    \[
        d((x_n), c_0) \leq \inf\limits_k d((x_n), (y_n^k)) = \inf\limits_k \sup\limits_{n \geq k} |x_n| = \liminf\limits_n |x_n|,
    \]
    which proves one direction. On the other hand, for any $(y_n) \in c_0$, 
    \[
        d((x_n), (y_n)) = \sup\limits_n |x_n - y_n| \geq \sup\limits_n |x_n| - |y_n| = \sup\limits_n |x_n|
        \geq \limsup\limits_n |x_n|,
    \]
    and thus $d((x_n), c_0) = \inf\limits_{(y_n) \in c_0} d((x_n), (y_n)) \geq \limsup\limits_n |x_n|$. 
    Then we have shown $d((x_n), c_0) = \limsup\limits_n |x_n|$.

    \item Let $\mathcal{U}$ be a non-principal ultrafilter on a set $I$, $(X_i)_{i \in I}$ a collection 
    of Banach spaces, and $(\prod_{i \in I} X_i)^{\mathcal{U}}$ its ultraproduct with respect to $\mathcal{U}$.

    We will show that for some $(x_i)_{i \in I}$, 
    \[
        \|(x_i)\|_{\mathcal{U}} = \lim\limits_{i, \mathcal{U}}\|x_i\|_{X_i}.
    \]

    Let $\lim\limits_{i, \mathcal{U}} \|x_i\|_{X_i} = a$. By the definition of the ultrafilter limit, 
    if we choose some $\epsilon > 0$ we get a set $U \in \mathcal{U}$ such that for $i \in U$,
    $|\|x_i\|_{X_i} - a| < \epsilon$. Then we know $\|x_i\| < a + \epsilon$ for all $i \in U$.
    
    Define a sequence 
    \[
        y_i = \begin{cases}x_i & i \notin U\\ 0 & i \in U\end{cases}
    \]
    so that 
    \[
        d((x_n),(y_n)) = \sup\limits_{i} \|x_i - y_i\| = \sup\limits_{i \in U} \|x_i\| < a + \epsilon.
    \]
    We know that $(y_n)_{i \in I}$ is an element of $N_\mathcal{U}$ since for any $\epsilon > 0$, the set of $i \in I$ for which 
    $\|y_i\|_{X_i} < \epsilon$ contains the set $U$ and thus is an element of $\mathcal{U}$. 

    Moreover, 
    \[
        \|(x_i)\|_{\mathcal{U}} = \inf\limits_{(z_n) \in N_\mathcal{U}} d((x_n), (z_n)) \leq d((x_n), (y_n)) < a + \epsilon.
    \]

    Then this holds for any $\epsilon > 0$, so in fact $\|(x_i)\|_\mathcal{U} \leq a = \lim\limits_{i, \mathcal{U}}\|x_i\|_{X_i}$.
    \bigbreak
    In the other direction, let $(y_n) \in N_{\mathcal{U}}$ and choose $\epsilon > 0$. Then there is a set 
    $U \in \mathcal{U}$ such that for $i \in U, \|y_i\| < \epsilon$, and a set $V \in \mathcal{U}$ such that 
    for $i \in V, \|x_i\| \geq a - \epsilon$. So
    \[
        d((x_n), (y_n)) = \inf\limits_i \|x_i - y_i\| \geq \sup\limits_{i \in U \cap V} \|x_i\| - \|y_i\|
        \geq a - 2\epsilon.
    \]
    Then $d((x_n), (y_n)) \geq \lim\limits_{i, \mathcal{U}} \|x_i\|$ for any $(y_n) \in N_\mathcal{U}$,
    and thus 
    \[
        \|(x_i)\|_{\mathcal{U}} = \inf\limits_{(y_n) \in N_\mathcal{U}} d((x_i), (y_n)) \geq \lim\limits_{i, \mathcal{U}} \|x_i\|
    \]
    and we are done.

    \item We will show that $\RR \simeq \RR^\mathcal{U}$, the ultrapower of $\RR$ with respect to $\mathcal{U}$.
    
    Define a map $\Phi: \RR^\mathcal{U} \rightarrow \RR$ by $(x_i) \mapsto \lim\limits_{i, \mathcal{U}}|x_i|$. The map
    is linear because of the linearity of the limit. It is an isometry since
    \[
        \|(x_i)\|_\mathcal{U} = \lim\limits_{i, \mathcal{U}}|x_i| = |\lim\limits_{i, \mathcal{U}}x_i|
    \]
    as $\lim\limits_{i, \mathcal{U}}x_i$ always exists. It is clearly surjective, since for any $x \in \RR$ 
    we can take $(x_n) = x$ for all $n$. Then the two spaces are isometrically isomorphic.
\end{enumerate}

\end{document}