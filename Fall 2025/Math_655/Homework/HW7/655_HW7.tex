\documentclass[11pt, reqno]{article}

\usepackage{amsmath, amsthm, amssymb}
\usepackage{enumitem}
\usepackage{tcolorbox}
\usepackage{hyperref}
\usepackage{tikz}
\usepackage{tikz-cd}
\usetikzlibrary{arrows.meta}
\usepackage{mathrsfs}
\usepackage{fancyhdr}
\usepackage[bottom=0.75in, top=1in, left=0.5in, right=0.5in]{geometry}
\usepackage{array}   % for \newcolumntype macro
\newcolumntype{L}{>{$}l<{$}}

\theoremstyle{plain}
\newtheorem*{theorem}{Theorem}
\newtheorem*{proposition}{Proposition}
\newtheorem{exercise}{Exercise}
\newtheorem*{lemma}{Lemma}
\newtheorem*{corollary}{Corollary}

\theoremstyle{definition}
\newtheorem*{definition}{Definition}
\newtheorem*{example}{Example}

\theoremstyle{remark}
\newtheorem*{remark}{Remark}

\renewcommand{\phi}{\varphi}
\renewcommand{\epsilon}{\varepsilon}
\renewcommand{\emptyset}{\varnothing}

\newcommand{\RR}{\mathbb{R}}
\newcommand{\ZZ}{\mathbb{Z}}
\newcommand{\NN}{\mathbb{N}}
\newcommand{\CC}{\mathbb{C}}
\newcommand{\QQ}{\mathbb{Q}}

\DeclareMathOperator{\ima}{\text{im}}

\begin{document}

\topmargin=-40pt
\rhead{Henry Woodburn}
\lhead{Math 655}
\renewcommand{\headrulewidth}{1pt}
\renewcommand{\headsep}{20pt}
\thispagestyle{fancy}

{\Huge \bfseries \noindent Homework 7}

\begin{enumerate}
    \item[1.] Suppose $T: X \rightarrow Y$ is a compact operator. To show that $T$ is bounded, we use 
    the fact that a compact set is bounded. Then clearly $T(B_X)$ is bounded as well, so $T$ is a bounded operator.

    Now suppose $T$ is not strictly singular. Then there is a subspace $A \subset X$ which is infinite dimensional,
    such that $T$ is an isomorphism of $A$ onto its image $T(A)$. Then $T(B_A) \subset T(B_X)$, so $\overline{T(B_A)}$ is 
    a compact subset of $Y$ and therefore of $T(A)$. But then
    \[
        T\big|_{A}^{-1}(\overline{T(B_A)})
    \]
    is compact, since bounded operators take compact sets to compat sets. Finally, 
    \[
        B_A = T\big|_A^{-1}(T(B_A)) \subset T\big|_{A}^{-1}(\overline{T(B_A)})
    \]
    implying $B_A$ is a compact subset of $A$, since it is a closed subset of a compact set. But this is 
    a contradiction, since the unit ball in any infinite dimensional space is not compact. Then
    $T$ must be strictly singular. 

    \item[2.] Let $T: X \rightarrow Y$ be a compact operator. We will show that $T$ is completely continuous,
    meaning it sends weak convergent sequences to strongly convergent sequences. Let $\{x_n\}$ be a weakly convergent
    sequence in $X$, and suppose $x_n \rightarrow_w x$ weakly. We know that $\{x_n\}$ is norm bounded by the uniform boundedness
    principle. Then $\{T(x_n)\}$ is contained in a relatively compact set. Consider some arbitrary subsequence $\{T(x_{n_k})\}$.
    Then there is a further subsequence $\{T(x_{n_{k_j}})\}$ which converges to some $y$. 

    But bounded operators take weakly convergent sequences to weakly convergent sequences: For any $f^* \in Y^*$, the 
    functional $f^*(T(\cdot))$ is in $X^*$. Then $T(x_n)$ converges weakly to $T(x)$. Then since strong convergence implies 
    weak convergence, the above subsequence must also converge to $T(x)$ by uniqueness of weak limits. Then all subsequences 
    have a subsequence which converges to $T(x)$, and thus the entire sequence $\{T(x_n)\}$ converges strongly to $T(x)$. 

    For the converse, take the identity map $\ell_1 \rightarrow \ell_1$. It is clearly bounded, and since $\ell_1$ has the schur 
    property, it is completely continuous. But it is not compact since the unit ball of an infinite dimensional space
    is not compact.

    \item[3.] Let $T: X \rightarrow Y$ be a bounded and completely continuous operator. We will show that $T(B_x)$ is relatively compact
    by showing every sequence in $T(B_X)$ has a convergent subsequence, converging in $Y$. This is equivalent to showing 
    every sequence in the closure has a subsequence which converges in the closure, which is equivalent to relative compactness
    in a metric space. 

    Let $\{x_n\}$ be a sequence in $T(B_X)$. Then choose $\{y_n\} \subset B_X$ such that $T(y_n) = x_n$. Since $x$ is reflexive, 
    the unit ball $B_X$ is weakly sequentially compact, so there is a subsequence $\{y_{n_k}\}$ which converges weakly.
    Then $\{x_{n_k}\} = \{T(x_{n_k})$ converges strongly since $T$ is completely continuous, and thus every sequence
    in $T(B_X)$ has a convergent subsequence, and $T(B_X)$ is relatively compact. 

    \item[4.] Let $1 \leq p < q < \infty$ and let $X$ be a closed subspace of $\ell_q$. Suppose $T: X \rightarrow \ell_p$ is 
    bounded. Since $X$ is a closed subspace of a reflexive space, $X$ is reflexive. 
    Then to show $T$ is compact, by problem 3 it is enough to show that $T$ is completely continuous. 

    Suppose $T$ is not completely continuous, and take a weakly null sequence $\{x_n\}$ in the unit sphere of $X$
    such that $T(x_n)$ does not converge to $0$ in norm. By passing to a subsequence, we can assume that $\{x_n\}$ is 
    equivalent to the canonical basis of $\ell_q$. By passing to a further subsequence, we can assume that $\|T(x_n)\| > a > 0$ 
    for all $n$, and we also know that $\{T(x_n)\}$ is weakly null since $T$ is bounded. By passing to a further subsequence 
    we can assume that $\{T(x_n)\}$ is equivalent to a basis of $\ell_p$. Then $T$ is a bounded map from $\ell_q$ to $\ell_p$
    which is impossible, hence $T$ must be completely continuous.

\end{enumerate}

\end{document}