\documentclass[11pt, reqno]{article}

\usepackage{amsmath, amsthm, amssymb}
\usepackage{enumitem}
\usepackage{tcolorbox}
\usepackage{hyperref}
\usepackage{tikz}
\usepackage{tikz-cd}
\usetikzlibrary{arrows.meta}
\usepackage{mathrsfs}
\usepackage{fancyhdr}
\usepackage[bottom=0.75in, top=1in, left=0.5in, right=0.5in]{geometry}
\usepackage{array}   % for \newcolumntype macro
\newcolumntype{L}{>{$}l<{$}}

\theoremstyle{plain}
\newtheorem*{theorem}{Theorem}
\newtheorem*{proposition}{Proposition}
\newtheorem{exercise}{Exercise}
\newtheorem*{lemma}{Lemma}
\newtheorem*{corollary}{Corollary}

\theoremstyle{definition}
\newtheorem*{definition}{Definition}
\newtheorem*{example}{Example}

\theoremstyle{remark}
\newtheorem*{remark}{Remark}

\renewcommand{\phi}{\varphi}
\renewcommand{\epsilon}{\varepsilon}
\renewcommand{\emptyset}{\varnothing}

\newcommand{\RR}{\mathbb{R}}
\newcommand{\ZZ}{\mathbb{Z}}
\newcommand{\NN}{\mathbb{N}}
\newcommand{\CC}{\mathbb{C}}
\newcommand{\QQ}{\mathbb{Q}}

\DeclareMathOperator{\ima}{\text{im}}

\begin{document}

\topmargin=-40pt
\rhead{Henry Woodburn}
\lhead{Math 655}
\renewcommand{\headrulewidth}{1pt}
\renewcommand{\headsep}{20pt}
\thispagestyle{fancy}

{\Huge \bfseries \noindent Homework 9}

Let $(x_n)$ be a sequence in a Banach space $X$.

\begin{enumerate}

\item[1.] (1.) We will show the following assertions are equivalent:
\begin{enumerate}
    \item[a.] The series $\sum_n x_n$ is weakly unconditionally cauchy
    \item[b.] There exists $C > 0$ such that $\sum_{n=1}^\infty |x^*(x_n)| \leq C\|x^*\|$ for all $x^* \in X^*$.
    \item[c.] There exists $C > 0$ such that $\sup_{k \in \NN}\|\sum_{n=1}^k \lambda_n x_n\| \leq C\|\lambda\|_{\infty}$ 
    for all $\lambda \in \ell_\infty$
    \item[d.] For all $\lambda \in c_0$, the series $\sum_n \lambda_n x_n$ converges
    \item[e.] There exists $C > 0$ such that $\sup_{(\epsilon_k) \subset \{-1,1\}}\|\sum_{n \in F} \epsilon_n x_n\| \leq C$
    for all finite subsets $F \subset \NN$.
\end{enumerate}

$(b.) \Rightarrow (a.)$ is obvious.

$(a.)\Rightarrow (b.)$: Construct a map $T: X^* \rightarrow \ell_1$ by 
$x^* \mapsto (x^*(x_n))$. We know $\|(x^*(x_n))\|_{\ell_1} < \infty$ from (a.). Also $T$ is clearly linear. 

We use the closed graph theorem to show $T$ is bounded. Suppose $x^*_n \rightarrow x^*$ and $Tx^*_n \rightarrow y$. 
Then $Tx^*_n$ must converge pointwise to $y$, so $y = x^*(x_n) = Tx^*$ and we are done.

Then if $C = \|T\|$, we have
\[
    \|(x^*(x_n))\|_{\ell_1} = \sum_1^\infty |x^*(x_n)| \leq C\|x^*\|
\]
for all $x^* \in X^*$. 

$(b.) \Rightarrow (c.)$: For any $\lambda \in \ell_\infty$ and any $x^* \in X^*$, we have

\[
    \left|x^*\left(\sum_1^m \lambda_n x_n\right)\right| = \left|\sum_1^m \lambda_n x^*(x_n)\right| \leq 
    \sum_1^m |\lambda_n x^*(x_n)| \leq \|\lambda\|_{\infty} \sum_1^m |x^*(x_n)| \leq C\|\lambda\|_\infty\|x^*\| = C\|\lambda\|_\infty
\]
Then since the $J$ map is an isometry, we have 
\[
    \|sum_1^m \lambda_n x_n\| \leq C \|\lambda\|_\infty
\]
for all $m$, and the result follows by taking the supremum over all $m$. 

$(c.) \Rightarrow (e.)$ This implication is easy, since for all $(\epsilon_n)$ and all finite 
subsets $F \subset \NN$, the modified sequence $(\epsilon'_n)$ which is equal to the original sequence on $F$
and zero elsewhere is an element of $\ell_\infty$ with norm 1, and thus we have 
\[
    \|\sum_1^k\epsilon'_n x_n\| = \|\sum_{n \in F} \epsilon_n x_n\| \leq C.
\]

$(d.) \Rightarrow (a.)$: Suppose (a.) does not hold. Then there exists $x^* \in X^*$ and a sequence $n_j$ with $n_1 = 1$ such that
$\sum_{n_j}^{n_{j+1}-1}|x^*(x_n)| \geq j$. For each $n$ choose $j$ such that $n_j \leq n < n_{j + 1}$ and let $\beta_n$
be a scalar such that $|\beta_n| = j^{-1}$ and $\beta x^*(x_n) = j^{-1}|x^*(x_n)|$. Then
\[
    \sum_1^\infty \beta_n x^*(x_n) = \infty,
\]

from which it follows that $\sum_1^N \beta_n x_n$ cannot converge, otherwise $x^*(\sum_1^N \beta_n x_n) = \sum_1^N \beta_n x^*(x_n)$
would converge to something finite. 

$(e.) \Rightarrow (a.)$: Again suppose (a.) does not hold, and choose a sequence $\epsilon_n$ in a similar fashion,
so that $\epsilon x^*(x_n) = |x^*(x_n)|$. Then similarly, we have
\[
    \sum_1^\infty \epsilon_n x^*(x_n) = \infty.
\]

Then it follows that condition (e.) cannot hold. Suppose it does. Then 
\[
    \sum_1^N \epsilon_n x^*(x_n) = \left|x^*\left(\sum_1^N \epsilon_n x_n\right)\right| \leq \|x^*\|\left\|\sum_1^N \epsilon_n x_n\right\| \leq C\|x^*\|
\] 
for all $N$, contradicting that the term on the left grows arbitrarily large in $N$.

This completes the proof.

(2.) Every unconditionally convergent series is weakly unconditionally cauchy. Since $\sum x_{\sigma(n)}$ is
finite for all permutations $\sigma$, we have that $\sum x^*(x_n)$ is unconditionally convergent,
which implies it is absolutely convergent since a sequence of real numbers is unconditionally convergent 
if and only if it is absolutely convergent. 

(3.) Every weakly unconditionally cauchy series is also weakly convergent, since absolute convergence implies
convergence, in the case of $\sum|x^*(x_n)|$.

\item[2.] (1.) We prove that 
\[
    \lim_{N \rightarrow \infty} \sup_{x^* \in B_{X^*}} \sum_N^\infty |x^*(x_n)| = 0.
\]

We need that for any $\epsilon > 0$ there is an $N$ 
such that $\sup_{(\epsilon_n) \in \{-1,1\}}\|\sum_N^\infty \epsilon_n x_n\| < \epsilon$. If this did not 
hold, we could choose blocks stretching out to infinity each corresponding to a different sequence $(\epsilon_n)$, 
and then could piece these sequences together so that $\sum \epsilon_n x_n$ diverges. 

Then fix $\epsilon > 0$ and choose such an $N$. 

For any $x^* \in B_{X^*}$, we can choose a sequence $(\epsilon_n) \subset \{-1,1\}$ such that 
$\sum |x^*(x_n)| = \sum \epsilon_n x^*(x_n)$. Then
\[
    \sum_N^\infty |x^*(x_n)| = \sum_N^\infty \epsilon_n x^*(x_n) \leq \|x^*\|\left\|\sum_N^\infty \epsilon_n x_n\right\|
    \leq \epsilon.
\]

It follows that $\sup_{x^* \in B_{X^*}}\sum_N^\infty |x^*(x_n)| < \epsilon$, and we are done. 

(2.) We prove that a series $\sum x_n$ is unconditionally convergent if for all $(\lambda_n) \in \ell_\infty$, the 
series $\sum \lambda_n x_n$ is convergent. 

If $\sum \lambda_n x_n$ is convergent for all $(\lambda_n) \in \ell_\infty$, we have subsequence convergence of $\sum x_n$ 
and thus $\sum x_n$ is unconditionally convergent.

Conversely, suppose $\sum x_n$ is unconditionally convergent and let $(\lambda_n) \in \ell_\infty$ be a sequence of norm 1 
without loss of generality. 

By the previous part, choose $N$ such that $\sum_n^\infty |x^*(x_n)| < \epsilon$ for all $x^* \in B_{X^*}$ for all $n > N$. 
Then fix any $n_1 < n_2$, with $n_1 < N$ and choose $x^* \in B_{X^*}$ such that 
\[
    \left|x^*(\sum_{n_1}^{n_2} \lambda_n x_n)\right| = \left\|\sum_{n_1}^{n_2} \lambda_n x_n\right\|.
\]
Then 
\[
    \left\|\sum_{n_1}^{n_2} \lambda_n x_n\right\| = \left|\sum_{n_1}^{n_2}\lambda_n x^*(x_n)\right|
    \leq \sum_{n_1}^{n_2}|\lambda_n||x^*(x_n)| \leq \sum_{n_1}^{n_2} |x^*(x_n)| \leq \sum_{n_1}^\infty |x^*(x_n)| \leq \epsilon,
\]
implying $\sum \lambda_n x_n$ is cauchy in norm and thus convergent.


\end{enumerate}

\end{document}