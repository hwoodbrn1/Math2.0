\documentclass[11pt, reqno]{article}

\usepackage{amsmath, amsthm, amssymb}
\usepackage{enumitem}
\usepackage{tcolorbox}
\usepackage{hyperref}
\usepackage{tikz}
\usepackage{tikz-cd}
\usetikzlibrary{arrows.meta}
\usepackage{mathrsfs}
\usepackage{fancyhdr}
\usepackage[bottom=0.75in, top=1in, left=0.5in, right=0.5in]{geometry}
\usepackage{array}   % for \newcolumntype macro
\newcolumntype{L}{>{$}l<{$}}

\theoremstyle{plain}
\newtheorem*{theorem}{Theorem}
\newtheorem*{proposition}{Proposition}
\newtheorem{exercise}{Exercise}
\newtheorem*{lemma}{Lemma}
\newtheorem*{corollary}{Corollary}

\theoremstyle{definition}
\newtheorem*{definition}{Definition}
\newtheorem*{example}{Example}

\theoremstyle{remark}
\newtheorem*{remark}{Remark}

\renewcommand{\phi}{\varphi}
\renewcommand{\epsilon}{\varepsilon}
\renewcommand{\emptyset}{\varnothing}

\newcommand{\RR}{\mathbb{R}}
\newcommand{\ZZ}{\mathbb{Z}}
\newcommand{\NN}{\mathbb{N}}
\newcommand{\CC}{\mathbb{C}}
\newcommand{\QQ}{\mathbb{Q}}

\DeclareMathOperator{\ima}{\text{im}}

\begin{document}

\topmargin=-40pt
\rhead{Henry Woodburn}
\lhead{Math 655}
\renewcommand{\headrulewidth}{1pt}
\renewcommand{\headsep}{20pt}
\thispagestyle{fancy}

{\Huge \bfseries \noindent Homework 6}

\textbf{Problem 1:} Let $(X, \|\cdot\|)$ be a Banach space.

\begin{enumerate}
    \item[1.] Let $E^* \subset X^*$ be a finite dimensional subspace, and let $S$ be a subset
    of $X^*$ such that $0$ is contained in the weak* closure of $S$ but not in the norm closure. We will 
    show that for any $\epsilon > 0$, there exists a point $x^* \in S$ such that we have 
    \[
        \|e^*\| \leq (1 + \epsilon)\|e^* + \lambda x^*\|
    \]
    for all $e^* \in E$ and all $\lambda \in \RR$. Note that it suffices to show this for $e^* \in S_{E^*}$ by 
    homogeneity, since we can multiply by any scalar $c \in \RR$ and take $c x^*$ instead.

    We are given some $\epsilon > 0$. First choose $\delta > 0$ such that $1 - \delta > \frac{1}{1 + \epsilon}$.
    Take $\{e_i^*\}_1^N$ a $\delta/4$ net in $S_{E^*}$ by compactness, and choose $\{e_i\}_1^N \subset X$ such that 
    $e_i^*(e_i) > 1 - \delta/2$ for $i = 1, \dots, N$. 

    Since $0$ is not in the norm closure of $S$, we know $\inf_{s \in S}\|S\| = \nu > 0$. But we have that $0$ is
    in the weak* closure of $S$, so any weak* neighborhood of $0$ intersects $S$. Then we can choose some $x^* \in S$
    such that $x^*(e_i) < \frac{\delta \nu}{8}$ for $i = 1, \dots, N$. 

    Then choose any $e^* \in S_{E^*}$ and any $\lambda \in \RR$. First, if $\lambda \geq \frac{2}{\|x^*\|}$, we have 
    \[
        \|e^* + \lambda x^*\| \geq \|\lambda x_n\| - \|e^*\| \geq 1 \geq 1 - \delta.
    \]

    Then suppose $\lambda < \frac{2}{\|x^*\|}$. Choose $i \in \{1, \dots, N\}$ so that $\|e^* - e_i^*\| < \frac{\delta}{4}$. 
    Then we have 
    \begin{align*}
        \|e^* + \lambda x^*\| &\geq \|e_i^* + \lambda x^*\| - \|e_i^* - e^*\|\\
        & \geq \|e_i^* + \lambda x^*\| - \frac{\delta}{4}\\
        & \geq |(e_i^* + \lambda x^*)(e_i)| - \frac{\delta}{4}\\
        & \geq |e_i^*(e_i)| - |\lambda||x^*(e_i)| - \frac{\delta}{4}\\
        & \geq 1 - \frac{\delta}{2} - \frac{\delta}{4} - \frac{\delta}{4} = 1 - \delta
    \end{align*}

    and we are done.

    \item[2.] Let $S$ be a subset as in part (1.). 
    
    Choose $\{\epsilon_i\} \subset (0, \infty)$ and choose some $x_1^* \in S$. Then $E_1 = \RR x_1^*$ is a finite
    dimensional subspace of $X$, so by (1.) we can choose some $x_2^* \in S$ such that 
    \[  
        \|a_1 x_1^*\| \leq (1 + \epsilon_1)\|a_1 x_1^* + a_2 x_2^*\|
    \]
    for any $a_1 x_1^* \in \RR x_1$ and $a_2 \in \RR$. Now $E_2 = \text{span}\{x_1^*, x_2^*\}$ is another finite dimensional
    subspace, so we can choose $x_3^* \in S$ such that
    \[
        \|a_1 x_1^* + a_2 x_2^*\| \leq (1 + \epsilon_2)\|a_1 x_1^* + a_2 x_2^* + a_3 x_3^*\|
    \]
    for any $a_1 x_1^* + a_2 x_2^* \in E_2$ and $a_3 \in \RR$.

    Inductively, there exists $x_{n + 1}^*$ such that for any $\sum_1^n a_i x_i^* \in E_n$ and any $a_{n + 1} \in \RR$,
    we have
    \[
        \|\sum_1^n a_i x_i^*\| \leq (1 + \epsilon_{n + 1})\|\sum_1^{n+1}a_i x_i^*\|.
    \]

    We can show $\{x_i^*\}_1^\infty$ satisfies the criterion for basic sequences and has basis coefficient
    smaller than $(1 + \epsilon)$. Let $m < n$ so that
    \[  
        \left\|\sum_1^m a_i x_i^*\right\| \leq (1 + \epsilon_{m+1})\left\|\sum_1^{m+1}a_i x_i^*\right\|
        \leq \left(\prod_{i = m+1}^n (1 + \epsilon_i)\right)\left\|\sum_1^n a_i x_i^*\right\|.
    \]
    Then we have the desired result as long as we choose $\epsilon_i$ such that
    \[
        \prod_{i = 1}^\infty (1 + \epsilon_i) < (1 + \epsilon).
    \]

    \item[3-4.] First we will address (4.) by showing that if $\{x_n\}_1^\infty$ is a weakly null sequence in 
    an infinite dimensional Banach space such that $\inf \|x_n\| > 0$, for every $\epsilon > 0$ there
    is a subsequence of $\{x_n\}_1^\infty$ which is a basic sequence with basis constant at most 
    $(1 + \epsilon)$. Then we will show that $0$ belongs to the weak closure of the unit sphere of any 
    infinite dimensional Banach space, which will allow us to solve (3.) in almost the same way as (4.). 
    We first prove a lemma.

    \textbf{Lemma 1:}
    Suppose $\{x_n\}_1^\infty$ is a weakly null sequence in an infinite dimensional Banach space, and 
    let $E$ be a finite dimensional subspace. Then for any $\epsilon > 0$ and $N > 0$, 
    there is some $n_0 > N$ such that 
    \[
        \|e\| \leq (1 + \epsilon)\|e + \lambda x_{n_0}\|
    \]
    for all $e \in E$ and $\lambda \in \RR$. Again we can take $e \in S_E$ without loss of generality.

    First choose $\delta > 0$ such that $1 - \delta > \frac{1}{1 + \epsilon}$. Choose a $\frac{\delta}{2}$-net
    $\{e_i\}_1^N$ in $S_E$ since it is compact, and choose $\{e_i^*\}_1^N \subset X^*$ such that
    $e_i^*(e_i) = 1$ for all $i$ using Hahn-Banach. 

    By similar reasoning as in (1.), we can choose $n_0 > N$ such that $e_i^*(x_{n_0}) < \frac{\delta \inf \|x_n\|}{4}$
    for all $i$ since $\{x_n\}$ is weakly convergent but bounded by a positive number in norm. 

    Now let $e \in S_E$ and choose $j$ such that $\|e - e_j\| < \frac{\delta}{2}$ and any $\lambda \in \RR$. 
    Similar to (1.), if $\lambda \geq \frac{2}{\|x_{n_0}}$, we immediately get
    \[  
        \|e + \lambda x_{n_0}\| \geq 1 - \delta,
    \]
    and in the other case this can be shown using almost exactly the same string of inequalities as in (1.),
    except with $|e_i^*(e_i)| = 1$ instead.

    Then we can carry out basically the same process as in (2.) to show that there is a subsequence of $\{x_n\}_1^\infty$
    which is a basic sequence. Start with $x_{n_1} = x_1$, and get an $x_{n_2}$ with $n_1 < n_2$ using $N = n_1$
    in lemma 1 with $\RR x_{n_1}$ as the finite dimensional subspace. Continue for $i = 1, \dots, \infty$
    to get a subsequence $\{x_{n_i}\}_1^\infty$ which is a basic sequence with basis constant at most $(1 + \epsilon)$,
    by choosing small $\epsilon_i$ at each step in the process as in (2.). 

    For (3.), let $\{\phi_1, \dots, \phi_n\}$ be linear functionals. Then $\cap_1^n \ker \phi_i$ is a set 
    with finite codimension, so it must be infinite dimensional since $X$ is infinite dimensional. Then 
    we can choose some $x \in S_X$ such that $\phi_i(x) = 0$ for $i = 1,\dots, n$. 

    We note that in the lemma, the sequence did not need to be weakly null; the same result holds for
    any set which is bounded below in norm and contains a point $s$ for all $\epsilon > 0$ such that 
    $e_i^*(s) < \epsilon$ for all $i = 1, \dots, n$. Then clearly another version of lemma 1 holds 
    with the unit sphere $S_X$ instead of the weakly null sequence. Here we dont need to worry about 
    $N$. 

    Now we can finish to show there is a basic sequence in $S_X$ by the same process as in (2.).

\end{enumerate}

\end{document}