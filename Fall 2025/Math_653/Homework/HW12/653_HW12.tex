\documentclass[11pt, reqno]{article}

\usepackage{amsmath, amsthm, amssymb}
\usepackage{enumitem}
\usepackage{tcolorbox}
\usepackage{hyperref}
\usepackage{tikz}
\usepackage{tikz-cd}
\usepackage{pgfplots}
\pgfplotsset{compat=1.18}
\usetikzlibrary{arrows.meta}
\usepackage{mathrsfs}
\usepackage{fancyhdr}
\usepackage[bottom=0.75in, top=1in, left=0.5in, right=0.5in]{geometry}
\usepackage{array}   % for \newcolumntype macro
\newcolumntype{L}{>{$}l<{$}}

\theoremstyle{plain}
\newtheorem*{theorem}{Theorem}
\newtheorem*{proposition}{Proposition}
\newtheorem{exercise}{Exercise}
\newtheorem*{lemma}{Lemma}
\newtheorem*{corollary}{Corollary}

\theoremstyle{definition}
\newtheorem*{definition}{Definition}
\newtheorem*{example}{Example}

\theoremstyle{remark}
\newtheorem*{remark}{Remark}

\renewcommand{\phi}{\varphi}
\renewcommand{\epsilon}{\varepsilon}
\renewcommand{\emptyset}{\varnothing}

\newcommand{\RR}{\mathbb{R}}
\newcommand{\ZZ}{\mathbb{Z}}
\newcommand{\NN}{\mathbb{N}}
\newcommand{\CC}{\mathbb{C}}
\newcommand{\QQ}{\mathbb{Q}}

\DeclareMathOperator{\ima}{\text{im}}

\begin{document}

\topmargin=-40pt
\rhead{Henry Woodburn}
\lhead{Math 653}
\renewcommand{\headrulewidth}{1pt}
\renewcommand{\headsep}{20pt}
\thispagestyle{fancy}

{\Huge \bfseries \noindent Homework 12}

\begin{enumerate}
    \item[73.] We prove that the ideal $I = (2,x)$ within $\ZZ[x]$ is not principal. We must show 
    there is no $a \in \ZZ[x]$ such that for all $i \in I$, there is some $r \in \ZZ[x]$ such 
    that $i = ra$. Suppose it is principal, $I = (a)$.

    Since $2 \in I$, $a$ must be a constant polynomial, otherwise we could never lower its degree to 
    get $2$ from multiplying by elements in $\ZZ[x]$. Then $a$ must be $2$, since it is not $1$ as $I$ is
    proper, and it cannot be greater than $2$ or we could not generate $2$. But then we have no way to 
    generate $x$, giving a contradiction. Then $I$ cannot be principal. 

    \item[74.] Let $p, r \in \NN$ with $p$ prime and $r > 0$. For any $a$ element of $\ZZ/p^r \ZZ$ which is
    relatively prime with $p$, by reversing the euclidean algorithm, a result commonly referred to as ''bezout's 
    identity,'' as has been verified and discussed in class, there exist integers $k$ and $\ell$ such that
    $ka + \ell p = 1$. In other words, there is an integer $k$ such that $ka$ equals $1$ mod $p$. Then $a$ is 
    a unit. Otherwise, if $a$ has a factor of $p$, we will never get $1$ by multiplying it by any other element as $1$
    has no factor of $p$. Then the elements of the group of units are precisely those which have no factors of $p$.

    I was not able to show the cyclicity. 

    \item[75.] Letting $i = \sqrt{-1}$, we first verify that the function $N: \ZZ[i] \rightarrow \NN$ is multiplicative:
    For elements $(a + bi)$ and $(c + di)$, we have
    \begin{align*}
        N((a + bi)(c + di)) = N((ac - bd) + (ad + bc)i)\\
        = a^2 c^2 - 2acbd + b^2 d^2 + a^2 d^2 + 2adbc + b^2 c^2\\
        = (a^2 + b^2)(c^2 + d^2) = N(a + bi)N(c + di).
    \end{align*}

    To find the units, suppose $a,b \in \ZZ[i]$ with $ab = 1$. Then $N(a)N(b) = N(1) = 1$, and thus
    we must have $N(a) = N(b) = \pm  1$. Moreover, suppose $N(c) = 1$ for some $c \in \ZZ[i]$. 
    Then it is clear that $c$ must equal $\pm i$ or $\pm 1$, each of these being units. Then 
    the units of $\ZZ[i]$ are only those with norm $1$. 

    Now suppose that $N(\alpha) = p$ for $p$ a prime. Then suppose $\beta, \gamma$ exist such that
    $\beta \gamma = \alpha$. Then $N(\beta)N(\gamma) = N(\alpha) = p$, so that one of $N(\beta)$ and 
    $N(\gamma)$ must be a unit. Thus $\alpha$ is irreducible. 

    Finally suppose $N(\alpha) = p^2$ for a prime $p$ such that $p = 3 \mod 4$. Suppose there exist
    non-units $\beta$ and $\gamma$ such that $\beta \gamma = \alpha$. Then we must have
    $N(\beta) = N(\gamma) = p$. This means that there are integers $a$ and $b$ such that 
    $a^2 + b^2 = p = 3 \mod 4$. Since $p$ is prime, we must have $a, b \neq 0$, otherwise $p$ would 
    be a square. We must also have that one of $a$ and $b$ must be odd and the other even-say $a$ is
    odd without loss of generality. 

    Then $b = 2r$ for an integer $r$ and $a = 2s + 1$. Then we have 
    \[
        a^2 + b^2 = 4r^2 + 1 + 4s + 4s^2,
    \]
    contradicting that $p = 3 \mod 4$, and thus $\alpha$ must be irreducible. 

    \item[76.] We will show $\ZZ[i]$ is a unique factorization domain by showing it is a euclidean domain, since
    every euclidean domain is a unique factorization domain. 

    Take $a, b \in \ZZ[i]$. We want to find elements $q, r \in \ZZ[i]$ such that $a = qb + r$, where
    either $N(r) < N(b)$ or $r = 0$. 

    First note that multiplication by an element of $\ZZ[i]$ consists of the sum of multiplication by a 
    scalar part and an imaginary part. The product with the imaginary component will be a vector 
    perpendicular to the product with the real component. Then it is clear that 
    multiples of $b$ lie on the square lattice generated by $b$. 
    
    If the point $a$ lies on one of these lattice points, and we can take $q$ to be the gaussian 
    integer which takes $b$ to this point and $r$ to be zero. Otherwise, $a$ lies within on of the lattice boxes.
    Each point within the lattice can be reached by adding a gaussian integer to a lattice point. We just need
    to make sure it can be reached by an element of norm less than $b$. But this is clear, since in the worst
    case, $a$ lies in the middle of one of these boxes, and the distance from $a$ to any lattice point 
    is still less than the length of $b$, and otherwise $a$ lies within
    a ball of radius of less than $\sqrt{2}/2$ times the length of $b$ about some lattice point. Take
    $r$ to be this difference between $a$ and its closest lattice point to have the desired 
    expression of $a$.
    
    \item[77.] We prove that every prime $p$ which is congruent to $1$ modulo $4$ is the sum of two squares.
    
    We have that $4$ divides $p-1$ by hypothesis. The group of units of $\ZZ_p^\times$ is cyclic, and
    thus there is an element $x \in \ZZ_p^\times$ such that the powers of $x$ generate the entire group.
    Since $x^{p-1/2}$ is order $2$, it must equal negative one. This is because if $x^2 -1 = 0$, then
    either $x = -1$ or $1$. But it is not 1. 

    Then since $p = 1 \mod 4$, the element $x^{p-1/4} = a$ is a square root of $-1$, and $p$ 
    divides $(a^2 + 1) = (a + i)(a - i)$. Then $p$ divides one of these terms.

    If $(r + si)p = a + i$, then $rp + spi = a + i$. But this is impossible. Likewise for the other
    term. Thus $p$ is not prime in $\ZZ[i]$, and thus is reducible since $\ZZ[i]$ is a UFD. 

    \item[78.] Let $R$ be a commutative ring and $S \subset R$ a multiplicatively closed subset. 
    We identify the kernel of the map $\iota: R \rightarrow R[S^{-1}]$. 

    First suppose that $r \in \ker(\iota)$, so that $\iota(r) = 0s/s$ for $s \in S$. Then 
    $rs/s = 0s/s$, and thus $t(rs^2) = 0$ for $t \in S$. Then since $R$ is commutative, $r(ts^2) = 0$,
    and $r$ is a zero divisor with $q = ts^2 \in S$ such that $rq = 0$.

    Conversely suppose that $r \in R$ such that there exists $s \in S$ with $rs = 0$. Then
    $rs/s = 0s/s$, since $rs^2 = 0$. 

    Then the kernel of $\iota$ are those elements $r \in R$ for which there is an element $s \in S$ such
    that $rs = 0$. 

    \item[79.] Let $S$ be a multiplicatively closed subset of an integral domain $R$ with $0 \notin S$. 
    Note that since $0 \notin S$, the ring $R[S^{-1}]$ is nontrivial. Let $P'$ be an ideal in $R[S^{-1}]$.
    Then $\iota^{-1}(P') = P$ is an ideal in $R$, and thus there is some $a \in R$ such that $P = (a)$.
    Moreover, $P' = (a)[S^{-1}] = \{ra/s: r \in R, s \in S\} = \{a/q \cdot rq/s: r \in R, s \in S\} = (a)$
    for any $q \in S$. Then $P'$ is principal. 

    \item[80.] Let $S \subset R$ be a submonoid that does not contain $0$. Let $P$ be a maximal element
    in the set of ideals which do not meet $S$. Suppose $ab \in P$ but $a,b \notin P$. 
    Then $P + (a)$ and $P + (b)$ must meet $S$, since they are ideals containing $P$. Then there 
    exist $r, s \in R$ and $p, q \in P$ such that $p + ra$ and $q + sb$ are in $S$. But then 
    their product, $pq + qra + psb + rsab$, is both an element of $S$, since $S$ is closed under multiplication,
    and an element of $P$, since $P$ is an ideal. Then this is a contradiction, and either $a$ or $b$
    is in $P$. Then $P$ is prime. 

    \item[81.] Let $p \in \ZZ$ be a prime number. The canonical map $\phi: \ZZ \to \ZZ_p$ sends
    every element of $\ZZ\setminus (p)$ to a unit. Then by the universal property of rings of fractions,
    there is a map $\psi: \ZZ_{(p)} \to \ZZ_p$ such that $\psi \circ \iota = \phi$.

    \item[82.] Suppose $R$ is a commutative ring. We will show $R$ is local if and only if for every
    $r,s \in R$, if $r + s = 1$, then either $r$ or $s$ is a unit. 

    First suppose $R$ is local with unique maximal ideal $M$. Take $r, s \in R$ such that 
    $r + s = 1$, and suppose neither is a unit. Then $(r)$ and $(s)$ are proper ideals 
    and are thus contained in $M$. But then $r + s = 1 \in M$, and thus $M = R$ giving a 
    contradiction. So one of $r$ and $s$ must be a unit.
    
    Conversely suppose the latter condition holds. Let $A$ be one maximal ideal and suppose $B$
    is another ideal not contained in $A$. Then $B + A$ contains $A$ and must be equal to $R$. But
    then there exist $a \in A$ and $b \in B$ such that $a + b = 1$, thus either $a$ or $b$
    is a unit. But $A$ is proper, so $b$ must be a unit and $B = R$. Then the only ideals not contained 
    in $A$ are $R$, and $A$ is the unique maximal ideal. 
\end{enumerate}

\end{document}