\documentclass[11pt, reqno]{article}

\usepackage{amsmath, amsthm, amssymb}
\usepackage{enumitem}
\usepackage{tcolorbox}
\usepackage{hyperref}
\usepackage{tikz}
\usepackage{tikz-cd}
\usetikzlibrary{arrows.meta}
\usepackage{mathrsfs}
\usepackage{fancyhdr}
\usepackage[bottom=0.75in, top=1in, left=0.5in, right=0.5in]{geometry}
\usepackage{array}   % for \newcolumntype macro
\newcolumntype{L}{>{$}l<{$}}

\theoremstyle{plain}
\newtheorem*{theorem}{Theorem}
\newtheorem*{proposition}{Proposition}
\newtheorem{exercise}{Exercise}
\newtheorem*{lemma}{Lemma}
\newtheorem*{corollary}{Corollary}

\theoremstyle{definition}
\newtheorem*{definition}{Definition}
\newtheorem*{example}{Example}

\theoremstyle{remark}
\newtheorem*{remark}{Remark}

\renewcommand{\phi}{\varphi}
\renewcommand{\epsilon}{\varepsilon}
\renewcommand{\emptyset}{\varnothing}

\newcommand{\RR}{\mathbb{R}}
\newcommand{\ZZ}{\mathbb{Z}}
\newcommand{\NN}{\mathbb{N}}
\newcommand{\CC}{\mathbb{C}}
\newcommand{\QQ}{\mathbb{Q}}

\DeclareMathOperator{\ima}{\text{im}}

\begin{document}

\topmargin=-40pt
\rhead{Henry Woodburn}
\lhead{Math 653}
\renewcommand{\headrulewidth}{1pt}
\renewcommand{\headsep}{20pt}
\thispagestyle{fancy}

{\Huge \bfseries \noindent Homework 4}

\begin{enumerate}
    \item[21.] Let $G$ be a group with subgroups $H$ and $K$. \begin{enumerate}
        \item[a.] $H \cap K$ is a subgroup. It clearly contains identity. Since both $H$ and 
        $K$ are subgroups, $H \cap K$ clearly contains inverses and is closed under the group operation.

        \item[b.] For any $g \in G$, we have $g(H \cap K)g^{-1} \subset H \cap K$ since $H\cap K$ is a subset
        of both $H$ and $K$, and thus its conjugates must lie within $H$ and $K$ since both are normal. Then we 
        are done, since $|g(H \cap K)g^{-1}| = |H \cap K|$ implies $g(H \cap K)g^{-1} = H \cap K$.

        \item[c.] If $H$ and $K$ are subgroups of $G$, $H \cup K$ is not always a subgroup. For example, 
        take $G = \ZZ \times \ZZ$, and let $H = \langle (1,0)\rangle$ and $K = \langle (0,1)\rangle$. Then
        their union does not contain the element $(1,1)$, so it cannot be a group. Since $H$ and $K$
        are both normal, this shows (b.) fails as well. 
    \end{enumerate}

    \item[22.] Let $G$ be a group and $[G,G]$ the subgroup generated by the commutators. To show it is 
    normal, consider one element $aba^{-1}b^{-1}$ and some $g \in G$. Then the conjugate
    \[
        gaba^{-1}b^{-1}g^{-1} = gag^{1}gbg^{-1}ga^{-1}g^{-1}gb^{-1}g^{-1} = (gag^{-1})(gbg^{-1})(gag^{-1})^{-1}(gbg^{-1})^{-1}
    \]
    is also a commutator. This also applies to arbitrary combinations of elements of $[G, G]$ and
    thus $g[G,G]g^{-1} \subset [G,G]$, which again shows that $g[G,G]g^{-1} = [G,G]$. Then $[G,G]$ is normal
    in $G$. 
    \bigbreak
    Now we show $G/[G,G]$ is abelian. Let $g' = g[G,G]$ and $h' = h[G,G]$ 
    be elements of $G/[G,G]$. Then by taking their commutator, we see that 
    \[
        g'h'g'^{-1}h'^{-1} = (ghg^{-1}h^{-1})[G,G] = [G,G]
    \]
    and thus $g'h' = h'g'$.
    \bigbreak
    Let $\phi: G \rightarrow H$ be a homomorphism of groups with $H$ abelian. To show that $[G,G] \subset \ker \phi$,
    take some $aba^{-1}b^{-1} \in [G,G]$. Then 
    \[
        \phi(aba^{-1}b^{-1}) = \phi(a)\phi(b)\phi(a^{-1})\phi(b^{-1}) = \phi(a)\phi(a)^{-1}\phi(b)\phi(b)^{-1} = e_H,
    \]
    since $H$ is abelian. Then the same applies to arbitrary elements in $[G,G]$. 

    Then by one of the isomorphism theorems, there is a unique homomorphism $\psi: G/[G,G] \rightarrow H$ such
    that we have a commutative diagram
    \begin{equation*}
        \begin{tikzcd}
            G \arrow[d, "\pi"] \arrow[r,"\phi"] & H \\
            G/[G,G] \arrow[ur, "\psi"]
        \end{tikzcd}
    \end{equation*}

    \item[23.] Assume the hypothesis of the problem. Then $N_1 = \ker p_2 \cap H$, and $N_2 = \ker p_1 \cap H$.
    Thus $N_1$ contains all of the elements of the form $(a,0) \in H$, and we may identify $N_1$ 
    with its projection $N_1'$ $G_1$. Then let $g \in G_1$ and $a \in N_1'$. Using the surjectivity of 
    $H$ under $P_1$, we have
    \[
        gag^{-1} = p_1(g,h)p_1(a,0)p_1(g^{-1},h^{-1}) = p_1((g,h)(a,0)(g^{-1},h^{-1}))
    \]
    for some $(g,h)$ in $H$ such that $p_1(g,h) = g$. Then because $N_1$ is normal in $G$, as it 
    is the kernel of $p_2$ restricted to $H$, we have that this equals $p_1((b,0)) = b$ for 
    some $b \in N_1'$. Then by the same argument as above we know that $N_1'$ is normal in $G_1$. 
    The same argument applies to $G_2$.
    \bigbreak
    Now we will construct an isomorphism $\phi: G_1/N_1 \rightarrow G_2/N_2$. Since $p_1(H) = G_1$, 
    for any $g \in G_1$ there is some $h \in G_2$ such that $(g,h) \in H$. Then let $\phi(gN_1) = hN_2$.
    
    We need to check that $\phi$ is well defined. Suppose we have elements $h_1, h_2 \in G_2$ such that
    both $(g,h_1)$ and $(g, h_2)$ are in $H$. Then $(0, h_1 h_2^{-1}) \in H$, and we see that $h_1 h_2^{-1} \in N_2$,
    identified as a subgroup of $G_2$. Thus $h_1 N_2 = h_2 N_2$ and the image of $gN_1$ is well defined.
    
    Moreover, suppose $g_1 N_1 = g_2 N_1$ for $g_1, g_2 \in G_1$ and take $h_1, h_2 \in G_2$ such that
    $(g_1, h_1)$ and $(g_2, h_2)$ are elements of $H$. Then we have $g_1 g_2^{-1} \in N_1$ and thus 
    \[
        (g_1 g_2^{-1}, h_1 h_2^{-1}) \in p_2^{-1}(g_1 g_2^{-1}) = (g_1 g_2^{-1}, 0) + \ker p_2.
    \]I
    It follows that $h_1 h_2^{-1} \in \ker p_2$ and thus $h_1 N_2 = h_2 N_2$.
    \bigbreak
    Now we show that $\phi$ is a homomorphism. Suppose $a,b \in G_1$. Then there exist $h,k \in G_2$
    such that $(a,h),(b,k) \in H$ and thus $(ab, hk) \in H$ as $H$ is a subgroup. Then 
    \[
        \phi(ab N_1) = hk N_2 = (h N_2)(k N_2) = \phi(a N_1)\phi(b N_1).
    \]
    Also $\phi(e N_1) = e N_2$ because $H$ is a subgroup. 
    \bigbreak 
    Finally, we show $\phi$ is an isomorphism. It is clearly surjective since for every $h \in G_2$ there 
    is some $g \in G_2$ such that $(g,h) \in H$. To show injectivity, suppose $\phi(g N_1) = e N_2$. Then 
    $(g, e) \in H$ and $g \in N_1$. 
    \bigbreak
    Then we have also shown that the image of $G$ in $G_1/N_1 \times G_2/N_2$ is the graph of this 
    isomorphism, since an element $(g,h) \in H$ corresponds to a pair of cosets $(gN_1, hN_2)$, 
    such that $gN_1$ is mapped to $hN_2$ under $\phi$. 

    \item[24.] Throughout we will denote $U(n, \mathbb{K})$ by $U$ and $B(2,\mathbb{K})$ by $B$.
    We first show the case $n = 2$. We proved in class that there is an exact sequence 
    \begin{equation*}
        \begin{tikzcd}
            1 \arrow[r] & U \arrow[r, "f"] & B \arrow[r, "g"] & \mathbb{T} \arrow[r] & 1,
        \end{tikzcd}
    \end{equation*}
    where $\mathbb{T}$ is the diagonal subgroup. 

    First, since $U$ is a subgroup in $B$ which is equal to the kernel of the homomorphism $g$, $U$ is normal in $B$.
    Moreover, it is clear that $U$ is abelian in the case $n = 2$. 

    This exact sequence implies that $B/U$ is isomorphic to $\mathbb{T}$, an abelian group. 
    
    Then we have the tower of subgroups 
    \[
        B \supset U \supset \{e\}
    \]

    where $U$ is normal in $B$, $\{e\}$ is normal in $U$, and the quotient groups $B/U$ and $U/\{e\} = U$
    are both abelian. Then we have shown $B(2, \mathbb{K})$ is solvable.
    \bigbreak
    For the case where $n = 3$, we must consider the commutator subgroup in $U$. Collapsing many steps of 
    the calculation, we see that for matricies 
    \[
        A = \begin{bmatrix}
            1 & a & b \\
            0 & 1 & c\\
            0 & 0 & 1
        \end{bmatrix}, 
        B = \begin{bmatrix}
            1 & d & e \\
            0 & 1 & f \\
            0 & 0 & 1
        \end{bmatrix},
    \]
    their commutator is 
    \[
        ABA^{-1}B^{-1} = \begin{bmatrix}
            1 & 0 & af - dc \\
            0 & 1 & 0 \\
            0 & 0 & 1 
        \end{bmatrix}.
    \]

    Then it is easy to see that $[U, U]$ is the subgroup of $U$ consisting of matrices with $1$'s on the 
    diagonal and all other entries zero except for the top right corner, which is an arbitrary element of 
    $\mathbb{K}$. Note that $[U, U]$ is abelian by a simple calculation. 

    By a previous excercise, $[U, U]$ is a normal subgroup of $U$, and $U/[U, U]$ is abelian. Then we have a 
    normal tower of subgroups 
    \[
        B \supset U \supset [U, U] \supset \{e\}.
    \]
    We have aleady shown that each quotient group is abelian as well, so it is an abelian tower and 
    $B(3, \mathbb{K})$ is solvable. 
    
    \item[25.] Recall that the quaternion group $\mathcal{Q}_8$ is generated by elements $i, j$, such that 
    if $k = ij$ and $m = i^2$, we have $i^2 = j^2 = k^2 = m$, $m^2 = e$, and $ij = mji$.

    Consider the subgroup $\langle i \rangle$ generated by $i$. Since $i^4 = e$, $\langle i \rangle $ is 
    order $4$ and cyclic, and has index $2$, meaning it is normal in $\mathcal{Q}_8$. It has a cyclic
    subgroup generated by $m$ of order $2$ which is thus normal in $\langle i \rangle$. Then we have 
    the normal tower 
    \[
        \mathcal{Q}_8 \supset \langle i \rangle \supset \langle m \rangle \supset \{e\}.
    \]
    Additionally, each of the quotient groups is order $2$, so that they have no nontrivial subgroups 
    and are simple. Then this is a composition series for $\mathcal{Q}_8$.

    Another equivalent composition series is
    \[
        \mathcal{Q}_8 \supset \langle j \rangle \supset \langle m \rangle \supset \{e\},
    \]
    since all possible quotients are the group of $2$ elements. 

    \item[26.] Recall that the dihedral group of order $8$, $D_4$, is the group generated by elements 
    $a, b$, such that $a^4 = b^2 = e$, and $bab^{-1} = a^{-1}$. 

    We have a subgroup of order $4$ generated by $a$ which is again normal in $D_4$. It has 
    a normal subgroup generated by $a^2$. Then we have a composition series 
    \[
        D_4 \supset \langle a \rangle \supset \langle a^2 \rangle \supset \{e\},
    \]
    since each of the quotients have order $2$. 

    Another subgroup of order $4$ is generated by elements $b$ and $aba^{-1}$, which is seen to 
    be isomorphic to $\ZZ_2 \times \ZZ_2$ by considering both elements as reflections about perpendicular 
    lines of symmetry of the square. Then we have another equivalent composition series 
    \[
        D_4 \supset \langle b, aba^{-1}\rangle \supset \langle b \rangle \supset \{e\}
    \]
    as all quotients are isomorphic to the group of order $2$. 

    \item[27.] Let $G$ be a group with normal subgroup $H$. First suppose $H$ and $G/H$ are solvable. Then
    we have abelian towers 
    \begin{align*}
        H & = H_0 \supset H_1 \supset \cdots \supset H_m = \{e\}\\
        G/H & = K_0 \supset K_1 \supset \cdots \supset K_n = \{e\}.
    \end{align*}
    Let $\pi: G \rightarrow G/H$ be the canonical projection. Define $G_i = \pi^{-1}(K_i)$, 
    so that there is a normal tower 
    \[
        G_0 = G \supset G_1 \supset \cdots G_n = H,
    \]
    which is an abelian tower by one of the isomorphism theorems which states a quotient of 
    2 quotient groups with the same denominator is isomorphic to the quotient of the first numerator 
    by the second. 

    Then we can join this with the first abelian tower to get an abelian tower 
    \[
        G_0 = G \supset G_1 \supset \cdot \supset G_n = H = H_0 \supset H_1 \supset \cdots \supset H_m = \{e\},
    \]
    proving that $G$ is solvable 
    \bigbreak
    Conversely suppose that $G$ is solvable, and we have an abelian tower 
    \[
        G = G_0 \supset G_1 \supset \cdots \supset G_n  = \{e\}.
    \]
    By intersecting with $H$ and letting $H_i = G_i \cap H$, we obtain a new normal tower 
    \[
        H = H_0 \supset H_1 \supset \cdots \supset H_0 = \{e\},
    \]
    after removing possible duplicates. In fact, it is an abelian tower since we have an embedding 
    $H_i/H_{i+1} \rightarrow G_i/G_{i+1}$. Then $H$ is solvable. 

    By taking quotients by $H$ instead and letting $K_i = G_i/H$, we obtain a new normal tower 
    \[
        G/H = K_0 \supset K_1 \supset \cdots \supset K_0 = \{e\},
    \]
    after removing duplicates. Then this is an abelian tower since $K_i / K_{i + 1}$ is isomorphic
    to $G_i / G_{i + 1}$ by the same isomorphism theorem. Then $G/H$ is solvable. 

\end{enumerate}

\end{document}