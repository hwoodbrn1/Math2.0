\documentclass[11pt, reqno]{article}

\usepackage{amsmath, amsthm, amssymb}
\usepackage{enumitem}
\usepackage{tcolorbox}
\usepackage{hyperref}
\usepackage{tikz}
\usepackage{tikz-cd}
\usepackage{pgfplots}
\pgfplotsset{compat=1.18}
\usetikzlibrary{arrows.meta}
\usepackage{mathrsfs}
\usepackage{fancyhdr}
\usepackage[bottom=0.75in, top=1in, left=0.5in, right=0.5in]{geometry}
\usepackage{array}   % for \newcolumntype macro
\newcolumntype{L}{>{$}l<{$}}

\theoremstyle{plain}
\newtheorem*{theorem}{Theorem}
\newtheorem*{proposition}{Proposition}
\newtheorem{exercise}{Exercise}
\newtheorem*{lemma}{Lemma}
\newtheorem*{corollary}{Corollary}

\theoremstyle{definition}
\newtheorem*{definition}{Definition}
\newtheorem*{example}{Example}

\theoremstyle{remark}
\newtheorem*{remark}{Remark}

\renewcommand{\phi}{\varphi}
\renewcommand{\epsilon}{\varepsilon}
\renewcommand{\emptyset}{\varnothing}

\newcommand{\RR}{\mathbb{R}}
\newcommand{\ZZ}{\mathbb{Z}}
\newcommand{\NN}{\mathbb{N}}
\newcommand{\CC}{\mathbb{C}}
\newcommand{\QQ}{\mathbb{Q}}

\DeclareMathOperator{\ima}{\text{im}}

\begin{document}

\topmargin=-40pt
\rhead{Henry Woodburn}
\lhead{Math 653}
\renewcommand{\headrulewidth}{1pt}
\renewcommand{\headsep}{20pt}
\thispagestyle{fancy}

{\Huge \bfseries \noindent Extra Problem}
\bigbreak\noindent
Let $D$ be a division ring.  Let $S$ be the ring of $n\times n$ matrices, $n > 1$. 

\begin{enumerate}
    \item[1.] $S$ has no proper ideals besides the zero ideal.
    
    We proved on homework 10 that the only ideals in such a matrix ring are of the form $M_n(I)$ for $I \subset D$ 
    an ideal. Since $D$ is a division ring, the only ideals are $D$ and the zero ideal. Then the only ideals 
    of $S$ are $S$ and the zero ideal. 

    \item[2.] $S$ has zero divisors: as long as $n > 1$, we have
    \[
        \begin{bmatrix}
            1 & 0 & \dots & 0\\
            0 & \ddots & \dots & 0\\
            \vdots  & & & \vdots\\
            0 & & & 0
        \end{bmatrix}
        \begin{bmatrix}
            0 & 1 & 0 & \dots & 0\\
            0 & \ddots & & \dots & 0\\
            \vdots  & & & & \vdots\\
            0 & & & & 0
        \end{bmatrix}
        = 0.
    \]

    \item[3.] Since $S$ is non-commutative, we must use the more general version of prime ideals.
    Suppose $AB \subset \langle 0 \rangle$. Since $D$ has no proper ideals, the only possibilities for 
    $A$ and $B$ are $0$ and $S$ itself. If both $A$ and $B$ are $S$, clearly their product is not contained
    in the trivial ideal. Then one or both of $A$ and $B$ must be equal to $0$ and thus the ideal $0$
    is prime. 
\end{enumerate}

\end{document}