\documentclass[11pt, reqno]{article}

\usepackage{amsmath, amsthm, amssymb}
\usepackage{enumitem}
\usepackage{tcolorbox}
\usepackage{hyperref}
\usepackage{tikz}
\usepackage{tikz-cd}
\usetikzlibrary{arrows.meta}
\usepackage{mathrsfs}
\usepackage{fancyhdr}
\usepackage[bottom=0.75in, top=1in, left=0.5in, right=0.5in]{geometry}
\usepackage{array}   % for \newcolumntype macro
\newcolumntype{L}{>{$}l<{$}}

\theoremstyle{plain}
\newtheorem*{theorem}{Theorem}
\newtheorem*{proposition}{Proposition}
\newtheorem{exercise}{Exercise}
\newtheorem*{lemma}{Lemma}
\newtheorem*{corollary}{Corollary}

\theoremstyle{definition}
\newtheorem*{definition}{Definition}
\newtheorem*{example}{Example}

\theoremstyle{remark}
\newtheorem*{remark}{Remark}

\renewcommand{\phi}{\varphi}
\renewcommand{\epsilon}{\varepsilon}
\renewcommand{\emptyset}{\varnothing}

\newcommand{\RR}{\mathbb{R}}
\newcommand{\ZZ}{\mathbb{Z}}
\newcommand{\NN}{\mathbb{N}}
\newcommand{\CC}{\mathbb{C}}
\newcommand{\QQ}{\mathbb{Q}}

\DeclareMathOperator{\ima}{\text{im}}

\begin{document}

\topmargin=-40pt
\rhead{Henry Woodburn}
\lhead{Math 653}
\renewcommand{\headrulewidth}{1pt}
\renewcommand{\headsep}{20pt}
\thispagestyle{fancy}

{\Huge \bfseries \noindent Homework 8}

\begin{enumerate}
    \item[43.] Let $G$ be a simple group of order 168. Then the number of 7-sylow subgroups, $n_7$,
    is equal to $1 \mod 7$, and divides $24$. Then we must have $n_7 = 1, 8$. But $G$ is simple,
    so we cannot have $n_7 = 1$. Then $n_7 = 8$. Since a group of order 7 is cyclic, each 7-sylow
    subgroup is generated by each of its non-identity elements. Then the 8 subgroups must have trivial
    intersection. They each contain $6$ elements of order $7$, and $G$ can have no other 
    elements of order $7$ as they would generate other 7-sylow subgroups. Then $G$ has 
    $48$ elements of order $7$.

    \item[44.] (a.) Let $\QQ$ be the additive group of the rational numbers. Suppose there is 
    a basis $B$ such that every element of $\QQ$ is a unique finite sum of elements of $B$. Take
    some $a \in B$. Then we can write
    \[
        \frac{a}{2} = \sum_1^n a_i b_i
    \]
    for $a_i \in \ZZ$ and $b_i \in B$. But then 
    \[
        \sum_1^n 2a_i b_i = a,
    \]
    which is only possible if some $b_j = a$ and $b_i = 0$ for $i \neq j$, and if $2a_i = 1$. But
    this is a contradiction as $\frac{1}{2} \notin \ZZ$. 

    (b.) Let $G$ be the group of nonzero rational numbers under multiplication. I claim that $G$
    is generated by the set of prime numbers. Choose any $a \in G$. We can uniquely write
    \[
        a = \frac{p_1^{a_1}\cdots p_n^{a_n}}{q_1^{b_1}\cdots q_m^{b_m}} = 
        p_1^{a_1}\cdots p_n^{a_n}\cdot q_1^{-b_1}\cdots q_m^{-b_m}
    \]

\end{enumerate}

\end{document}