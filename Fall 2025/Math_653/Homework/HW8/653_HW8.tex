\documentclass[11pt, reqno]{article}

\usepackage{amsmath, amsthm, amssymb}
\usepackage{enumitem}
\usepackage{tcolorbox}
\usepackage{hyperref}
\usepackage{tikz}
\usepackage{tikz-cd}
\usetikzlibrary{arrows.meta}
\usepackage{mathrsfs}
\usepackage{fancyhdr}
\usepackage[bottom=0.75in, top=1in, left=0.5in, right=0.5in]{geometry}
\usepackage{array}   % for \newcolumntype macro
\newcolumntype{L}{>{$}l<{$}}

\theoremstyle{plain}
\newtheorem*{theorem}{Theorem}
\newtheorem*{proposition}{Proposition}
\newtheorem{exercise}{Exercise}
\newtheorem*{lemma}{Lemma}
\newtheorem*{corollary}{Corollary}

\theoremstyle{definition}
\newtheorem*{definition}{Definition}
\newtheorem*{example}{Example}

\theoremstyle{remark}
\newtheorem*{remark}{Remark}

\renewcommand{\phi}{\varphi}
\renewcommand{\epsilon}{\varepsilon}
\renewcommand{\emptyset}{\varnothing}

\newcommand{\RR}{\mathbb{R}}
\newcommand{\ZZ}{\mathbb{Z}}
\newcommand{\NN}{\mathbb{N}}
\newcommand{\CC}{\mathbb{C}}
\newcommand{\QQ}{\mathbb{Q}}

\DeclareMathOperator{\ima}{\text{im}}

\begin{document}

\topmargin=-40pt
\rhead{Henry Woodburn}
\lhead{Math 653}
\renewcommand{\headrulewidth}{1pt}
\renewcommand{\headsep}{20pt}
\thispagestyle{fancy}

{\Huge \bfseries \noindent Homework 8}

\begin{enumerate}
    \item[43.] Let $G$ be a simple group of order 168. Then the number of 7-sylow subgroups, $n_7$,
    is equal to $1 \mod 7$, and divides $24$. Then we must have $n_7 = 1, 8$. But $G$ is simple,
    so we cannot have $n_7 = 1$. Then $n_7 = 8$. Since a group of order 7 is cyclic, each 7-sylow
    subgroup is generated by each of its non-identity elements. Then the 8 subgroups must have trivial
    intersection. They each contain $6$ elements of order $7$, and $G$ can have no other 
    elements of order $7$ as they would generate other 7-sylow subgroups. Then $G$ has 
    $48$ elements of order $7$.

    \item[44.] (a.) Let $\QQ$ be the additive group of the rational numbers. Suppose there is 
    a basis $B$ such that every element of $\QQ$ is a unique finite sum of elements of $B$. Take
    some $a \in B$. Then we can write
    \[
        \frac{a}{2} = \sum_1^n a_i b_i
    \]
    for $a_i \in \ZZ$ and $b_i \in B$. But then 
    \[
        \sum_1^n 2a_i b_i = a,
    \]
    which is only possible if some $b_j = a$ and $b_i = 0$ for $i \neq j$, and if $2a_i = 1$. But
    this is a contradiction as $\frac{1}{2} \notin \ZZ$. 

    (b.) Let $G$ be the group of nonzero rational numbers under multiplication. I claim that $G$
    has a basis given by the set of prime numbers. Choose any $a \in G$. We can uniquely write
    \[
        a = \frac{p_1^{a_1}\cdots p_n^{a_n}}{q_1^{b_1}\cdots q_m^{b_m}} = 
        p_1^{a_1}\cdots p_n^{a_n}\cdot q_1^{-b_1}\cdots q_m^{-b_m},
    \]
    which is a countable product for any $a \in G$. Our basis is a countable and thus $G$ is a free group 
    with countable rank. 

    \item[45.] The element $1/n \in \QQ/\ZZ$ generates a subgroup of order $n$. 
    
    Moreover, let $A$ be a subgroup of $\QQ/\ZZ$ of order $n$. Then we have $nx = 0$ for any $x \in A$,
    and thus $nx = m$ for some $m \in \ZZ$. Then $x = \frac{m}{n}$ for some $0 \leq m < n$. But there are 
    $n$ distinct elements in $A$, so in fact $A = \{\frac{m}{n}: 0 \leq m < n\}$ and $A$ is cyclic and 
    generated by $\frac{1}{n}$.

    Then $\langle \frac{1}{n}\rangle$ is the unique subgroup of order $n$ in $\QQ/\ZZ$. 

    \item[46.] Let $\CC^x$ be the group of nonzero complex number under multiplication, and let 
    $\mathbb{T}$ be the subset of unit complex numbers. 

    Suppose $x \in \mathbb{T}_{tor}$. We can write $x = e^{ia\pi}$ for some $a \in \RR$, and thus
    \[
        x^n = e^{ina\pi} = 1 = e^{0}.
    \]
    Then we have $na = 2m\pi$ for some $m \in \ZZ$, and thus $a$ is the product of a rational
    number with $2\pi$. 

    Moreover, for every $\frac{p}{q} \in \QQ$, the element $e^{\frac{2p}{q}i\pi}$ is $q$-torsion, 
    so that the set of torsion elements is exactly the set $\{e^{2si\pi}: s \in \QQ\} \subset \mathbb{T}$.

    \item[47.] First we show that $\QQ/\ZZ$ is divisible. For any nonzero $x \in \QQ/\ZZ$, we have
    $x = \frac{p}{q}$ for $0 < q$ and $0 < p < q$. Then let $y = \frac{p}{nq}$ so that $ny = x$, 
    and clearly $y \in \QQ/\ZZ$ since $0 < q$ and $0 < p < q$ implies $0 < nq$ and $0 < p < nq$. 

    Now we show that if $A$ is an abelian group with subgroup $B$, and $\phi: B \rightarrow \QQ/\ZZ$ is a 
    homomorphism, then there is an extension $\psi: A \rightarrow \QQ/\ZZ$ such that $\phi$ is the 
    restriction of $\psi$ to $B$. 

    First define a set $B' = \{x \in A: nx \in B\ \text{for some}\ n\}$. We can extend $\phi$ to a map $\phi'$ on $B'$ using divisibility
    of $\QQ/\ZZ$, and clearly $B \subset B'$. For each $x \in B'$, let $n_x$ be the smallest positive 
    natural number such that $n_x x \in B$. If $x \in B$, then $n_x = 1$. For numbers in $\QQ/\ZZ$, we need to define
    what it means to divide by a positive integer. For $\frac{p}{q} \in \QQ/\ZZ$ with $p < q$ and a positive integer $n$, say
    \[
        \frac{\frac{p}{q}}{\ n\ } := \frac{p}{nq}.
    \]
    Also for $p/q, r/q \in \QQ/\ZZ$, we have
    \[
        \frac{p/q}{s} + \frac{r/q}{s} = \frac{p}{sq} + \frac{r}{sq} = \frac{p+r}{qs} = \frac{(p+r)/q}{s}
    \]
    so that the operation is well defined. 

    Then for any $x \in B'$, let

    \[
        \phi'(x) := \frac{\phi(n_x x)}{n_x}
    \]
    using the above definition for division by $n_x$. Since $n_x$ is unique for each $x$, the map
    $\phi'$ is well defined. We see that $\phi(nx) = n\phi(x)$, so we just need to show additivity. 

    First note that if $nx \in B$, then $n_x$ must divide $n$. Suppose here that $n = an_x + b$ for some $b < n_x$. Then
    $(an_x + b)(x) = (an_x)(x) + b(x) \in B$ and thus $b(x) \in B$ since $(an_x)(x) \in B$. But this is a contradiction 
    since $b < n_x$. 

    Now choose $x, y \in B'$. Letting $a = \operatorname{lcm}(n_x, n_y)$, we clearly have $a(x + y) \in B$,
    and thus $kn_{x + y} = a$ for some $k$. We have
    \[
        \phi'(x + y) = \frac{\phi(n_{x + y}(x + y))}{n_{x + y}} = \frac{\phi(kn_{x + y}x + kn_{x + y}y)}{kx_{x + y}}
        = \frac{\phi(ax)}{kn_{x + y}} + \frac{\phi(ay)}{kn_{x + y}} = \phi'(x) + \phi'(y),
    \]
    since by $a = p n_x$ for some $p$,
    \[
        \frac{\phi(ax)}{kn_{x + y}} = \frac{b\phi(n_x x)}{kn_{x + y}} = \frac{a\phi(x)}{k} = \phi'(x),
    \]
    and similar for $\phi(ay)$.

    Unless $B' = B$, we have extended the domain of $\phi$. If $B = B'$, we can 
    extend to $B \oplus \ZZ x$ for some $x \notin B$ by defining $\phi'(b + nx) = \phi(b)$
    where $b \in B$. It is clear that this is a homomorphism on this direct sum, and that
    $B \cup \ZZ x$ really is a direct sum. In either case,
    we have some extension of $\phi$. 

    We can order the set of extensions of $\phi$ by considering them as subsets of $A \times \QQ/\ZZ$
    and ordering by inclusion. We have already shown this set is nonempty, and the union of a chain
    is an upper bound. Then by Zorn's lemma, there is a maximal extension of $\phi$, name it $\tilde{\phi}$.

    Suppose $X := \text{dom}(\tilde{\phi})$ is not all of $A$. Then it is clear from above how we can
    further extend $\tilde{\phi}$ to include some $x \in A \setminus X$, either if $nx \in X$ for some $n$, or 
    if $\{nx: n \in \ZZ\} \cap X = 0$, contradicting the maximality of $\tilde{\phi}$. Then the
    domain of $\tilde{\phi}$ is $A$.

    The extension is not unique since in order to define $\phi'(x)$ for $nx \in B$, we had to make an
    arbitrary decision of which element should be the solution to $\phi(nx) = ny$.

    \item[48.] Let $G$ be a finite abelian $p$-group. We claim that $G$ is generated by its elements of 
    maximal order. Let $P$ be the set of elements of maximal order, say $p^n$, and let $a \notin P$
    be some element of order $p^r$, $r < n$. Then for any $r \leq s < n$, suppose that 
    \[
        p^s(a - b) = 0. 
    \] 
    Then we have $p^s(a) = 0 = p^s b$, contradicting the order of $b$. Then $(a-b)$ is order $p^n$ as well,
    and thus $a = (a-b) + b$, showing that any element of $G$ is a sum of elements of maximal order.

    \item[49.] (a.) Let $p$ be a prime. We want to find the number of subgroups of order $p$ in $\ZZ_p\oplus \ZZ_p$.
    Any such subgroup will be cyclic. 

    We can generate some by the elements $(1,a)$ for $0 \leq a < p$. These subgroups intersect trivially,
    since if $n(1, a) = m(1, b)$ for $0 < m, n < p$, then $(n, na) = (m, mb)$, and thus $n = m$ and $a = b$. 

    Another subgroup is generated by $(0, 1)$, which is distinct from the subgroups above which do not have
    $0$ in their first coordinate except in the identity element. 

    Finally, there are $p - 1$ distinct elements of order $p$ in each of the $p + 1$ subgroups, which together 
    with the identity give $p^2$ elements total. Then there can be no other subgroups of order $p$,
    so we have listed all $p+1$ of them. 

    (b.) We want to find the number of cyclic subgroups with order $p^2$ in $\ZZ_{p^2} \oplus \ZZ_{p^2}$. 
    We can list $p^2$ of them, with generators $(1, a)$ for $0 \leq a < p^2$. 
    More can be generated by $(np, 1)$ for $0 \leq n < p$. 

    We know each subgroup has $p(p-1)$ generators, and distinct subgroups cannot share generators. So far we have listed
    $p^2 + p = p(p+1)$ subgroups, each with $p(p-1)$ distinct elements of order $p^2$. This gives $p^2(p^2 - 1)$ 
    elements of order $p^2$. But $\ZZ_{p^2} \oplus \ZZ_{p^2}$ has exactly this many elements of order $p^2$, so 
    there can be no other subgroups and we have listed all $p^2 + p$ of them. 

    (c.) Let $G = \ZZ_{p^3}\oplus \ZZ_{p^2}$ be a group and let $H \simeq \ZZ_{p^2}$ be a subgroup. Then we know 
    $G/H$ will have order $p^3$, and will be isomorphic to either $\ZZ_{p^3}, \ZZ_{p}\oplus \ZZ_{p^2}$, or $\ZZ_p\oplus\ZZ_p\oplus\ZZ_p$.

    The first option is possible when we take $H = \langle (0,1)\rangle$.

    The second option occurs when $H = \langle (p,0)\rangle$. 

    The last option is not possible. $G$ is generated by two elements, while $\ZZ_p \oplus\ZZ_p\oplus\ZZ_p$ is generated
    by three. Then there cannot be a surjective homomorphism $G \rightarrow \ZZ_p \oplus\ZZ_p\oplus\ZZ_p$.
    
    \item[50.] Let $p$ be a prime. The tate group $T_p(\QQ/\ZZ)$ has elements given by sequences
    \[
        x = (x_0, x_1, x_2, \dots),
    \]
    where $x_{n-1} = px_{n}$, and $x_i \in p^{-i} \ZZ/\ZZ$.

    If $x$ has at least one $x_i$ nonzero, all $x_j$ must be nonzero for $j \leq i$. Then say $x_i$ has order $p^r$. 
    if $p^s(x_{i+1}) = 0$, this implies $p^{s-1}x_i = 0$, which is a contradiction unless $s-1 \geq r$. So
    the orders of elements increase by $p$ at each step. Then it is impossible for $x$ to have finite order 
    unless all of its elements are zero. 

    The group $\QQ/\ZZ$ can be embedded into $\mathbb{T}$ by $\frac{p}{q} \mapsto e^{\frac{2ip}{q}\pi}$.
    No other points in $\mathbb{T}$ are torsion. Then $T_p(\QQ/\ZZ)$ is isomorphic to $T_p(\mathbb{T})$.

    $T_p(\CC^\times)$ is also isomorphic to $T_p(\mathbb{T})$ since all of the torsion elements must lie 
    in the unit circle. 

\end{enumerate}

\end{document}