\documentclass[11pt, reqno]{article}

\usepackage{amsmath, amsthm, amssymb}
\usepackage{enumitem}
\usepackage{tcolorbox}
\usepackage{hyperref}
\usepackage{tikz}
\usetikzlibrary{arrows.meta}
\usepackage{mathrsfs}
\usepackage{fancyhdr}
\usepackage[bottom=0.75in, top=1in, left=0.5in, right=0.5in]{geometry}
\usepackage{array}   % for \newcolumntype macro
\newcolumntype{L}{>{$}l<{$}}

\theoremstyle{plain}
\newtheorem*{theorem}{Theorem}
\newtheorem*{proposition}{Proposition}
\newtheorem{exercise}{Exercise}
\newtheorem*{lemma}{Lemma}
\newtheorem*{corollary}{Corollary}

\theoremstyle{definition}
\newtheorem*{definition}{Definition}
\newtheorem*{example}{Example}

\theoremstyle{remark}
\newtheorem*{remark}{Remark}

\renewcommand{\phi}{\varphi}
\renewcommand{\epsilon}{\varepsilon}
\renewcommand{\emptyset}{\varnothing}

\newcommand{\RR}{\mathbb{R}}
\newcommand{\ZZ}{\mathbb{Z}}
\newcommand{\NN}{\mathbb{N}}
\newcommand{\CC}{\mathbb{C}}
\newcommand{\QQ}{\mathbb{Q}}

\DeclareMathOperator{\ima}{\text{im}}

\begin{document}

\topmargin=-40pt
\rhead{Henry Woodburn}
\lhead{Math 653}
\renewcommand{\headrulewidth}{1pt}
\renewcommand{\headsep}{20pt}
\thispagestyle{fancy}

{\Huge \bfseries \noindent Homework 5}

\begin{enumerate}
    \item[1.] Let $m \geq 2$ and set $\ZZ_m^* := \{k \in \ZZ_m: \gcd(k,m) = 1\}$.
    \begin{enumerate}
        \item[a.] First we show that every element of $\ZZ_m^*$ generates 
        $\ZZ_m$. We want to show there are $m$ distinct cosets in the cyclic subgroup generated by $k$.
        Let $n$ be the smallest positive integer such that $nk + m\ZZ = m\ZZ$. 
        Then $nk$ is the least common multiple of $k$ and $m$. But since $\gcd(k,m) = 1$, 
        we must have $nk = mk$ and $n = m$. Then $k$ generates $\ZZ_m$.

        Now take some element $l$ such that $\gcd(l,m) > 1$. Then there is some $r$ such that 
        $m = nr$ and $l = pr$. Then $nl$ is a multiple of both $l$ and $m$, and $n < m$. Then
        $n$ is the order of $l$, and $l$ does not generate $\ZZ_m$.

        \item[b.] We will show $\ZZ_m^*$ is a group under multiplication. Clearly it contains the
        identity $1$ since $\gcd(1,m) = 1$. 

        To show inverses, let $n \in \ZZ_m^*$. Then by applying the euclidean algorithm, there
        are integers $a$ and $b$ such that $an + bm = 1$ and thus $an = 1 - bm$. 
        Then we have $an + m\ZZ = 1 + m\ZZ$, and
        inverse of $a$ is $n$ in $\ZZ_m^*$. 

        Finally suppose $a, b \in \ZZ_m^*$. If $p$ is a prime which divides $m$ and $ab$, then 
        $p$ must divide either $a$ or $b$. But this is impossible. Then $ab \in \ZZ_m^*$.

        \item[c.] Suppose $\gcd(a,m) = 1$. Then any element of $a + m\ZZ$ is also relatively prime 
        with $m$, and thus $a + m\ZZ$ generates $\ZZ_m^*$. Then the cyclic group generated by 
        $a + m\ZZ$ under multiplication must be at most order $m$, and thus $(a + m\ZZ)^{\phi(m)} = 1 + m\ZZ$,
        where $\phi(m)$ is the order of $\ZZ_m^*$. But $(a + m\ZZ)^{\phi(m)} = a^{\phi(m)} + m\ZZ = 1 + m\ZZ$,
        and we have $a^{\phi(m)} \equiv m\mod m$.

        \item[d.] Suppose $\gcd(a,b) = 1$. Then $\ZZ_a \times \ZZ_b$ is a group of order $ab$. Moreover,
        the order of the element $(1_a,1_b)$ is the least common multiple of the orders $a$ and $b$
        of $1_a$ and $1_b$, which must be $ab$. We have shown $(1,1)$ generates $\ZZ_a \times \ZZ_b$, and 
        thus $\ZZ_a \times \ZZ_b$ is a cyclic group isomorphic to $\ZZ_{ab}$. 

        Then $\ZZ_a \times \ZZ_b$ has the same number of generators as $\ZZ_{ab}$. Let $p \in \ZZ_a$
        and $q \in \ZZ_b$ both be generators with order $a$ and $b$ respectively. By homework $2$ problem 
        $3$, the order of $(p,q)$ is the least common multiple of $a$ and $b$, $ab$. Then $(p,q)$ generates
        $\ZZ_a \times \ZZ_b$, along with every other pair of generators. Then there are $\phi(a)\phi(b)$
        generators of $\ZZ_a \times \ZZ_b$ and thus of $\ZZ_{ab}$. Finally, we have shown that this 
        number is precisely $\phi(ab)$, so that $\phi(ab) = \phi(a)\phi(b)$.
        \bigbreak 
        Let $p$ be a prime number. Then the only divisors of $p$ are itself and one. Then 
        every number $1, 2, \dots, p-1$ is relatively prime with $p$ and $\phi(p) = p-1$.
        \bigbreak
        To calculate $\phi(p^n)$, note that if we have $\gcd(m, p^n) > 1$ for some $1 \leq m \leq p^n$, then $m$ must be 
        a multiple of $p$ less than or equal to $p^n$. There are $p^{n-1}$ such numbers. 
        Then the remaining $p^n - p^{n-1}$ numbers are relatively prime to $p^n$ and $\phi(p^n) = p^n - p^{n-1}$.
        \bigbreak 
        Combining the above results, let $m = p_1^{a_1}\cdot p_2^{a_2}\cdots p_n^{a_n}$. In the above
        result, notice that $\phi(p^n) = p^n - p^{n-1} = p^n(1 - \frac{1}{p})$. Then we can write 
        \[
            \phi(m) = \prod_{i = 1}^n \phi(p_i^{a_i}) = \prod_{i = 1}^n p_i^{a_i} (1 - \frac{1}{p_i}) = m \prod_{i = 1}^n (1 - \frac{1}{p_i}).
        \]
        \bigbreak
        Let $a \in \ZZ$ and let $p$ be prime. If $a$ is a multiple of $p$, we have 
        \[
            a^p \equiv 0\ \text{mod}\ p = a\ \text{mod}\ p.
        \]
        Otherwise, $a$ is relatively prime with $p$, and we have $a^{\phi(p)} = a^{p - 1} = 1\ \text{mod}\ p$ and thus 
        \[
            a^p = a\ \text{mod}\ p.
        \]

    \end{enumerate}

    \item[2.] Let $\mathfrak{H} := \{z \in \CC: \operatorname{Im}(z) > 0\}$ be the upper half plane 
    in the complex numbers. Let $G := SL(2, \RR)$. 
    \[
        \text{For}\ z \in \CC\ \text{and}\ \alpha = \begin{pmatrix} a & b \\ c & d \end{pmatrix},\ 
        \text{let}\ a.z = \frac{az + b}{cz + d}.
    \]
    We will show that this defines an action on $\mathfrak{H}$. First we need to check that 
    the action sends $\mathfrak{H}$ into $\mathfrak{H}$. Indeed, if $\alpha = \left(\begin{smallmatrix}
        a & b \\ c & d
    \end{smallmatrix}\right) \in SL(2, \RR)$ and $z \in \mathfrak{H}$, then 
    \[
        \operatorname{Im}\alpha. z = \operatorname{Im}\left(\frac{az + b}{cz + d}\right) = \frac{ad - bc}{|cz + d|^2}\operatorname{Im}(z) > 0.
    \]
    
    For $e = \begin{pmatrix}
        1 & 0 \\ 0 & 1
    \end{pmatrix}$, we have 
    \[
        e.z = \frac{z}{1} = z.
    \]
    Finally, if $\alpha = \left(\begin{smallmatrix}
            a & b \\ c & d
        \end{smallmatrix}\right)$ and $\beta = \left(\begin{smallmatrix}
            e & f \\ g & h
        \end{smallmatrix}\right)$,
    then 
    \[
        \beta .(\alpha . z) = \frac{(ea + cf)z + (eb + fd)}{(ag + ch)z + (gb + dh)} = (\beta\alpha).z
    \]
    as desired.

    Now we will show the isotropy group of $i$ is the group 
    \[
        K := \left\{\begin{pmatrix}
            \cos(\theta) & \sin(\theta) \\
            -\sin(\theta) & \cos(\theta)
        \end{pmatrix}: \theta \in \RR\right\}.
    \]
    Suppose $\frac{ai + b}{ci + d} = i$. Then $ai + b = di - c$, and we must have 
    \begin{align*}
        &a = d\\
        &b = -c.
    \end{align*}
    Then together with $ad - bc = 1$, we must have $a^2 + b^2 = 1$. Then it is clear that the values of 
    $a, b$ must equal $\cos(\theta)$ and $\sin(\theta)$ respectively, with $c$ and $d$ determined by 
    the above equations as well.

    Finally we will show that $G$ acts transitively on $\mathfrak{H}$. We must show that there is only one orbit. 
    Equivalently, we may show that any element of $\mathfrak{H}$ may be obtained from the action of $G$ on one element,
    namely $i$. This means that every element of $\mathfrak{H}$ is in the same orbit, and thus there is only one. 

    We need to solve $\frac{ai + b}{ci + d} = z$ for an arbitrary $z = p + qi \in \mathfrak{H}$. we have 
    $ai + b = cpi - cq + dp + dqi$, and thus 
    \begin{align*}
        &b = dp - cq\\
        &a = cp + dq.
    \end{align*}
    Here we may set $c = 0$ as not all four values $a, b, c, d$ are uniquely determined. 

    Then $b = dp, a = dq$, and thus using $ad - bc = 1$, we have $d^2q = 1$. Here, the positivity of $q$ 
    ensures that $d$ is a real number. In total, we get $d = \frac{1}{\sqrt{q}}$, $a = \sqrt{q}$, and $b = \frac{p}{\sqrt{q}}$,
    and indeed we get that 
    \[
        \frac{\sqrt{q}i + \frac{p}{\sqrt{q}}}{\frac{1}{\sqrt{q}}} = p + qi.
    \]

    \item[3.] Let $G$ be a group and $H$ a subgroup. Let $\operatorname{core}(H)$ be the 
    intersection of all conjugates of $H$ by elements of $G$. Let $S$ be the set of left cosets 
    of $H$ in $G$. For $g \in G$, define $g_*: S \rightarrow S$ by $g_*(xH) = gxH$.
    \begin{enumerate}
        \item[a.] We will show that $g_*$ is an element of the symmetric group on $S$. We know 
        that left multiplication by $g$ is a bijection on $G$, so it is clearly surjective 
        on $S$. Also, it could not map two cosets to the same coset, as this would
        contradict the injectivity of left multiplication by $g$. Then we only must show 
        that the map $g_*$ is well defined, sending different representatives of the same coset
        to the same coset. Suppose $aH = bH$, so that $b^{-1}a \in H$. Then $b^{-1}g^{-1}ga = 
        (gb)^{-1}ga \in H$, and thus $g_*(aH) = g_*(bH)$.

        \item[b.] Let $\phi: G \rightarrow \operatorname{sym}(S)$ be the map $g \mapsto g_*$.
        To show that $\phi$ is a homomorphism, we clearly have $\phi(e) = I$, the identity
        map on $S$. 

        Also, for any coset $aH$, we have $\phi(gh)(aH) = (gh)_*(aH) = ghaH = g_*(h_*(aH))
        = (g_*\circ h_*)(aH) = \phi(g)\circ \phi(h)(aH)$, and thus $\phi(gh) = \phi(g)\circ\phi(h)$
        and $\phi$ is a homomorphism $G \rightarrow \operatorname{sym}(S)$.
        \bigbreak
        Lastly we will show that the kernel of $\phi$ is exactly the core of $H$. Suppose 
        $\phi(g) = I$. Then $gaH = aH$ for all $a \in G$, and we have $ga \in aH$, or 
        $g \in aHa^{-1}$. Then $g$ is in the core of $H$. 

        Alternatively, suppose $g \in \operatorname{core}(H)$. Then $g \in aHa^{-1}$ for all $a \in G$,
        and thus $ga \in aH$ and we have $gaH = aH$. Then $\phi(g) = I$. 

        Then we have shown that $\ker \phi = \operatorname{core}(H)$.
    \end{enumerate}
\end{enumerate}

\end{document}