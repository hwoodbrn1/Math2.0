\documentclass[11pt, reqno]{article}

\usepackage{amsmath, amsthm, amssymb}
\usepackage{enumitem}
\usepackage{tcolorbox}
\usepackage{hyperref}
\usepackage{tikz}
\usetikzlibrary{arrows.meta}
\usepackage{mathrsfs}
\usepackage{fancyhdr}
\usepackage[bottom=0.75in, top=1in, left=0.5in, right=0.5in]{geometry}
\usepackage{array}   % for \newcolumntype macro
\newcolumntype{L}{>{$}l<{$}}

\theoremstyle{plain}
\newtheorem*{theorem}{Theorem}
\newtheorem*{proposition}{Proposition}
\newtheorem{exercise}{Exercise}
\newtheorem*{lemma}{Lemma}
\newtheorem*{corollary}{Corollary}

\theoremstyle{definition}
\newtheorem*{definition}{Definition}
\newtheorem*{example}{Example}

\theoremstyle{remark}
\newtheorem*{remark}{Remark}

\renewcommand{\phi}{\varphi}
\renewcommand{\epsilon}{\varepsilon}
\renewcommand{\emptyset}{\varnothing}

\newcommand{\RR}{\mathbb{R}}
\newcommand{\ZZ}{\mathbb{Z}}
\newcommand{\NN}{\mathbb{N}}
\newcommand{\CC}{\mathbb{C}}
\newcommand{\QQ}{\mathbb{Q}}

\DeclareMathOperator{\ima}{\text{im}}

\begin{document}

\topmargin=-40pt
\rhead{Henry Woodburn}
\lhead{Math 653}
\renewcommand{\headrulewidth}{1pt}
\renewcommand{\headsep}{20pt}
\thispagestyle{fancy}

{\Huge \bfseries \noindent Homework 5}

\begin{enumerate}
    \item[1.] Let $m \geq 2$ and set $\ZZ_m^* := \{k \in \ZZ_m: \gcd(k,m) = 1\}$.
    \begin{enumerate}
        \item[a.] First we show that every element of $\ZZ_m^*$ generates 
        $\ZZ_m$. We want to show there are $m$ distinct cosets in the cyclic subgroup generated by $k$.
        Let $n$ be the smallest positive integer such that $nk + m\ZZ = m\ZZ$. 
        Then $nk$ is the least common multiple of $k$ and $m$. But since $\gcd(k,m) = 1$, 
        we must have $nk = mk$ and $n = m$. Then $k$ generates $\ZZ_m$.

        Now take some element $l$ such that $\gcd(l,m) > 1$. Then there is some $r$ such that 
        $m = nr$ and $l = pr$. Then $nl$ is a multiple of both $l$ and $m$, and $n < m$. Then
        $n$ is the order of $l$, and $l$ does not generate $\ZZ_m$.

        \item[b.] We will show $\ZZ_m^*$ is a group under multiplication. Clearly it contains the
        identity $1$ since $\gcd(1,m) = 1$. 

        To show inverses, let $n \in \ZZ_m^*$. Then by applying the euclidean algorithm, there
        are integers $a$ and $b$ such that $an + bm = 1$ and thus $an = 1 - bm$. 
        Then we have $an + m\ZZ = 1 + m\ZZ$, and
        inverse of $a$ is $n$ in $\ZZ_m^*$. 

        Finally suppose $a, b \in \ZZ_m^*$. If $p$ is a prime which divides $m$ and $ab$, then 
        $p$ must divide either $a$ or $b$. But this is impossible. Then $ab \in \ZZ_m^*$.

        \item[c.] Suppose $\gcd(a,m) = 1$. Then any element of $a + m\ZZ$ is also relatively prime 
        with $m$, and thus $a + m\ZZ$ generates $\ZZ_m^*$. Then the cyclic group generated by 
        $a + m\ZZ$ under multiplication must be at most order $m$, and thus $(a + m\ZZ)^{\phi(m)} = 1 + m\ZZ$,
        where $\phi(m)$ is the order of $\ZZ_m^*$. But $(a + m\ZZ)^{\phi(m)} = a^{\phi(m)} + m\ZZ = 1 + m\ZZ$,
        and we have $a^{\phi(m)} \equiv m\mod m$.

        \item[d.] Suppose $\gcd(a,b) = 1$. Then $\ZZ_a \times \ZZ_b$ is a group of order $ab$. Moreover,
        the order of the element $(1_a,1_b)$ is the least common multiple of the orders $a$ and $b$
        of $1_a$ and $1_b$, which must be $ab$. We have shown $(1,1)$ generates $\ZZ_a \times \ZZ_b$, and 
        thus $\ZZ_a \times \ZZ_b$ is a cyclic group isomorphic to $\ZZ_{ab}$. 

        Then $\ZZ_a \times \ZZ_b$ has the same number of generators as $\ZZ_{ab}$. Let $p \in \ZZ_a$
        and $q \in \ZZ_b$ both be generators with order $a$ and $b$ respectively. By homework $2$ problem 
        $3$, the order of $(p,q)$ is the least common multiple of $a$ and $b$, $ab$. Then $(p,q)$ generates
        $\ZZ_a \times \ZZ_b$, along with every other pair of generators. Then there are $\phi(a)\phi(b)$
        generators of $\ZZ_a \times \ZZ_b$ and thus of $\ZZ_{ab}$. Finally, we have shown that this 
        number is precisely $\phi(ab)$, so that $\phi(ab) = \phi(a)\phi(b)$.
        \bigbreak 
        Let $p$ be a prime number. Then the only divisors of $p$ are itself and one. Then 
        every number $1, 2, \dots, p-1$ is relatively prime with $p$ and $\phi(p) = p-1$.
        \bigbreak
        To calculate $\phi(p^n)$, note that if we have $\gcd(m, p^n) > 1$ for some $1 \leq m \leq p^n$, then $m$ must be 
        a multiple of $p$ less than or equal to $p^n$. There are $p^{n-1}$ such numbers. 
        Then the remaining $p^n - p^{n-1}$ numbers are relatively prime to $p^n$ and $\phi(p^n) = p^n - p^{n-1}$.
        \bigbreak 
        Combining the above results, let $m = p_1^{a_1}\cdot p_2^{a_2}\cdots p_n^{a_n}$. In the above
        result, notice that $\phi(p^n) = p^n - p^{n-1} = p^n(1 - \frac{1}{p})$. Then we can write 
        \[
            \phi(m) = \prod_{i = 1}^n \phi(p_i^{a_i}) = \prod_{i = 1}^n p_i^{a_i} (1 - \frac{1}{p_i}) = m \prod_{i = 1}^n (1 - \frac{1}{p_i}).
        \]
        \bigbreak
        Let $a \in \ZZ$ and let $p$ be prime. If $a$ is a multiple of $p$, we have 
        \[
            a^p \equiv 0\ \text{mod}\ p = a\ \text{mod}\ p.
        \]
        Otherwise, $a$ is relatively prime with $p$, and we have $a^{\phi(p)} = a^{p - 1} = 1\ \text{mod}\ p$ and thus 
        \[
            a^p = a\ \text{mod}\ p.
        \]

    \end{enumerate}
\end{enumerate}

\end{document}