\documentclass[11pt, reqno]{article}

\usepackage{amsmath, amsthm, amssymb}
\usepackage{enumitem}
\usepackage{bookmark}
\usepackage{fullpage}
\usepackage{tcolorbox}
\usepackage{hyperref}
\usepackage{tikz}
\usetikzlibrary{arrows.meta}
\usepackage{pdfpages}
\usepackage{mathrsfs}
\usepackage{fancyhdr}
\usepackage[bottom=0.5in, top=1in, left=0.5in, right=0.5in]{geometry}
\usepackage{array}   % for \newcolumntype macro
\newcolumntype{L}{>{$}l<{$}}

\begin{document}

\topmargin=-40pt
\rhead{Henry Woodburn}
\lhead{Math 653}
\renewcommand{\headrulewidth}{1pt}
\renewcommand{\headsep}{20pt}
\thispagestyle{fancy}

\section*{Homework 2}

\begin{enumerate}
    \item Let $G$ be a group, $H \subset G$ be a subgroup, and $g \in G$. We will show
    $gHg^{-1} = \{ghg^{-1}: h \in H\}$ is a subgroup. 

    It contains the identity since $geg^{-1} = gg^{-1} = e$. 
    
    It contains inverses: let $gag^{-1}$
    be some element of $gHg^{-1}$, with $a \in H$. Then since $H$ is a subgroup, $a^{-1} \in H$, 
    and $ga^{-1}g^{-1} \in gHg^{-1}$ is the inverse of $gag^{-1}$ since $gag^{-1}ga^{-1}g^{-1}
    = gaa^{-1}g^{-1} = e$.

    Finally, $gHg^{-1}$ is closed under multiplication since for $gag^{-1}, gbg^{-1} \in gHg^{-1}$,
    $gag^{-1}gbg^{-1} = gabg^{-1}$. As $H$ is a subgroup, we have $ab \in H$ and thus $gabg^{-1} \in gHg^{-1}$.

    Now we show that $gHg^{-1}$ is isomorphic to $H$. I claim the map $\varphi: H \rightarrow gHg^{-1}$
    defined by $a \mapsto gag^{-1}$ is an isomorphism of subgroups. It is a group homomorphism
    since $\varphi(ab) = gabg^{-1} = gag^{-1}gbg^{-1} = \varphi(a)\varphi(b)$. It is injective: 
    suppose $\varphi(a) = e$. Then $gag^{-1} = e$, and thus $a = g^{-1}eg = e$. It is surjective: 
    choose any $gag^{-1} \in gHg^{-1}$. Then $\varphi(a) = gag^{-1}$. Thus $\varphi$ is an isomorphism.

    \item Let $G$ be a group and $H, K$ be proper subgroups of $G$ such that $G = H \cup K$. 
    We cannot have $H \subset K$ or $K \subset H$ since this would mean either $H = G$ or $K = G$,
    a contradiction. 

    Then there exists some $h \in H$ such that $h \notin K$. For any $k \in K$, I claim that 
    $hk \notin K$. Suppose it is. Then there is some $k' \in K$ such that $hk = k'$, and $h = k'k^{-1}$.
    Since $K$ is a subgroup, this implies $h \in K$, a contradiction. 

    There also exists $k \in K$ with $k \notin H$. Applying the above reasoning also to $K$, we have $hk \notin H$ 
    and $hk \notin K$. But $hk \in G$, which contradicts that $G = H \cup K$. Then we are done.\\
    --------\\
    We now show that $\mathbb{Z}\oplus\mathbb{Z}$ is the union of three proper subgroups. 
    Let 
    \[
    A_1 = \langle(1,1), (0,2)\rangle, A_2 = \langle(2,2), (0,1)\rangle, 
    A_3 = \langle(2,2), (1,0)\rangle,
    \]
    each subgroups generated by two elements. They are each proper subgroups: 
    $A_1$ does not contain elements of the form $(a,a+1)$, $A_2$ does not contain
    elements of the form $(2a + 1, b)$, and similarly $A_3$ does not contain elements
    $(a, 2b + 1)$. 
    
    I claim that 
    $\mathbb{Z}\otimes\mathbb{Z}$ is the union of these three subgroups. 

    Choose any $(x,y) \in \mathbb{Z}\oplus\mathbb{Z}$. If $x = 2a$ for some $a \in \mathbb{Z}$,
    we can express $(x,y) = (2a, 2a + b) = a(2,2) + b(0,1)$, where $b = y-x$. Then $(x,y) \in A_2$.

    If this is not the case, then if $y = 2b$ for some $b \in \mathbb{Z}$, we can 
    write $(x,y) = (a + 2b, 2b) = a(1,0) + b(2,2)$, where $a = x-y$, and $(x,y) \in A_3$.

    Finally, if neither of these conditions are true, both $x$ and $y$ must be odd, and 
    we have $y - x = 2a$ for some $a \in \mathbb{Z}$. Then we can write 
    $(x,y) = (x, x + 2a) = x(1,1) + a(0,2)$, and $(x,y) \in A_1$
    
    We have shown that every element of $\mathbb{Z}\oplus\mathbb{Z}$ is an element of at least 
    one of the three proper subgroups. Then $\mathbb{Z}\oplus\mathbb{Z} = A_1 \cup A_2 \cup A_3$
    as sets.

    \item Let $A$ and $B$ be groups with elements $a \in A$ and $b \in B$ and consider $(a,b) \in A\times B$. 
    If either $a$ or $b$ has infinite order, the order of $(a,b)$ must be infinite. 
    
    Otherwise, let $\alpha$ and $\beta$ be the orders of $a$ and $b$, respectively, 
    and $m = \operatorname{lcm}(\alpha, \beta)$, the least common multiple. 
    Then $m = p\alpha = q\beta$ for some $p,q \in \mathbb{N}$, and $(a,b)^m = (a^m, b^m) = ({(a^\alpha)}^p, {(b^\beta)}^q)
    = (e,e)$. Since $m$ is by definition the smallest element for which $a^m = b^m = e$, it must be the order of $(a,b)$.

    \item Let $G = \{e, a, b, c\}$ be a group of four elements with identity $e$. Suppose $G$ has no element of order
    4. We will not assume that the order of a subgroup divides the order of a group.

    Suppose $a$ has order 3, so that $a^3 = e$. Then WLOG assume $a^2 = b$. $\langle a \rangle$ is a cyclic subgroup
    of $G$ of order 3. I claim that the element $ca$ is not in $\langle a \rangle$. If $ca = e$, then $c = a^2 = b$,
    a contradiction. If $ca = a$, then $c = e$, also a contradiction. Finally if $ca = a^2$, then $c = a$, also a 
    contradiction, and we have shown $ca \notin \langle a \rangle$.

    Then $ca = c$, but this is impossible since $a \neq e$. Then $a$ is not order $3$. The same argument holds for $b$ and $c$.

    Then each non identity element has order $2$. We must have $ab = c$, since neither $a$ nor $b$ are the identity,
    and $ab = e$ implies $a = b$ which is impossible. Similarly, $ba = c$. 

    The same is true as well for the other products of nonidentity elements. This shows $G$ is abelian, and 
    we have completely determined the group structure of $G$. 

    \item Let $\mathbb{Q}$ be the rational numbers and let $A = \langle \frac{a_1}{b_1}, \dots, \frac{a_n}{b_n}\rangle$
    be a finitely generated subgroup. 

    Let $m = \operatorname{lcm} (b_1, \dots, b_n)$.
    Then $\langle \frac{a_1}{b_1}, \dots, \frac{a_n}{b_n}\rangle = \frac{1}{m}\langle p_1 a_1, \dots, p_n a_n\rangle$
    for $p_1, \dots, p_n \in \mathbb{Z}$. 

    Now let $n = \gcd(p_1 a_1, \dots, p_n a_n)$, so that 
    $\frac{1}{m}\langle p_1 a_1, \dots, p_n a_n\rangle = \frac{n}{m}\langle p_1 q_1, \dots, p_n q_n\rangle$
    again for some $q_1, \dots, q_n \in \mathbb{Z}$. 

    Then $\gcd(p_1 a_1, \dots, p_n a_n) = n$ implies that 
    \[
    \gcd\left(\frac{p_1 a_1}{n}, \dots, \frac{p_n a_n}{n}\right) = \gcd\left(p_1 q_1, \dots, p_n q_n\right) = 1,
    \]
    and by Bezout's identity, there exist integers $r_1, \dots, r_n$ such that $r_1 p_1 q_1 + \cdots + r_n p_n q_n = 1$.
    In other words, $1 \in \langle p_1 q_1, \dots, p_n q_n \rangle$ and therefore $\langle p_1 q_1, \dots, p_n q_n\rangle = \langle 1 \rangle$.\

    Finally, we have
    \[
    \left\langle \frac{a_1}{b_1}, \dots, \frac{a_n}{b_n}\right\rangle = \frac{n}{m}\langle p_1 q_1, \dots, p_n q_n\rangle = 
    \left\langle \frac{n}{m}\right\rangle
    \]

    and thus $\langle \frac{a_1}{b_1}, \dots, \frac{a_n}{b_n}\rangle$ is cyclic.

    Now we will show that there is a subgroup of $\mathbb{Q}$ which is not finitely 
    generated. Let $A \subset \mathbb{Q}$ be the rational numbers whose denominator
    is a power of $2$. We just showed that such a subgroup being finitely generated is equivalent 
    to it being cyclic. Then we only need to show $A$ has no generator.

    Suppose it does, and $A = \langle x \rangle$ for some $x \in \mathbb{Q}$. We must have $x \in A$, 
    so $x = \frac{a}{2^b}$ for $a, b \in \mathbb{Z}$. Then there is some $c \in \mathbb{Z}$ such that 
    $\frac{1}{2^{b + 1}} = \frac{ca}{2^{b}}$, so $ca = \frac{1}{2}$ which is impossible if $a$ and $c$ are
    integers. 

    \item Let $D_4$ be the group generated by $S := \left(\begin{smallmatrix}0 & 1 \\ -1 & 0\end{smallmatrix}\right)$
    and $R = \left(\begin{smallmatrix}0 & 1 \\ 1 & 0\end{smallmatrix}\right)$ under matrix multiplication. 

    Since $S$ and $R$ have a determinate of $\pm 1$, we know that any element of $D_4$ must also have a determinate of $\pm 1$.
    Also, any matrix in this group must have entries $\pm 1$ since $S$ and $R$ only contain these values. Then we can only have matrices of the form
    $\left(\begin{smallmatrix}\pm 1 & 0 \\0 & \pm 1\end{smallmatrix}\right)$ or 
    $\left(\begin{smallmatrix} 0 & \pm 1 \\ \pm 1 & 0\end{smallmatrix}\right)$.

    We must also show that each of the 8 possibilities can be generated by $S$ and $R$. We have 
    \begin{align*}
        & S = \begin{pmatrix}
        0 & 1 \\ -1 & 0
        \end{pmatrix} \qquad
        S^2 = \begin{pmatrix}
            -1 & 0 \\ 0 & -1
        \end{pmatrix}\\
        & S^3 = \begin{pmatrix}
            0 & -1 \\ 1 & 0
        \end{pmatrix}\qquad 
        S^4 = I = \begin{pmatrix}
            1 & 0 \\ 0 & 1
        \end{pmatrix}\\
        & R 
         = \begin{pmatrix}
        0 & 1 \\ 1 & 0 
        \end{pmatrix}\qquad
        SR = \begin{pmatrix}
            1 & 0 \\ 0 & -1
        \end{pmatrix}\\
        & S^2R = \begin{pmatrix}
            0 & -1 \\ -1 & 0
        \end{pmatrix}\qquad 
        S^3R = \begin{pmatrix}
            -1 & 0 \\ 0 & 1
        \end{pmatrix}\\
    \end{align*}
    
    It is nonabelian since $RS = S^3R$ differs from $SR$ above.

    Each element of $D_4$ permutes the vertices $(\pm 1, \pm 1)$ of the square. The linear maps are injective,
    and any vector $(\pm 1, \pm 1)$ is sent to another $(\pm 1, \pm 1)$. The first four elements above correspond to 
    rotations about the origin. The next four correspond to reflection about $x = y$, reflection about $x = 0$,
    reflection about $y = -x$, and reflection about $y = 0$, respectively.

    \item Let $Q_8$ be the group generated by the matrices $\textbf{i} := \left(\begin{smallmatrix}0 & 1 \\ -1 & 0\end{smallmatrix}\right)$
    and $\textbf{j} := \left(\begin{smallmatrix} 0 & i \\ i & 0\end{smallmatrix}\right)$, where $i = \sqrt{-1}$. We have relations $\textbf{i}^4 = 
    \textbf{j}^4 = e$, and $\textbf{i}^2 = \textbf{j}^2$, as well as $\textbf{ij} = \textbf{j}\textbf{i}^3$
    and $\textbf{ji} = \textbf{i}^3\textbf{j}$.

    From the last relation, we can conclude that any element of $Q_8$ is of the form $\textbf{i}^a \textbf{j}^b$. This
    is because for any combination of elements $\textbf{i}$ and $\textbf{j}$, we can move all the $\textbf{i}'s$ to the right 
    as many times as needed, gaining exponents each shift. Also, $a,b \in \{1, 2, 3, 4\}$ because of the first relation. 

    Note that the first and second relations imply that $\textbf{i}^2\textbf{j}^2 = e$.
    Then we can rewrite $\textbf{i}^a\textbf{j}^b = \textbf{i}^{a-2}\textbf{a}^2\textbf{j}^2\textbf{j}^{b-2} = 
    \textbf{i}^{a-2}\textbf{j}^{b-2}$. This imposes a restriction on the total number of elements $\textbf{i}^a\textbf{j}^b$
    so that there can be at most $8$, since $8$ of the possibilities are equal to $8$ others. 

    We can also write out $8$ elements:
    \begin{align*}
        & \textbf{i} = \begin{pmatrix}
        0 & 1 \\ -1 & 0
        \end{pmatrix} \qquad
        \textbf{i}^2 = \begin{pmatrix}
            -1 & 0 \\ 0 & -1
        \end{pmatrix}\\
        & \textbf{i}^3 = \begin{pmatrix}
            0 & -1 \\ 1 & 0
        \end{pmatrix}\qquad 
        \textbf{i}^4 = I = \begin{pmatrix}
            1 & 0 \\ 0 & 1
        \end{pmatrix}\\
        & \textbf{j} = \begin{pmatrix}
            0 & i \\ i & 0
        \end{pmatrix} \qquad
        \textbf{j}^3 = \begin{pmatrix}
            0 & -i \\ -i & 0
        \end{pmatrix}\\
        & \textbf{ij} = \begin{pmatrix}
            i & 0 \\ 0 & -i
        \end{pmatrix}\qquad
        \textbf{ji} = \begin{pmatrix}
            -i & 0 \\ 0 & i
        \end{pmatrix}.
    \end{align*}

    These these must be the 8 elements. $Q_8$ is nonabelian since $ij \neq ji$.

    Note that $D_4$ has $3$ elements of order $2$, listed above as $S^2, R,$ and $S^3R$, while the only 
    element of $Q_8$ which has order 2 is $\textbf{i}^2$. Then these groups are not isomorphic. 

\end{enumerate}

\end{document}