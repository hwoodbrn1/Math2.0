\documentclass[11pt, reqno]{article}

\usepackage{amsmath, amsthm, amssymb}
\usepackage{enumitem}
\usepackage{tcolorbox}
\usepackage{hyperref}
\usepackage{tikz}
\usepackage{tikz-cd}
\usetikzlibrary{arrows.meta}
\usepackage{mathrsfs}
\usepackage{fancyhdr}
\usepackage[bottom=0.75in, top=1in, left=0.5in, right=0.5in]{geometry}
\usepackage{array}   % for \newcolumntype macro
\newcolumntype{L}{>{$}l<{$}}

\theoremstyle{plain}
\newtheorem*{theorem}{Theorem}
\newtheorem*{proposition}{Proposition}
\newtheorem{exercise}{Exercise}
\newtheorem*{lemma}{Lemma}
\newtheorem*{corollary}{Corollary}

\theoremstyle{definition}
\newtheorem*{definition}{Definition}
\newtheorem*{example}{Example}

\theoremstyle{remark}
\newtheorem*{remark}{Remark}

\renewcommand{\phi}{\varphi}
\renewcommand{\epsilon}{\varepsilon}
\renewcommand{\emptyset}{\varnothing}

\newcommand{\RR}{\mathbb{R}}
\newcommand{\ZZ}{\mathbb{Z}}
\newcommand{\NN}{\mathbb{N}}
\newcommand{\CC}{\mathbb{C}}
\newcommand{\QQ}{\mathbb{Q}}

\DeclareMathOperator{\ima}{\text{im}}

\begin{document}

\topmargin=-40pt
\rhead{Henry Woodburn}
\lhead{Math 653}
\renewcommand{\headrulewidth}{1pt}
\renewcommand{\headsep}{20pt}
\thispagestyle{fancy}

{\Huge \bfseries \noindent Homework 9}

\begin{enumerate}
    \item[1.] Let $p$ be a prime. We have natural surjections $f_m^n: \ZZ/p^n \ZZ \twoheadleftarrow \ZZ/p^m \ZZ$ for $n < m$. 
    We also have surjections $\phi_n: \ZZ \twoheadrightarrow \ZZ/p^n \ZZ$ defined in the usual way, with
    $\phi_n = f_m^n \circ \phi_m$. Then by the universal 
    property of the inverse limit, there is a unique homomorphism $u: \ZZ \rightarrow \hat{\ZZ}_p$ and we have 
    a commutative diagram 
    \[
        \begin{tikzcd}
                         & \mathbb{Z} \arrow[d, "u"] \arrow[ldd, "\phi_n"', bend right] \arrow[rdd, "\phi_m", bend left] &                                              \\
                         & \hat{\mathbb{Z}}_p \arrow[ld, "\eta_n"'] \arrow[rd, "\eta_m"]                                 &                                              \\
\mathbb{Z}/p^n\mathbb{Z} &                                                                                               & \mathbb{Z}/p^m\mathbb{Z} \arrow[ll, "f_m^n"]
\end{tikzcd},
    \]
    where $\eta_n$ are the maps obtained from the inverse limit construction.

    Now take $a \in \ZZ$ and suppose $u(a) = 0 \in \hat{\ZZ}_p$. Then for all $n$, $\eta_n(u(a)) = 0$ since $\eta_n$
    is a homomorphism, and thus $\phi_n(a) = 0$ by the commutativity of the diagram. Then $a$ must equal zero, 
    otherwise it would have to be divisible by every power of $p$. Thus $u$ is an injective map. 

    Similarly for the next case, we have maps $\phi_m: \ZZ \rightarrow \ZZ/m\ZZ$ for every integer $m$ which 
    are compatible in the above manner with the surjections $\ZZ/nm\ZZ \twoheadrightarrow \ZZ/m\ZZ$, and thus 
    there is a unique map $v: \ZZ \rightarrow \hat{\ZZ}$ with a commuting diagram similar to the 
    one given above. 

    To show $v$ is injective is the same. Take $a \in \ZZ$ and suppose $v(a) = 0 \in \hat{\ZZ}$. Then by commutativity,
    we have $\phi_m(a) = 0$ for all $m$, and $a$ must be zero, or else it would need to be divisible by every positive integer.

    Finally, we show existence of an injection $\hat{\ZZ}_p \rightarrow \hat{\ZZ}$. Note that for any $m \in \mathbb{P}$, 
    we can write $m = p^a n$ where $n$ and $p^a$ are relatively prime. Then $\ZZ/m\ZZ$ is isomorphic to the direct 
    sum $\ZZ/p^a \ZZ \oplus \ZZ/n\ZZ$. We already have maps $\eta_a: \hat{\ZZ}_p \rightarrow \ZZ/p^a\ZZ$ for every $a$ by the 
    inverse limit construction. Then define a map $\phi_m: \hat{\ZZ}_p \rightarrow \ZZ/m\ZZ$ by $\phi_m(x) = (\eta_a(x), 0)$,
    where $a$ is determined by the given $m$. Moreover, if $n$ divides $m$, $n$ must contain a lesser power of $p$, 
    and the map $\ZZ/m\ZZ \rightarrow \ZZ/n\ZZ$ is compatible with the maps $\phi_i$ by mapping their $p$ power component 
    using the previously mentioned map, and the rest by taking some quotient. Then the universal property 
    gives us a map $k: \hat{\ZZ}_p \rightarrow \hat{\ZZ}$. 

    Let $s \in \hat{\ZZ}_p$ and suppose $k(s) = 0$. Then in particular, $\phi_{p^a}(s) = 0$ for every $a$, and 
    by considering $\hat{\ZZ}_p$ as a sequence, every coordinate of $a$ must be zero, and thus $a = 0$. 

    \item[2.] Let $A, B, C, D$ be abelian groups with homomorphisms 
    \[
        f: A \rightarrow B, g: A \rightarrow C, h: B \rightarrow D, k: C \rightarrow D
    \]
    (a.) Lang defines the fiber product $B \times_D C$ of abelian groups as the subset of elements $(b,c)$ of $B \times C$ such that 
    $h(b) = k(c)$. Now suppose we have a commuting square 
    \[
        \begin{tikzcd}
                  & G \arrow[ld, "\gamma"'] \arrow[rd, "\rho"] &                    \\
B \arrow[rd, "h"] &                                            & C \arrow[ld, "k"'] \\
                  & D                                          &                   
\end{tikzcd}
    \]
    for some other abelian group $G$ and homomorphisms $\gamma$ and $\rho$. Define a map $u: G \rightarrow B\times_D C$
    by $u(g) = (\gamma(g), \rho(g))$. We know $(\gamma(g), \rho(g))$ is an element of the fiber product since
    $h(\gamma(g)) = k(\rho(g))$ from the commutative diagram. 

    Let $p_1$ and $p_2$ be the projections from $B\times_D C$ into $B$ and $C$ respectively. Then we indeed 
    have a commutative diagram
    \[
        \begin{tikzcd}
             & G \arrow[ld, "\gamma"'] \arrow[d, "u"] \arrow[rd, "\rho"] & \\
            B &\arrow[l, "p_1"] B \times_D C \arrow[r, "p_2"'] & C 
        \end{tikzcd},
    \]
    since $p_1(u(g)) = p_1(\gamma(g), \rho(g)) = \gamma(g)$, and the same for the other side. 

    Finally we must show $u$ is unique. Suppose we have some other map $k: G \rightarrow B\times_D C$ 
    such that a similar diagram commutes. Then the first component of $k$ must be $\gamma(g)$ and the second
    must be $\rho(g)$. Thus $k = u$, and $u$ is the unique map which gives a commutative diagram. 

    (b.) Suppose the map $k$ is surjective. Then we will show $p_1$ is surjective. Take any $b \in B$. 
    Then by the surjectivity of $g$, there is an element $c \in C$ such that $k(c) = h(b)$. Then 
    $(b,c)$ is an element of $B\times_D C$, and we have $p_1(b,c) = b$. Thus $p_1$ is surjective. 

    (c.) Lang defines the fibered coproduct as the quotient of $B \oplus C$ by the subgroup $W$ of elements 
    of the form $(f(a), -g(a))$. Let $q_1: B \rightarrow B \oplus C/W$ be the map $b \mapsto (b,0) + W$,
    and similarly for $q_2: C \rightarrow B\oplus C/W$. This gives us a commutative diagram
    \[
        \begin{tikzcd}
                  & A \arrow[ld, "f"'] \arrow[rd, "g"] &                    \\
B \arrow[rd, "q_1"] &                                            & C \arrow[ld, "q_2"'] \\
                  & B\oplus C/W                                          &                   
\end{tikzcd},
    \]
    since $q_1(f(a)) = f(a) + W = g(a) + W$ by a change of variables within the definition of $W$.

    Suppose we have another abelian group $T$ with maps $\gamma: B \rightarrow T$
    and $\nu: C \rightarrow T$ such that $\gamma\circ f = \nu \circ g$.

    We can define a map $u: B \oplus C \rightarrow T$ by $u(b,c) = \gamma(b) + \nu(c)$. Then we have 
    $W \subset \ker(u)$, since $u(f(z), -g(z)) = \gamma(f(z)) - \nu(g(z)) = 0$. Thus 
    there is an induced map $\overline{u}: B \oplus C/W \rightarrow T$ by the universal property of quotients,
    such that $\overline{u}((b,c) + W) = u(b, c)$. Then we have $\gamma = \overline{u}\circ q_1$, since
    $\overline{u}\circ q_1(b) = \overline{u}((b,0) + W) = u(b,0) = \gamma(b)$, and the same for $\nu$ as desired. 

    Finally we need to show $\overline{u}$ is unique. Suppose there was another such map $\overline{k}: B \oplus C/W \rightarrow T$ so that
    its diagram commutes. Then there is an induced map $k: B \oplus C \rightarrow T$ such that $k(W) = 0$. Then by 
    the universal property of direct sums, we have $\gamma = k \circ p_1$ where $p_1$ is the canonical injection
    of $B \rightarrow B \oplus C$. Then $k \circ p_1(b) = k(b, 0) = \overline{k}(b,0) = \gamma(b)$,
    and the same in the second coordinate. Thus the components of $k$ are determined by $\gamma$ and $\nu$, and
    thus so is $\overline{k}$. Then $u$ is unique.

    (d.) Now suppose $g$ is injective. We will show $q_1$ is injective. Suppose $q_1(b) = 0$. Then by the definition
    of $q_1$, we have $(b, 0) \in W$ and $b = f(a)$ for some $a \in A$ and $g(a) = 0$. Then $a = 0$ 
    since $g$ is injective, and thus $f(a) = 0 = b$. 

    (e.) The two constructions have nothing in common since one incorporates arrows into $B$ and $C$ from another
    group, and the other has arrows out of $B$ and $C$ to an unrelated group. 
    
    \item[3.] (a.) Let $A$ be an object of a locally small category $\mathcal{C}$. Then we are given that 
    $\text{Mor}(A, A)$ contains an identity morphism, and that the arrows satisfy the associative property. 
    This is exactly the definition of a monoid. 

    (b.) $text{Aut}(A)$ contains a monoid by the argument above, and also has the property that for every $g \in \text{Aut}(A)$,
    there is an inverse $f$ such that $g \circ f = f \circ g = id_A$. Then $\text{Aut}(A)$ is a group. 

    \item[4.] Let $G$ be a group and let $\mathcal{C}$ be the category of $G$-sets with maps between $G$-sets
    compatible with the group action. We will verify that this forms a category.

    We clearly have a collection of objects and morphisms. Let $S, T, U$ be $G$-sets. We will check for a composition
    map $\text{Mor}(T, U) \times \text{Mor}(S, T) \rightarrow \text{Mor}(S, U)$. Let 
    $g \in \text{Mor}(S, T)$ and $h \in \text{Mor}(T, U)$. Then $h \circ g$ is a well defined map $S \rightarrow U$.
    Moreover, for any $k \in G$ we have
    \[
        h \circ g(k \cdot s) = h(k\cdot g(s)) = k\cdot(h\circ g(s))
    \]
    so that $h \circ g \in \text{Mor}(S, U)$. 

    This map satisfies the associative property since function composition is associative. Moreover, the
    morphism sets are well defined, i.e. disjoint, since we know this to be true in the category of sets, and
    morphisms in this category are the subset of these maps compatible with the group action. 

    \item[5.] Let $\mathcal{G}$ be the category of groups, and $\mathcal{A}b$ the category of abelian groups.
    
    (a.) Let $\phi: G \rightarrow H$ be a group homomorphism. This induces a map $\overline{\phi}: G \rightarrow \text{ab}(H)$ 
    by $g \mapsto \phi(g)[H,H]$. Then since we have a map $G$ into an abelian group, this factors 
    through the abelianization of $G$, and hence we have a map $\text{ab}(\phi): \text{ab}(G) \rightarrow \text{ab}(H)$
    such that $\text{ab}(\phi)(g[G, G]) = \phi(g)[H, H]$. 

    (b.) Now we show that $\text{ab}$ is a functor. We just need to show it respects composition of morphisms and 
    sends identity to identity. 

    But $\text{ab}(\text{id}_G)(g[G,G]) = g[G,G]$ by the definition of the abelianized maps above, and 
    thus $\text{ab}(\text{id}_G) = \text{id}_{\text{ab}(G)}$.

    For composition, let $f: G \rightarrow H$ and $g: H \rightarrow K$. Then 
    \[
        \text{ab}(g) \circ \text{ab}(f)(a[G, G]) = \text{ab}(g)(f(a)[H, H]) = (g\circ f)(a)[K, K] = \text{ab}(g \circ f)(a[G, G])
    \]
    as desired. Then $\text{ab}$ is a functor from groups to abelian groups. 

    (c.) We have that the abelianization functor is adjoint to the forgetful functor $i$ on abelian groups, i.e.
    for a group $H$ and an abelian group $G$, there is a natural bijection
    \[
        \text{Hom}_{\text{ab gp}}(\text{ab}(H), G) \simeq \text{Hom}_{\text{gp}}(H, i(G)).
    \]
    For any map $f: \text{ab}(H) \rightarrow G$, we can compose $f$ with the projection onto the abelianization of 
    $H$ to get a unique map $f': H \rightarrow G$. Also for any $g: H \rightarrow i(G)$, there is a 
    unique map $\text{ab}(g): \text{ab}(H) \rightarrow G$. This is the bijective correspondence. 

    \item[6.] We will show that every nonidentity element of a free group has infinite order. Each nonidentity
    element is of the form 
    \[
        s = s_1^{a_1} \cdots s_n^{a_n}
    \]
    where $F$ is free on $S$ and $s_i \in S$, and $a_i \in \mathbb{Z}$. If $n$ is odd, multiplying $s$ by itself
    could only cancel at most the terms to the left of the middle term with the terms to the right of 
    the middle element, producing a new element of the form
    \[
        s_1^{a_1} \cdots s_m^{2a_n} s_{m+1}^{a_{m+1}} \cdots s_n^{a_n}
    \]
    where $s_m$ is the middle term. If not all elements cancel between the middle terms, the situation is worse. 
    Then clearly repeated multiplication of $s$ with itself will never yeild the identity, as the middle terms cannot cancel.

    The case for even word length is similar, as there will be two middle terms which do not cancel, and even 
    if there is some cancellation, we will always be left with products of these middle two terms which cannot be 
    reduced. 

    \item[7.] Let $F$ be a free gorup and define $N$ to be the subgroup of all $n$-th power elements for some $n \in \mathbb{P}$.
    This is a normal subgroup: For any $g \in F$, we have $g x^n g^{-1} = \left(gxg^{-1}\right)^n$, as 
    $gg^{-1} = e$. Then $gNg \in N$, and likewise $g^{-1}Ng \subset N$, so we are done. 

    \item[8.] Let $G * H$ be the free product of $G$ with $H$. We have maps $p_1: G \rightarrow G*H$ 
    which send $g \in G$ to the element $g$ of the free product, and similarly $p_2: H \rightarrow G*H$. 
    Elements of the free product are of the form $g_1^{a_1}h_1^{b_1}\cdots g_n^{a_n}h_n^{b_n}$
    for $g_i \in G$, $h_i \in H$, and $a_i, b_i \in \ZZ$. This is after reducing words in the free product
    until they are formed by alternating elements of $G$ and $H$. 

    Suppose $i: G \rightarrow A$ and $j: H \rightarrow A$ are two homomorphisms into another group $A$. Then
    define a map $\phi: G * H \rightarrow A$ by 
    \[
        \phi(g_1^{a_1}h_1^{b_1}\cdots g_n^{a_n}h_n^{b_n}) = i(g_1^{a_1})j(h_1^{b_1})\cdots i(g_n^{a_n})j(h_n^{b_n}).
    \]
    We have a commutative diagram
    \[
        \begin{tikzcd}
            G \arrow[r, "p_1"] \arrow[rd, "i"] & G * H \arrow[d, "\phi"] & H \arrow[dl, "j"] \arrow[l, "p_2"]\\
            & A
        \end{tikzcd}
    \]
    since $\phi\circ p_1(g) = i(g)$ and similar for the other side.

    Moreover, suppose we have some other homomorphism $\psi: G * H$ which makes the diagram commute. 
    This means $\psi(g) = i(g)$ for $g \in G$ and $\psi(h) = j(h)$ for $h \in H$. Also, we have
    \[
        \psi(g_1^{a_1}h_1^{b_1}\cdots g_n^{a_n}h_n^{b_n}) = \psi(g_1^{a_1})\psi(h_1^{b_1})\cdots \psi(g_n^{a_n})\psi(h_n^{b_n}) = i(g_1^{a_1})j(h_1^{b_1})\cdots i(g_n^{a_n})j(h_n^{b_n})
    \]
    for any element of $F$, and thus $\psi$ is equal to the map $\phi$ and $\phi$ is unique.



\end{enumerate}

\end{document}