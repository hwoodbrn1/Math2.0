\documentclass[11pt, reqno]{article}

\usepackage{amsmath, amsthm, amssymb}
\usepackage{enumitem}
\usepackage{tcolorbox}
\usepackage{hyperref}
\usepackage{tikz}
\usepackage{tikz-cd}
\usetikzlibrary{arrows.meta}
\usepackage{mathrsfs}
\usepackage{fancyhdr}
\usepackage[bottom=0.75in, top=1in, left=0.5in, right=0.5in]{geometry}
\usepackage{array}   % for \newcolumntype macro
\newcolumntype{L}{>{$}l<{$}}

\theoremstyle{plain}
\newtheorem*{theorem}{Theorem}
\newtheorem*{proposition}{Proposition}
\newtheorem{exercise}{Exercise}
\newtheorem*{lemma}{Lemma}
\newtheorem*{corollary}{Corollary}

\theoremstyle{definition}
\newtheorem*{definition}{Definition}
\newtheorem*{example}{Example}

\theoremstyle{remark}
\newtheorem*{remark}{Remark}

\renewcommand{\phi}{\varphi}
\renewcommand{\epsilon}{\varepsilon}
\renewcommand{\emptyset}{\varnothing}

\newcommand{\RR}{\mathbb{R}}
\newcommand{\ZZ}{\mathbb{Z}}
\newcommand{\NN}{\mathbb{N}}
\newcommand{\CC}{\mathbb{C}}
\newcommand{\QQ}{\mathbb{Q}}

\DeclareMathOperator{\ima}{\text{im}}

\begin{document}

\topmargin=-40pt
\rhead{Henry Woodburn}
\lhead{Math 653}
\renewcommand{\headrulewidth}{1pt}
\renewcommand{\headsep}{20pt}
\thispagestyle{fancy}

{\Huge \bfseries \noindent Homework 11}

\begin{enumerate}
    \item[1.] Define the binomial coefficient $\binom{a}{b}$ to be 
    \[
        \frac{n!}{k!(n-k)!}
    \]
    for integers $0 \leq k \leq n$. Let $R$ be a commutative ring. We will prove the binomial theorem:
    \[
        \forall a, b \in R\ \forall n \in \NN, \qquad (a + b)^n = \sum_0^k \binom{n}{k} a^k b^{n-k}
    \]
    We will prove by induction. First, clearly the formula holds for $n = 1$. Then assume the fomula
    holds for $(a + b)^n$, and we will show it is valid for $(a + b)^{n+1}$ as well. 

    We have
    \begin{align*}
        (a + b)^{n+1} = (a + b)^n(a + b)& = \left(\sum_0^n \binom{n}{k}a^k b^{n-k}\right)(a + b)\\
        & = \sum_0^n \binom{n}{k} a^{k+1}b^{n-k} + \sum_0^n \binom{n}{k}a^k b^{n-k+1}\\
        & = \sum_1^{n+1}\binom{n}{k-1}a^k b^{n-k+1} + \sum_0^n \binom{n}{k}a^k b^{n-k+1}\\
        & = b^{n+1} + \sum_1^n\left(\binom{n}{k-1} + \binom{n}{k}\right)a^k b^{n+1-k} + a^{n+1}\\
        & = b^{n+1} + \sum_1^n \binom{n+1}{k}a^k b^{n+1-k}+ a^{n+1}\\
        & = \sum_0^{n+1}\binom{n+1}{k}a^k b^{(n+1)-k}.
    \end{align*}

    as desired. The fact that $\binom{n}{k-1} + \binom{n}{k} = \binom{n+1}{k}$ comes from the following calculation:
    \begin{align*}
        \binom{n}{k-1} + \binom{n}{k} =\ &\frac{n!}{(k-1)!(n-k+1)!} + \frac{n!}{k!(n-k)!}\\
        =\ &\frac{n!k}{k!(n-k+1)!} + \frac{n!(n-k+1)}{k!(n-k+1)!}\\
        =\ &\frac{n!k+n\cdot n! - n!k + n!}{k!(n-k+1)!}\\
        =\ &\frac{n!(n+1)}{k!(n-k+1)!} = \frac{(n+1)!}{k!((n+1) - k)!} = \binom{n+1}{k}
    \end{align*}

    \item[2.] Let $R$ be a commutative ring of characteristic $p$ for $p$ prime. Consider the map 
    $a \mapsto a^p$. We will prove this is a ring homomorphism. 

    Since $R$ is commutative, we have $(ab)^p = a^p b^p$.

    For addition, note that when $k$ is not zero or $p$, the binomial coefficient $\binom{p}{k}$ is divisible
    by $p$. Then since $R$ is characteristic $p$, we have
    \[
        (a + b)^p = \sum_{k=0}^p \binom{p}{k} a^k b^{n-k} = a^p + b^p
    \]
    verifying that the map is a ring homomorphism.

    \item[3.] Let $R$ be a commutative ring and let $Z$ be the set consisting of all zero divisors and zero. 
    
    The set of ideals contained in $Z$ is ordered by set containment. Any linearly ordered subset has
    an upper bound, the union of the subset. Then by Zorn's lemma, there is an ideal $M$ which is maximal
    with respect to containment amongst the set of all ideals contained in $Z$. We must show $M$ is prime.

    Suppose not. Then there exist $a, b \in R$ such that $ab \in M$ but neither $a$ or $b$ is contained in $M$.
    Then the ideal $M + (a)$ is strictly larger than $M$, and the same for $M + (b)$. By maximality of $M$, there
    must exist elements $x \in M + (a)$ and $y \in M + (b)$ such that $x,y \notin Z$. Then we can write
    \[
        x = m_1 + r_1 a, \quad y = m_2 + r_2 b,
    \]
    and their product
    \[
        xy = m_1 m_2 + r_1 a m_2 + r_2 b m_1 + r_1 r_2 ab.
    \]

    Then $xy \in M$ since $m_1, m_2, ab$ are each in $M$ and $M$ is an ideal. But then $xy$ is a zero 
    divisor, implying either $x$ or $y$ must be a zero divisor: if $c(ab) = 0$, then either $ca$ is zero 
    and $a$ is a zero divisor, or $ca$ is nonzero and $b$ is a zero divisor. This is a contradiction. Thus 
    $M$ is a prime ideal contained in $Z$.

    \item[4.] Let $G$ be a finite group. We will show the center $Z(\CC[G])$ of the group 
    algebra $\CC[G]$ has dimension equal to the number of conjugacy classes in $G$. 

    To do this, we will give an explicit basis. For $g \in G$, define
    \[
        e_{[g]} = \sum_{h \in G}hgh^{-1}.
    \]

    Note that for all $h \in [g]$, $e_{[h]} = e_{[g]}$. Then the number of elements $e_{[g]}$ is equal to 
    the number of conjugacy classes in $G$. Let $N \subset G$ be a set containing one representative
    from each conjugacy class.

    We first show $e_{[g]} \in Z(\CC[G])$ for all $g \in G$. Fix $g \in G$ and take any $a \in \CC[G]$, with
    \[
        a = \sum_{g \in G} a_g\cdot g
    \]

    for some $a_g \in \CC$. We will show these elements commute. We have
    \[
        e_{[g]}a = \sum_{h,k \in G} a_k\cdot hgh^{-1}k = \sum_{k \in G}\left(\sum_{h \in G} a_k hgh^{-1}k\right)
        = \sum_{k \in G}\left(\sum_{h \in G} a_k khgh^{-1}k^{-1}k\right) = \sum_{h,k \in G} a_k khgh^{-1},
    \]
    
    since the map $h \mapsto kh$ is a bijection of $G$, allowing us to interchange $h$ with $kh$ for a fixed $k$.

    Since the conjugacy classes of $G$ are disjoint, if we have
    \[
        \sum_{g \in N} a_g e_{[g]} = 0,
    \]
    it must be that $a_g = 0$ for all $g \in N$. Then $\{e_{[g]}: g \in N\}$ is a linearly independent set. 

    Finally we will show these elements span $Z(\CC[G])$. 
    Suppose there is an element $f$ which cannot be written as a linear combination
    of the $e_{[g]}$'s. Then $f$ is not contstant on the conjugacy classes of $G$. Writing
    \[
        f = \sum_{g \in G} f(g) \cdot g,
    \]
    this means there exist $h,g \in G$ such that $f_(g) \neq f(hgh^{-1})$. Then consider the element
    $h = 1\cdot h \in \CC[G]$. We have
    \[
        fh(hg) = f(hgh^{-1}),
    \]
    but 
    \[
        hf(hg) = f(g).
    \]

    Then $fh \neq hf$, meaning $f \notin Z(\CC[G])$. By the contrapositive, every element in the center 
    $Z(\CC[G])$ is constant on the conjugacy classes of $G$. Then $Z(\CC[G])$ is spanned by 
    $\{e_{[g]}: g \in N\}$ and its dimension is equal to the size of this set, the number 
    of conjugacy classes in $G$. 

    \item[5.] For an arbitrary ring $R$ and an infinite cyclic group $G$ with generator $\xi$, the group ring
    $R[G]$ is not necessarily isomorphic to the polynomial ring in one variable $R[x]$. 
    
    To see this, consider $R = \ZZ$. We will show the group of units of $\ZZ[G]$ is not isomorphic
    to the group of units of $\ZZ[x]$. The units of $\ZZ[x]$ are only the constant polynomials $1$ and 
    $-1$, as multiplication of any two polynomials in $\ZZ[x]$ results in a polynomial of equal or higher
    degree. In particular, you will never get $1$. However, the units of $\ZZ[G]$ include the constant 
    polynomials $1$ and $-1$, as well as the elements $\xi^n$ for $n \in \ZZ$. Then the group of 
    units of $\ZZ[G]$ is infinite, and thus cannot be isomorphic to the two element group of units
    of $\ZZ[x]$. Then the two rings cannot be isomorphic. 


\end{enumerate}

\end{document}