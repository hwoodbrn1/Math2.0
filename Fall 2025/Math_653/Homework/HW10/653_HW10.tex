\documentclass[11pt, reqno]{article}

\usepackage{amsmath, amsthm, amssymb}
\usepackage{enumitem}
\usepackage{tcolorbox}
\usepackage{hyperref}
\usepackage{tikz}
\usepackage{tikz-cd}
\usetikzlibrary{arrows.meta}
\usepackage{mathrsfs}
\usepackage{fancyhdr}
\usepackage[bottom=0.75in, top=1in, left=0.5in, right=0.5in]{geometry}
\usepackage{array}   % for \newcolumntype macro
\newcolumntype{L}{>{$}l<{$}}

\theoremstyle{plain}
\newtheorem*{theorem}{Theorem}
\newtheorem*{proposition}{Proposition}
\newtheorem{exercise}{Exercise}
\newtheorem*{lemma}{Lemma}
\newtheorem*{corollary}{Corollary}

\theoremstyle{definition}
\newtheorem*{definition}{Definition}
\newtheorem*{example}{Example}

\theoremstyle{remark}
\newtheorem*{remark}{Remark}

\renewcommand{\phi}{\varphi}
\renewcommand{\epsilon}{\varepsilon}
\renewcommand{\emptyset}{\varnothing}

\newcommand{\RR}{\mathbb{R}}
\newcommand{\ZZ}{\mathbb{Z}}
\newcommand{\NN}{\mathbb{N}}
\newcommand{\CC}{\mathbb{C}}
\newcommand{\QQ}{\mathbb{Q}}

\DeclareMathOperator{\ima}{\text{im}}

\begin{document}

\topmargin=-40pt
\rhead{Henry Woodburn}
\lhead{Math 653}
\renewcommand{\headrulewidth}{1pt}
\renewcommand{\headsep}{20pt}
\thispagestyle{fancy}

{\Huge \bfseries \noindent Homework 10}

\begin{enumerate}
    \item[1.] For every positive integer $n$ let $R_n$ be a ring and suppose for $0 < m < n$ there exist
    ring homomorphisms $\phi_{n,m}: R_n \rightarrow R_m$ such that if $0 < l < m < n$, we have 
    $\phi_{n,l} = \phi_{m,l} \circ \phi_{n,m}$. This defines an inverse system. 

    We have previously shown that the inverse limit $\varinjlim R_n$ of the groups $R_n$ exists. Now we
    will show that this inverse limit of rings has a ring structure. 

    The inverse limit of groups is defined to be the subset $\Gamma \subset \prod_{1}^\infty R_n$ consisting
    of elements $(x_i)$ such that for $i < j$, we have $\phi_{j, i}(x_j) = x_i$.  We define the ring operation
    component-wise. This is well defined since products exist in the category of rings (to be shown), so 
    we just need to show that it takes values in $\Gamma$. Let $(a_i), (b_i) \in \Gamma$ and $0 < i < j$. Then
    \[
        \phi_{j,i}(a_j b_j) = \phi_{j,i}(a_j) \phi_{j,i}(b_j) = a_i b_i
    \]
    so indeed $(a_i b_i) \in \Gamma$. 

    \item[2.] We will define the product of rings $G$ and $H$ as the cartesian product $G \times H$ equipped 
    with component-wise addition and multiplication. Let $p_1$ and $p_2$ be the projections onto $G$ and $H$ 
    respectively. 

    Suppose $K$ is another ring with homomorphisms $f: K \rightarrow G$ and $g: K \rightarrow H$. 
    We must show there is a map $u$ such that the following diagram is commutative: 
    \[
        \begin{tikzcd}
             & K \arrow[dl, "f"'] \arrow[d, "u"] \arrow[dr, "g"]\\
            G & \arrow[l, "p_1"'] G\times H \arrow[r, "p_2"] & H
        \end{tikzcd}
    \]

    If we did have such a map, for $k \in K$ we would have $u(k) = (a,b)$, and $p_1\circ u(k) = a = f(k)$,
    and likewise $b = g(k)$. Then $u(k) = (f(k), g(k))$ for all such maps. 

    Indeed, the map $u: k \mapsto (f(k), g(k))$ makes the diagram commute, and is the only such map
    by the above argument. Then $G\times H$ is a direct product of rings.

    \item[3.] Let $\eta(R)$ be the set of nilpotent elements in a commutative ring $R$. First we show that $\eta(R)$ is an 
    ideal. Let $a \in R$ and $b \in \eta(R)$ with $b^n = 0$ for some $n$. Then $(ab)^n = a^n b^n = a^n \cdot 0 = 0$.

    Next we show that $\eta\left(R/\eta(R)\right) = \{0\}$. Let $a + \eta(R) \in R/\eta(R)$ and suppose $(a + \eta(R))^n = 0$
    for some $n$. Then by the definition of the quotient ring, we have $a^n + \eta(R) = 0$ and thus $a^n \in \eta(R)$.
    Then there is some $m$ such that $(a^n)^m = 0$, and thus $a^{nm} = 0$ and $a \in \eta(R)$. Then 
    $a + \eta(R) = \eta(R) = 0 \in R/\eta(R)$. So the only nilpotent element of $R/\eta(R)$ is $0$. 

    \item[4.] We will prove that $\text{End}(\ZZ \oplus \ZZ)$, the ring of endomorphisms of $\ZZ\oplus\ZZ$, is 
    noncommutative. 

    Note that $\ZZ\oplus\ZZ$ is free, so any map of its generators $(1,0)$ and $(0,1)$ into a group $G$ extends 
    to a homomorphism $\ZZ \oplus \ZZ \rightarrow G$. 

    Consider the following homomorphisms $\ZZ \oplus \ZZ \rightarrow \ZZ \oplus \ZZ$, defined by their action on generators:
    \begin{align*}
        \phi: 
        \begin{split}
            (1,0) \mapsto (1,1)\\
            (0,1) \mapsto (1,1)\\
        \end{split}\\
         \\
        \psi:
        \begin{split}
            (1,0) \mapsto (1,0)\\
            (0,1) \mapsto (1,1)\\
        \end{split}.
    \end{align*}

    Then $\phi(\psi(0,1)) = \phi(1,1) = (2,2)$, but $\psi(\phi(0,1)) = \psi(1,1) = (2,1)$, and thus $\phi\circ\psi \neq 
    \psi \circ \phi$. Then the endomorphism ring of $\ZZ\oplus\ZZ$ is noncommutative. 

    \item[5.] Let $R$ be a ring and $I \subset R$ and ideal. 
    (1.) $M_n(I)$ is an ideal: Let $A \in M_n(R)$ and $B \in M_n(I)$. Then the entries of $AB$ are each
    sums of elements in $I$ multiplied on the left by elements of $R$. Since $I$ is an ideal and hence a left ideal, 
    each entry of $AB$ is in $I$, so $AB \in M_n(I)$. Then $M_n(I)$ is a left ideal. A similar proof shows
    $BA \in M_n(I)$, so that $M_n(I)$ is both a left and right ideal. 

    (2.) Let $I$ be an ideal in $M_n(R)$.

    Denote by $I_{i,j}$ the set of all values in the $(i,j)$ coordinate of 
    matrices in $I$. For any $a \in I_{i,j}$ and any $r \in R$, we can obtain a matrix in $I$ with $rar$ in 
    the $(i,j)$ coordinate and zeros otherwise by multiplying the matrix with only $a$ at position $(i,j)$ on the 
    left and right by the matrices with all entries $r$. Then $rar \in I_{i,j}$, so $I_{i,j}$ is an ideal for 
    any $0 \leq i,j \leq n$. 
    
    Note that for any $A \in I$,
    by multiplying $A$ on the right by the matrix with all zeros except $1$ 
    in the $(j,j)$ coordinate we can reduce every column of $A$ to zero except the $j$th one. 
    by multiplying $A$ on the left by a matrix with all zeros except a $1$ in the $(i,i)$ coordinate, we obtain 
    a matrix with only the $i$th row of $A$. This allows us to obtain a new matrix with only the $(i,j)$th position equal to 
    that of $A$, and all other entries zero. Note also that this new matrix is in $I$, since $I$ is a left and right ideal
    within $M_n(R)$. 

    Also note that for any matrix $A$ in $M_n(R)$, we can obtain a new matrix equal to $A$ with any two of its rows or columns swapped
    by multiplying by a certain matrix on the left or right. $I$ is closed under this operation since it is an ideal. 
    Then for any $0 \leq h,i,j,k \leq n$, we can produce a matrix with any of the values of $I_{h,i}$ appearing 
    in the $(j,k)$th coordinate. Hence, $I_{h,i} \subset I_{j,k}$, and likewise $I_{j,k} \subset I_{h,i}$. 
    Then it follows that the $I_{i,j}$'s are equal for any choice of $0 \leq i,j \leq n$. Call this set $I'$. 

    This proves that $I \subset M_n(I')$. We showed that $I$ contains all the matrices with only one coordinate nonzero
    and with this entry equal to any element of $I'$. Then we can add these to obtain any element of $M_n(I')$. 
    $I$ is closed under addition, and hence $M_n(I') \subset I$ and we are done.

    \item[6.] We prove $R$ is a division ring if and only if it has no proper left ideals. 
    
    First suppose $R$ is a division ring. Then every element of $R$ is a unit. Thus any nonzero ideal must contain a unit
    and is thus equal to $R$ itself. 

    Conversely suppose $R$ has no proper ideals. Then for any nonzero $a \in R$, the left ideal $\langle a \rangle$ 
    must equal $R$. Then $\langle a \rangle$ contains $1$, so there is some $b \in R$ such that $ba = 1$. This proves every 
    nonzero element of $R$ has a left inverse. 

    This in fact proves that every nonzero element has a right inverse as well. Again let $a,b \in R$ such that $ba = 1$.
    Take some $c \in R$ such that $cb = 1$. Then 
    \[
        ab = cbab = cb = 1,
    \]
    and we see $b$ is the right inverse of $a$ as well. Then $R$ is a division ring. 

    \item[7.] Let $m$ be a positive integer and consider the ring $\ZZ_m$ of integers modulo $m$. Note
    that this is a commutative ring. 
    
    \textbf{Proposition:} $\ZZ_m$ is an integral domain if and only if $m$ is prime. In particular, $\ZZ_p$ is a 
    field for $p$ prime. 

    \textit{Proof:} If $m$ is not prime, say $m = nl$, then we have $nl = 0 \mod m$ and thus $n$ and $l$ 
    are zero divisors. 

    If $m$ is prime, suppose there are $a,b \in \ZZ_m$ such that $ab = 0 \mod m$. Then $ab | m$ and $m$ must 
    divide either $a$ or $b$, hence one of them must be $0$ in $\ZZ_m$. 

    If $p$ is prime, then by Fermat's Little Theorem, for any nonzero $a \in \ZZ_p$ we have
    \[
        a^{p-1} = 1 \mod p
    \]
    and we see that $a^{p-2}$ is the multiplicative inverse of $a$ in $\ZZ_p$. 

    \textbf{Proposition:} If $R$ is a commutative ring and $M \subset R$ is an ideal, then 
    $M$ is maximal if and only if $R/M$ is a field. 

    \textit{Proof:} The direction $R/M$ is a field if $M$ is maximal was proved in class.

    Conversely suppose $R/M$ is a field. Suppose $N \subset R$ is another ideal, not necessarily proper,
    and $M \subsetneq N$. Take $a \in N\setminus M$ so that $\overline{a} \in R/M$ is not equal to zero.
    Then $\overline{a}$ is a unit since $R/M$ is a field, so there is some $\overline{b} = b + M$, $b \in R$,
    such that $\overline{a}\overline{b} = 1$. Then $ab + m = 1$ for some $m \in M$. But $M \subset N$, 
    so $1 \in N$ and hence $N = R$. Then $M$ is maximal.

    It was proven in class that $R/I$ is an integral domain if and only if $I$ is a prime ideal. 

    Finally, we know every ideal in $\ZZ_m$ is principal. 

    If $m$ is prime, $\ZZ_m$ has no proper ideals. 

    Otherwise, the maximal ideals and prime ideals coincide except for the ideal $\langle 0\rangle$ which is prime.
    Every nonzero maximal or prime ideal is generated by a prime number, and every prime number generates 
    an ideal which is both prime and maximal. 

    To see this, suppose $\langle a \rangle$ is prime or maximal. Then $\ZZ_m/\langle a \rangle$ is a field, and all quotients of 
    $\ZZ_m$ are isomorphic to $\ZZ_n$ for some $n$. Then we must have $\ZZ_m/\langle a \rangle \simeq \ZZ_p$ for some $p$,
    and thus $a = p$. 

    Moreover for any prime $p < m$, we have $\ZZ_m/\langle p \rangle = \ZZ_p$ which is a field. Then 
    $\langle p \rangle$ is prime and maximal. 

    \item[8.] Let $S$ be a subset of a ring $R$. We will show the intersection of all ideals containing $S$ is the set
    \[
        N = \left\{ \sum_1^n r_i s_i t_i: r_1, t_1, \dots, r_n, t_n \in R, s_1, \dots, s_n \in S, n \in \NN\right\}.
    \]

    First suppose 
    \[
        x \in \bigcap \{I: I\ \text{an ideal with}\ S \subset I\}.
    \]

    Note that $N$ is an ideal since multiplication satisfies the 
    distributive property and because $N$ is closed under addition, and that $S$ is contained in $N$. Then 
    $x \in N$. 

    Conversely, suppose $x \in N$ with
    \[
        x = \sum_1^n r_i s_i t_i,
    \]
    for $r_i, t_i$ elements of $R$ and $s_i$ elements of $S$. Any ideal $I$ containing $S$
    must contain every $s_i$ and thus contains $x$. Then $x$ is contained in the intersection 
    of all ideals containing $S$. 

    
\end{enumerate}

\end{document}