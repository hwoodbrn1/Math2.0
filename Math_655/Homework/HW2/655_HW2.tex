\documentclass[11pt, reqno]{article}

\usepackage{amsmath, amsthm, amssymb}
\usepackage{enumitem}
\usepackage{tcolorbox}
\usepackage{hyperref}
\usepackage{tikz}
\usetikzlibrary{arrows.meta}
\usepackage{mathrsfs}
\usepackage{fancyhdr}
\usepackage[bottom=0.75in, top=1in, left=0.5in, right=0.5in]{geometry}
\usepackage{array}   % for \newcolumntype macro
\newcolumntype{L}{>{$}l<{$}}

\theoremstyle{plain}
\newtheorem*{theorem}{Theorem}
\newtheorem*{proposition}{Proposition}
\newtheorem{exercise}{Exercise}
\newtheorem*{lemma}{Lemma}
\newtheorem*{corollary}{Corollary}

\theoremstyle{definition}
\newtheorem*{definition}{Definition}
\newtheorem*{example}{Example}

\theoremstyle{remark}
\newtheorem*{remark}{Remark}

\renewcommand{\phi}{\varphi}
\renewcommand{\epsilon}{\varepsilon}
\renewcommand{\emptyset}{\varnothing}

\newcommand{\RR}{\mathbb{R}}
\newcommand{\ZZ}{\mathbb{Z}}
\newcommand{\NN}{\mathbb{N}}
\newcommand{\CC}{\mathbb{C}}
\newcommand{\QQ}{\mathbb{Q}}

\DeclareMathOperator{\ima}{\text{im}}

\begin{document}

\topmargin=-40pt
\rhead{Henry Woodburn}
\lhead{Math 655}
\renewcommand{\headrulewidth}{1pt}
\renewcommand{\headsep}{20pt}
\thispagestyle{fancy}

{\Huge \bfseries \noindent Homework 2}

\begin{enumerate}
    \item[1.] Suppose $X$ is $\sigma(X, X^*)$ separable. Let $\{x_n\}$ be a countable 
    $\sigma(X, X^*)$ dense set in $X$. Let $A = \overline{\operatorname{span}_\QQ\{a_n\}}^{\|\cdot\|}$ 
    be the norm closure of the rational
    span of the $a_n$'s. Then $A$ is a norm closed, convex set, and thus $A = \overline{A}^{\sigma(X ,X^*)}$
    by Mazur's theorem. But $A$ contains $\{a_n\}$ whose weak closure is $X$, so we must have 
    $A = X$. Then $\operatorname{span}_\QQ\{a_n\}$ is a countable set whose norm closure 
    is $X$, and thus $X$ is separable. 

    \item[2.] Suppose $(B_{X^*}, \sigma(X^*, X))$ is metrizable. Then $B_{X^*}$ is a $\sigma(X^*, X)$
    compact, metrizable space. Thus the space $C(B_{X^*})$ of $\sigma(X^{*}, X)$-continuous functions $B_{X^*} \rightarrow \RR$
    is separable in the supremum norm. Also note that the subspace of $C(B_{X^*})$ consisting 
    of functions which are linear is exactly the image of $X$ under the canonical 
    surjection $J: X \rightarrow X^*$.

    Then it follows that $J(X)$ is separable as a subspace of $C(B_{X^*})$, and thus 
    $X$ is separable since the map $J$ is an isometry. 

    \item[3.] 

\end{enumerate}

\end{document}