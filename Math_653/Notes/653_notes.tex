\documentclass[12pt, reqno]{article}

\usepackage{amsmath, amsthm, amssymb, amsfonts}
\usepackage{enumitem}
\usepackage{bookmark}
\usepackage{fullpage}
\usepackage{tcolorbox}
\usepackage{hyperref}
\usepackage{tikz, tikz-cd}
\usetikzlibrary{arrows.meta}
\usepackage{pdfpages}
\usepackage{mathrsfs}
\usepackage{fancyhdr}
\usepackage[bottom=0.5in, top=1in, left=0.5in, right=0.5in]{geometry}
\usepackage{array}   % for \newcolumntype macro
\newcolumntype{L}{>{$}l<{$}}

\theoremstyle{plain}
\newtheorem{theorem}{Theorem}[section]
\newtheorem{proposition}{Proposition}
\newtheorem{exercise}{Exercise}

\theoremstyle{definition}
\newtheorem*{definition}{Definition}
\newtheorem*{example}{Example}

\theoremstyle{remark}
\newtheorem*{remark}{Remark}

\renewcommand{\phi}{\varphi}
\renewcommand{\epsilon}{\varepsilon}
\renewcommand{\emptyset}{\varnothing}

\newcommand{\RR}{\mathbb{R}}
\newcommand{\ZZ}{\mathbb{Z}}
\newcommand{\NN}{\mathbb{N}}

\DeclareMathOperator{\ima}{\text{im}}

\begin{document}

\topmargin=-40pt
\rhead{Henry Woodburn}
\lhead{Math 653}
\renewcommand{\headrulewidth}{1pt}
\renewcommand{\headsep}{20pt}
\thispagestyle{fancy}

{\Huge \bfseries \noindent 653 Notes}

\section*{Group Theory}

Let $S$ be a set. A \textbf{Product} on $S$ is a function $S\times S \rightarrow S$,
where $(s,t) \mapsto s\cdot t$. 
If $s\cdot t = t \cdot s$, we say $\cdot$ is \textbf{commutative} and write $s + t$. 
A product is \textbf{associative} if $(s\cdot t)\cdot u = s \cdot (t \cdot u)$.
An element $e \in S$ is an \textbf{identity} if for all $s \in S$, we have $e\cdot s = s \cdot e = s$.
Identities are unique. 
A \textbf{Monoid} is a set $M$ equipped with an associative product that contains an identity.
\begin{example} 
    The set $\operatorname{func}(S)$ of functions on $S$ is a monoid under function 
    composition with identity $e: s \mapsto s$.
\end{example}

\begin{example}
    The subsets of a set $S$ form a monoid under intersection with identity $X$,
    as well as under set union with identity $\emptyset$.
\end{example}

If a monoid $M$ has a commutative product, $M$ is called an \textbf{abelian monoid}. A \textbf{submonoid} of a
monoid $M$ is a subset $H \subset M$ with $e \in H$ and $xy \in H$ for all $x, y \in H$. 

\begin{example}
    The set $\mathbb{N} = \{n \in \mathbb{Z}: n \geq 0\}$ is a monoid under $+$ with identity $0$, and under 
    $\cdot$ with identity $1$. The element $0$ is called absorbing in this case.
\end{example}

\begin{example}
    For all $a \in \mathbb{N}$, $a\mathbb{N}$ is a monoid under addition but not multiplication unless $a = 1$, 
    since it does not contain $1$. 
\end{example}

A \textbf{Group} $G$ is a monoid such that for every $x \in G$, there exists a $y \in G$ such that $xy = e$. In this case
we write $y = x^{-1}$. Note that $xy = e$ implies that $yx = e$. In a group, both inverses and the identity are unique.
In a group, equations $ax = b$ and $xa = b$ have unique solutions. A \textbf{Subgroup} of a group $G$ is a submonoid
of $G$ that is closed under the action of taking inverse.

\begin{example}
    $\{e\}$ is a trivial example of a group. $\mathbb{Z}, \mathbb{Q}, \mathbb{R}$, and $\mathbb{C}$ are all
    examples of groups under addition.
\end{example}

\begin{example}
    $\mathbb{Q}^\times := \mathbb{Q} \setminus \{0\}$ is a group under multiplication, along with $\mathbb{R}^\times$ 
    and $\mathbb{C}^\times$, defined in an analagous way.
\end{example}

\begin{example}
    The unit complex numbers $S^1$ form a group under complex multiplication
\end{example}

\begin{example}
    Let $S$ be a set and define $\operatorname{Sym}(S)$ to be the set of bijections $S \rightarrow S$. Then $\operatorname{Sym}(S)$
    is a group under composition called the \textbf{Symmetric Group} on $S$.
\end{example}

Let $M, M'$ be monoids with identities $e, e'$ respectively. A \textbf{homomorphism} of monoids is a function 
$f: M \rightarrow M'$ such that $f(e) = e'$, and for all $x,y \in M$, we have $f(xy) = f(x)f(y)$. A monoid homomorphism
between groups is a group homomorphism.
\bigbreak
We say a group is \textbf{cyclic} if there exists $a \in G$ such that any $g \in G$ can be written $g = a^n$ for 
some $n \in \mathbb{Z}$. When this occurs, we say $a$ \textbf{generates} $G$. 

\begin{example}
    $\mathbb{Z}$ has two generators, $1$ and $-1$.
\end{example}

\begin{example}
    The $n$th roots of unity, denoted $C_n$, has generators $e^{2\pi\frac{k}{n}}$, where $\gcd(n,k) = 1$.
\end{example}

Let $G$ and $H$ be groups. We can define a product on $G\times H$ by $(g_1, h_1)\cdot(g_2,h_2) = (g_1 g_2, h_1 h_2)$.
Then $G\times H$ is a group with identity $e = (e_G,e_H)$ and with inverse $(g,h)^{-1} = (g^{-1}, h^{-1})$. This 
construction generalizes to arbitrary product with component-wise multiplication.
\bigbreak
Let $G$ be a group and $S \subset G$. We define $\langle S\rangle$, the subgroup \textbf{generated} by $S$ to
be the collection of all finite combinations of elements of $S$. Equivalently, $\langle S \rangle$ is the smallest
subgroup of $G$ containing $S$, or the intersection of all subgroups containing $S$. If $a \in G$, the order of $a$
is the smallest $n > 0$ such that $a^n = e$. Equivalently the order of $a$ is the number of elements in $\langle a \rangle$. 

\begin{remark}
    Suppose $S \subset G$ and $G = \langle S \rangle$. Then any homomorphism $G \rightarrow H$ 
    is determined by its restriction to $S$. 

    Not all functions $\phi: S \rightarrow H$ give homomorhpisms. 
\end{remark}

\begin{definition}
    An isomorphism $G \rightarrow G$ is called an automorphism. We denote $\operatorname{Aut}(G)$
    the set of automorphisms of a group $G$.
\end{definition}
dy
\begin{example}
    For $m \in \mathbb{Z}$, $a \mapsto a\cdot m$ is a homomorphism $\mathbb{Z} \rightarrow \mathbb{Z}$.
    If $m \neq 0$, the map is an injective homomorphism, called a monomorphism.
\end{example}

\begin{definition}
    We denote $\ZZ_m$ the set of integers mod $m$.

    The map $a \mapsto a \mod m$ is a homomorphism $\ZZ \rightarrow \ZZ_m$.
\end{definition}

\begin{example}
    The exponential map is a homomorphism $(\RR, +) \rightarrow (\RR_>, \cdot)$. The inverse
    map is the logarithm.
\end{example}

\begin{theorem}
    Let $f$ be a group homomorphism. Then $\ker f = \{e\}$ if an only if $f$ is injective.
\end{theorem}

\begin{proposition}[Internal Direct Product]
    Let $G$ be a group with subgroups $H$ and $K$, such that $H \cap K = \{e\}$, and 
    $H\cdot K = G$, and $hk = kh$ for all $h \in H, k \in K$. Then the map 
    $\phi: H \times K \rightarrow G$ given by $(h,k) \mapsto h\cdot k$ is an isomorphism.
\end{proposition}

\textit{Proof.} $\phi$ is surjective by $H\cdot K = G$. Homomorphism easy to check. To show injective, 
if $\phi(h,k) = e$, then $hk = e$ and $k \in H$, therefore $k = e$. The same applies for $h = e$. Then $(h,k) = (e,e)$.

\subsection*{Cosets and Lagrange's Theorem}

\begin{definition}
    Let $H$ be a subgroup of a group $G$. A left (right) coset of $H$ in $G$ is a subset of the form 
    $aH$ ($Ha$) for some $a \in G$. 
\end{definition}

\begin{theorem}
    Let $H$ be a subgroup of a group $G$. Then
    \begin{itemize}
        \item $aH = bH$ iff $b \in aH$ iff $aH \cap bH \neq \emptyset$ iff $b^{-1}a \in H$
        \item for all $a \in G$, $H$ and $aH$ are in non-canonical bijection
        \item the relation $a \sim b$ if $aH = bH$ is an equivalence relation on $G$.
        \item the map $aH \mapsto Ha^{-1}$ is a bijection between left and right cosets of $H$.
    \end{itemize}
\end{theorem}

\begin{definition}
    The index of a subgroup $H$ of $G$, denoted $[G:H]$, is the cardinal number of the set of right cosets of $H$
    in $G$. 
\end{definition}

\begin{theorem}
    Let $G$ be a group and $H$ a subgroup. Then $|G| = [G:H]\cdot |H|$.
\end{theorem}

\textit{Proof.} The cosets of $H$ partition $G$ and are equinumerous with $H$.

\subsection*{Normal Subgroups}

\begin{definition}
    A subgroup $N$ of $G$ is called normal if for all $g \in G$, $gN = Ng$. 
\end{definition}

\begin{theorem}
    Let $N$ be normal in $G$ and let $G/N$ be the set of cosets of $N$ in $G$. Then $G/N$ is a group 
    with product $aN \cdot bN = abN$. We call $G/N$ the quotient or factor group of $G$ by $N$.
\end{theorem}

\textit{Proof.} Let $\alpha \in aN$ and $\beta \in bN$. Then there exist $m,n \in N$ such that $\alpha = an$
and $\beta = bm$. Then $\alpha \cdot \beta = anbm = ab(b^{-1}nb)m \in ab N$. 

One also must check for inverses and identity. 

We call the map $G \rightarrow G/N$ sending $a \rightarrow aN$ the canonical surjection/map. $N$ is the kernel 
of the canonical surjection. 

\begin{definition}
    A sequence 
    \begin{equation*}
        \begin{tikzcd}
            A \arrow[r, "f"] & G \arrow[r, "g"]  & K
        \end{tikzcd}
    \end{equation*}
    is called exact at $G$ if $\ker g = \text{im} f$.
\end{definition}

If $N \trianglelefteq G$, then 
\begin{equation*}
    \begin{tikzcd}
        0 \arrow[r, "i"] & N \arrow[r, "j"] & G \arrow[r, "\phi"] & G/N \arrow[r, "f"] & 0
    \end{tikzcd}
\end{equation*}

is exact everywhere.

Suppose 
\begin{equation*}
    \begin{tikzcd}
        e \arrow[r] & H \arrow[r, "f"] & G \arrow[r, "g"] & K \arrow[r] & e
    \end{tikzcd}
\end{equation*}

is exact. We call this a short exact sequence. Let $N = \text{im} f$. Then we get a commutative diagram
\begin{equation*}
    \begin{tikzcd}
        e \arrow[r] & H \arrow[r, "f"] \arrow[d, "f"] & G \arrow[r, "g"] \arrow[d] & K \arrow[r, "p"] \arrow[d,"\psi"] & e\\
        e \arrow[r] & N \arrow[r, "i"] & G \arrow[r, "\phi"] & G/N \arrow[r] & 0
    \end{tikzcd}
\end{equation*}
where the vertical arrows are isomorphisms.

\textit{Proof.} Let $k \in K$. There exists $a \in G$ such that $g(a) = k$ since $\ima g = \ker p = K$.
Then $\phi(a) \in G/N$. Set $\psi(k) = \phi(a)$. Suppose $g(b) = k$. Then $\phi(a)\phi(b)^{-1} = \phi(ab^{-1}) = e$,
because $g(a) = g(b)$ implies $ab^{-1} \in \ker g = \ima f = N = \ker \phi$.


\end{document}