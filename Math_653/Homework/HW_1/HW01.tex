%HW01.tex
%
% First Homework for Graduate Algebra
% Frank Sottile
%%%%%%%%%%%%%%%%%%%%%%%%%%%%%%%%%%%%%%%%%%%%%%%%%%%%%%%%%%%%%%%%%%%%%%%
\documentclass[12pt]{article}
\usepackage{multicol,amsfonts, amssymb,  mathtools,amsmath}
\usepackage{colordvi,graphicx}
\headheight=8pt
%
\topmargin=-75pt
\textheight=720pt   \textwidth=560pt
\oddsidemargin=-60pt \evensidemargin=-60pt


\pagestyle{empty}

%%%%%%%%%%%%%%%%%%%%%%%%%%%%%%%%%%%%%%%%%%%%
\newcommand{\CC}{{\mathbb C}}
\newcommand{\KK}{{\mathbb K}}
\newcommand{\NN}{{\mathbb N}}
\newcommand{\QQ}{{\mathbb Q}}
\newcommand{\RR}{{\mathbb R}}
\newcommand{\TT}{{\mathbb T}}
\newcommand{\ZZ}{{\mathbb Z}}

\newcommand{\calA}{{\mathcal A}}
\newcommand{\be}{{\bf e}}

\newcommand{\Hom}{\mbox{Hom}}
\newcommand{\spec}{\mbox{spec}}
\newcommand{\cone}{\mbox{cone}}

\newcommand{\vect}[2]{(\begin{smallmatrix}#1\\#2\end{smallmatrix})}

\def\Color#1#2{\special{color push cmyk #1}#2\special{color pop}}
%\def\Indigo#1{\Color{.42 1. 0. .49}{#1}}
\def\Indigo#1{\Color{1. .95 .05 .4}{#1}}
\def\MyViolet#1{\Color{.6 1. 0. .15}{#1}}
\def\TAMU#1{\Color{.15 1. .39 .69}{#1}}

\newcommand{\barsl}{\noindent\begin{minipage}[t]{590pt}
\Indigo{\rule{590pt}{1.2pt}}\vspace{-5.7mm}\\
\MyViolet{\rule{590pt}{1.2pt}}\vspace{-5.7mm}\\
\Blue{\rule{590pt}{1.2pt}}\vspace{-5.7mm}\\
\Green{\rule{590pt}{1.2pt}}\vspace{-5.7mm}\\
\Yellow{\rule{590pt}{1.2pt}}\vspace{-5.7mm}\\
\Orange{\rule{590pt}{1.2pt}}\vspace{-5.7mm}\\
\Red{\rule{590pt}{1.2pt}}\bigskip
\end{minipage}}


\newcommand{\barsn}{\noindent\begin{minipage}[t]{590pt}
\Indigo{\rule{590pt}{1.1pt}}\vspace{-4.5mm}\\
\MyViolet{\rule{590pt}{1.1pt}}\vspace{-4.5mm}\\
\Blue{\rule{590pt}{1.1pt}}\vspace{-4.5mm}\\
\Green{\rule{590pt}{1.1pt}}\vspace{-4.5mm}\\
\Yellow{\rule{590pt}{1.1pt}}\vspace{-4.5mm}\\
\Orange{\rule{590pt}{1.1pt}}\vspace{-4.5mm}\\
\Red{\rule{590pt}{1.1pt}}\bigskip
\end{minipage}}

\def\demph#1{\TAMU{{\sl #1}}}
\def\defcolor#1{\TAMU{#1}}

\begin{document}
\LARGE 
\noindent
Algebra \ \ Autumn 2025\vspace{1pt}\\
Frank Sottile\vspace{1pt}\\
\Large 26 August 2025 \hfill
\sf
 First Homework\makebox[40pt][l]{\ }
\normalsize\vspace{10pt}

\noindent
Write your answers neatly, in complete sentences.  
I highly recommend recopying your work before handing it in.
Correct and crisp proofs are greatly appreciated; oftentimes your work can be shortened and made clearer.

\barsn

\noindent\Maroon{{\sf Hand in at the start of class, Thursday 28 August:}} 

\begin{enumerate}

%%%%%%%%%%%%%%%%%%%%%%%%%%%%%%%%%%%%%%%%%%%%%%%%%%%%%%%%%%%%%%%%%%%%%%%%%%%%%%%%%%%%%%%%%%%%%%%%%%%%
\item[0.] Read Sections I.1 and I.2 of Lang's Algebra.
%%%%%%%%%%%%%%%%%%%%%%%%%%%%%%%%%%%%%%%%%%%%%%%%%%%%%%%%%%%%%%%%%%%%%%%%%%%%%%%%%%%%%%%%%%%%%%%%%%%%
\item
  We explained that \defcolor{$\mbox{Func}(S)$}, the set of functions $f\colon S\to S$, where $S$ is  set, forms a monoid.
  Let $S\vcentcolon=\{1,2\}$, a set with two elements.
  We write the elements of $S$ in 2-line notation as
  \[
    e\vcentcolon=\begin{pmatrix}1&2\\1&2\end{pmatrix}\qquad
  \alpha\vcentcolon=\begin{pmatrix}1&2\\2&1\end{pmatrix}\qquad
  \beta\vcentcolon=\begin{pmatrix}1&2\\1&1\end{pmatrix}\qquad
  \gamma\vcentcolon=\begin{pmatrix}1&2\\2&2\end{pmatrix}\,.
  \]
  Please give the composition (multiplication) table for this monoid $\mbox{Func}(\{1,2\})$.

  \begin{tabular}{c||c|c|c|c|}
     $\circ$  & $e$  & $\alpha$ & $\beta$ & $\gamma$ \\\hline\hline
     $e$      &      & & & \\\hline
     $\alpha$ &      & &$\alpha\circ\beta$ & \\\hline
     $\beta$  &      & & & \\\hline
     $\gamma$ &      & & & \\\hline
  \end{tabular}

  (Evaluate the composition $\alpha\circ\beta$, place it in that cell, and do the same for the other 15 cells.)
  
  Is this monoid commutative?

%%%%%%%%%%%%%%%%%%%%%%%%%%%%%%%%%%%%%%%%%%%%%%%%%%%%%%%%%%%%%%%%%%%%%%%%%%%%%%%%%%%%%%%%%%%%%%%%%%%%
\item 
  Let \defcolor{$\RR_{\geq 0}$} be the monoid of nonnegative real numbers under addition, and let
  \defcolor{$\RR_+$} denote the monoid of positive real numbers under multiplication.  
  \begin{enumerate}
  \item Find the smallest submonoid of $\RR_{\geq 0}$ that contains $\sqrt{3}$.
    (That is, describe it elements as a set.
    The term ``smallest'' here means, by defintion, the submonoid that is contained in any other submonoid that contains $\sqrt{3}$.)
  \item Find the smallest submonoid of $\RR_+$ that contains $\sqrt{3}$.
  \item Are either of  $\RR_{\geq 0}$ or $\RR_+$  a group?
        If so, do the answers to the previous quesions change if ``submonoid'' is replaced by ``subgroup''?
  \end{enumerate}
  (You need not justify your answers.)
%%%%%%%%%%%%%%%%%%%%%%%%%%%%%%%%%%%%%%%%%%%%%%%%%%%%%%%%%%%%%%%%%%%%%%%%%%%%%%%%%%%%%%%%%%%%%%%%%%%%
  
%%%%%%%%%%%%%%%%%%%%%%%%%%%%%%%%%%%%%%%%%%%%%%%%%%%%%%%%%%%%%%%%%%%%%%%%%%%%%%%%%%%%%%%%%%%%%%%%%%%%
\item Let $\defcolor{B_2(\RR)}\vcentcolon=\left\{ \left(\begin{smallmatrix}a&b\\0&c\end{smallmatrix}\right)
                 \mid a,b,c\in\RR,\ ac\neq 0\right\}$.
  Show that $B_2(\RR)$ is a subgroup of $\mbox{GL}(2,\RR)$, under matrix multiplication.
  (This subgroup is called a \demph{Borel subgroup}.)
%%%%%%%%%%%%%%%%%%%%%%%%%%%%%%%%%%%%%%%%%%%%%%%%%%%%%%%%%%%%%%%%%%%%%%%%%%%%%%%%%%%%%%%%%%%%%%%%%%%%

%%%%%%%%%%%%%%%%%%%%%%%%%%%%%%%%%%%%%%%%%%%%%%%%%%%%%%%%%%%%%%%%%%%%%%%%%%%%%%%%%%%%%%%%%%%%%%%%%%%%
\item Let \defcolor{$\CC^\times$} be the group of nonzero complex numbers under multiplication.
  Define $f\colon \CC^\times\to \mbox{GL}(2,\RR)$ by $f(z)=\left(\begin{smallmatrix} a(z) & b(z)\\-b(z)&a(z)\end{smallmatrix}\right)$, where
  $a(z)\vcentcolon= \frac{z+\overline{z}}{2}$ and $b(z)\vcentcolon= \frac{z-\overline{z}}{2\sqrt{-1}}$ are the real and
  imaginary parts of the complex number $z$.

 Show that $f$ is an injective group homomorphism.
%%%%%%%%%%%%%%%%%%%%%%%%%%%%%%%%%%%%%%%%%%%%%%%%%%%%%%%%%%%%%%%%%%%%%%%%%%%%%%%%%%%%%%%%%%%%%%%%%%%%

%%%%%%%%%%%%%%%%%%%%%%%%%%%%%%%%%%%%%%%%%%%%%%%%%%%%%%%%%%%%%%%%%%%%%%%%%%%%%%%%%%%%%%%%%%%%%%%%%%%%
\item
  The \demph{center} of a group $G$ is the set $\defcolor{Z(G)}\vcentcolon=\{a\in G\mid ag=ga\ \forall g\in G\}$.
  For a fixed $g\in G$, the \demph{centralizer} of $g$ is the set
  $\defcolor{C_G(g)}\vcentcolon=\{ a\in G\mid ag=ga\}$.
  Prove that both $Z(G)$ and $C_G(g)$ are subgroups of $G$.
%%%%%%%%%%%%%%%%%%%%%%%%%%%%%%%%%%%%%%%%%%%%%%%%%%%%%%%%%%%%%%%%%%%%%%%%%%%%%%%%%%%%%%%%%%%%%%%%%%%%
  
    
      
\end{enumerate}
%%%%%%%%%%%%%%%%%%%%%%%%%%%%%%%%%%%%%%%%%%%%%%%%%%%%%%%%%%%%%%%%%%%%%%%%%%%%%%%%%%%%%%%%%%%%%%%%%%%%

\end{document}
