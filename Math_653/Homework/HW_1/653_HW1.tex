\documentclass[11pt, reqno]{article}

\usepackage{amsmath, amsthm, amssymb}
\usepackage{enumitem}
\usepackage{bookmark}
\usepackage{fullpage}
\usepackage{tcolorbox}
\usepackage{hyperref}
\usepackage{tikz}
\usetikzlibrary{arrows.meta}
\usepackage{pdfpages}
\usepackage{mathrsfs}
\usepackage{fancyhdr}
\usepackage[bottom=0.5in, top=1in, left=0.5in, right=0.5in]{geometry}
\usepackage{array}   % for \newcolumntype macro
\newcolumntype{L}{>{$}l<{$}}

\begin{document}


\topmargin=-40pt
\rhead{Henry Woodburn}
\lhead{Math 653}
\renewcommand{\headrulewidth}{1pt}
\renewcommand{\headsep}{20pt}
\thispagestyle{fancy}

\section*{Homework 1}

\begin{enumerate}
    \item Let $S = \{1, 2\}$, a set with 2 elements. We write the elements of $\operatorname{Func}(S)$
    as
    \[
    e := \begin{pmatrix}
        1 & 2 \\ 1 & 2
    \end{pmatrix}
    \qquad 
    \alpha := \begin{pmatrix}
        1 & 2 \\ 2 & 1
    \end{pmatrix}
    \qquad 
    \beta := \begin{pmatrix}
        1 & 2 \\ 1 & 1
    \end{pmatrix}
    \qquad 
    \gamma := \begin{pmatrix}
        1 & 2 \\ 2 & 2
    \end{pmatrix}.
    \]

    Then the multiplication table for the monoid $\operatorname{Func}(S)$ under composition is as follows:

    \begin{tabular}{L || L | L | L | L}
        \circ   &    e     &   \alpha  &   \beta   &   \gamma\\
        \hline
        e       &    e     &   \alpha  &   \beta   &   \gamma\\
        \hline
        \alpha  &   \alpha &   e       &   \gamma  &   \beta\\
        \hline
        \beta   &   \beta  &   \beta   &   \beta   &   \beta\\
        \hline
        \gamma  & \gamma   &   \gamma  &   \gamma  &   \gamma 
    \end{tabular}

    This monoid is not commutative since $\gamma \circ \beta = \beta$ but $\beta \circ \gamma = \gamma$. 

    \item Let $\mathbb{R}_{\geq 0}$ be the monoid of nonnegative real numbers under addition,
    and let $\mathbb{R}^+$ denote the monoid of positive real numbers under multiplication.
    \begin{enumerate}
        \item[(a.)] We know that any submonoid of $\mathbb{R}_{\geq 0}$ must contain $0$. Also,
        any submonoid containing $\sqrt{3}$ must contain $\sqrt{3}\mathbb{N}$, where $\mathbb{N}$
        is the monoid of nonnegative integers under addition. $\sqrt{3}\mathbb{N}$ is in fact the smallest submonoid
        containing $\sqrt{3}$, since no smaller subset is a monoid, and any submonoid
        containing $\sqrt{3}$ must contain $\sqrt{3}\mathbb{N}$.

        \item[(b.)] A submonoid of $\mathbb{R}^+$ containing $\sqrt{3}$ necessarily contains 
        $\sqrt{3}^n$ for $n = 0, 1, 2, \dots$. The set $\{\sqrt{3}^n: n = 0, 1, 2, \dots\}$
        is a submonoid of $\mathbb{R}^+$ under multiplication, and removing any element makes it not a monoid. 
        This is the smallest monoid containing $\sqrt{3}$.

        \item[(c.)] $\mathbb{R}_{\geq 0}$ is not a group because none of the positive
        reals have additive inverses.

        $\mathbb{R}^+$ is a group because it is a monoid where every $r \in \mathbb{R}^+$ has 
        an inverse, namely $1/r$ since $r > 0$.

        If we want to find the smallest subgroup of $\mathbb{R}^+$ containing $\sqrt{3}$,
        we need all integer powers of $\sqrt{3}$, not just the nonnegative ones. It is trivial
        to check that this also defines a submonoid. Also, every
        element $(\sqrt{3})^n$ has inverse $(\sqrt{3})^{-n}$, and we still have identity $1 = (\sqrt{3})^0$.
    \end{enumerate}

    \item The identity in $\operatorname{GL}(2, \mathbb{R})$ is the matrix $\left(\begin{smallmatrix}
        1 & 0 \\ 0 & 1
    \end{smallmatrix}\right)$. This is indeed contained in 
    $B_2(\mathbb{R}) := \{\left(\begin{smallmatrix} a & b \\ 0 & c \end{smallmatrix}\right): a,b,c \in \mathbb{R},\ ac \neq 0\}$.

    We check that $B_2(\mathbb{R})$ is closed under multiplication: for $\left(\begin{smallmatrix}
        a & b \\ 0 & c
    \end{smallmatrix}\right), \left(\begin{smallmatrix}
        d & e \\ 0 & f
    \end{smallmatrix}\right) \in B_2(\mathbb{R})$, we have
    \[
    \begin{pmatrix}
        a & b \\ 0 & c
    \end{pmatrix}
    \begin{pmatrix}
        d & e \\ 0 & f
    \end{pmatrix}
    = \begin{pmatrix}
        ad & ea + bf \\ 0 & cf
    \end{pmatrix} \in B_2(\mathbb{R}),
    \]
    since $adcf = (ac)(df) \neq 0$.

    Also, if 
    \[
    \begin{pmatrix}
        a & b \\ 0 & c
    \end{pmatrix}
    \begin{pmatrix}
        d & e \\ 0 & f
    \end{pmatrix}
    = \begin{pmatrix}
        1 & 0 \\ 0 & 1
    \end{pmatrix},
    \]
    Then $ad = cf = 1$, and $ea + bf = 0$. Then we can solve for $d, e, f$ to get $d = a^{-1}$, 
    $f = c^{-1}$, and $e = -cab^{-1}$, giving us the inverse of $\left(\begin{smallmatrix}
        a & b \\ 0 & c
    \end{smallmatrix}\right)$.

    \item To see that $f$ is injective, we note that a complex number is uniquely determined by 
    its real and imaginary parts. 

    We have 
    \[
    f((a + bi)(c + di)) = f((ac - bd) + (bc + ad)i) = \begin{pmatrix}
        ac - bd & bc + ad \\ -(bc + ad) & ac - bd
    \end{pmatrix}
    \]
    \[
    = \begin{pmatrix}
        a & b \\ -b & a
    \end{pmatrix}
    \begin{pmatrix}
        c & d \\ -d & c
    \end{pmatrix}
    =f(a + bi)f(c + di),\]
    so that $f$ is a monoid homomorphism and thus a group homomorphism.

    \item We have $e \in Z(G)$ since for any $g \in G$, $eg = ge = g$.
    
    If $a \in Z(G)$, then $a^{-1} \in Z(G)$, since for any $g \in G$, 
    \[a^{-1}g = a^{-1}gaa^{-1} = 
    a^{-1}aga^{-1} = ga^{-1}.\]

    If $a,b \in Z(G)$, then the product $ab \in Z(G)$ since for any $g \in G$,
    \[
    gab = agb = abg.
    \]

    Then $Z(G)$ is a subgroup since it contains identity, is closed, and contains inverses.

    $C_G(g)$ is a subgroup for the same reasons and the proof is similar.

\end{enumerate}

\end{document}