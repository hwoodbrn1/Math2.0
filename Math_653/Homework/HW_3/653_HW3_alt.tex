\documentclass[11pt, reqno]{article}

\usepackage{amsmath, amsthm, amssymb}
\usepackage{enumitem}
\usepackage{tcolorbox}
\usepackage{hyperref}
\usepackage{tikz}
\usetikzlibrary{arrows.meta}
\usepackage{mathrsfs}
\usepackage{fancyhdr}
\usepackage[bottom=0.75in, top=1in, left=0.5in, right=0.5in]{geometry}
\usepackage{array}   % for \newcolumntype macro
\newcolumntype{L}{>{$}l<{$}}

\theoremstyle{plain}
\newtheorem*{theorem}{Theorem}
\newtheorem*{proposition}{Proposition}
\newtheorem{exercise}{Exercise}
\newtheorem*{lemma}{Lemma}
\newtheorem*{corollary}{Corollary}

\theoremstyle{definition}
\newtheorem*{definition}{Definition}
\newtheorem*{example}{Example}

\theoremstyle{remark}
\newtheorem*{remark}{Remark}

\renewcommand{\phi}{\varphi}
\renewcommand{\epsilon}{\varepsilon}
\renewcommand{\emptyset}{\varnothing}

\newcommand{\RR}{\mathbb{R}}
\newcommand{\ZZ}{\mathbb{Z}}
\newcommand{\NN}{\mathbb{N}}
\newcommand{\CC}{\mathbb{C}}

\DeclareMathOperator{\ima}{\text{im}}

\begin{document}

\topmargin=-40pt
\rhead{Henry Woodburn}
\lhead{Math 653}
\renewcommand{\headrulewidth}{1pt}
\renewcommand{\headsep}{20pt}
\thispagestyle{fancy}

{\Huge \bfseries \noindent Homework 3}

\begin{enumerate}
    \item[13.] Let $a \in G$, we will show $a(H \cap K) = (aH) \cap (aK)$. Let $x \in a(H \cap K)$.
    Then $x = ar$ for some $r \in H \cap K$, thus $x \in aH \cap aK$. 

    If $x \in (aH) \cap (aK)$, we have $x = ah = ak$ for $h \in H$, $k \in K$, and thus $h = k$ and 
    $x \in a(H \cap K)$. 

    Now suppose $H$ and $K$ have finite index. Let $\phi: G/H\cap K \rightarrow G/H \times G/K$ be 
    the map $a(H\cap K) \mapsto (aH, aK)$. We will show $\phi$ is an injection, proving the theorem. 
    First note that $\phi$ is well defined, since $a(H \cap K) = b(H \cap K)$ if and only if 
    $(aH) \cap (aK) = (bH) \cap (bK)$, in which case $aH \cap bH \neq \emptyset$ and $aH = bH$.
    similarly $aK = bK$. 

    Then if $\phi(a(H\cap K)) = \phi(b(H \cap K))$, we must have $(aH, aK) = (bH, bK)$. This gives
    $aH = bH$, $aK = bK$, meaning $(aH) \cap (aK) = (bH) \cap (bK)$ and finally $a(H\cap K) = b(H \cap K)$.
    This proves $\phi$ is injective. Then we have an injection into a set of finite cardinality, so $G/H\cap K$
    must have finite cardinality. 
\end{enumerate}

\end{document}