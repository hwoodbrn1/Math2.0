\documentclass[11pt, reqno]{article}

\usepackage{amsmath, amsthm, amssymb}
\usepackage{enumitem}
\usepackage{bookmark}
\usepackage{fullpage}
\usepackage{tcolorbox}
\usepackage{hyperref}
\usepackage{tikz}
\usetikzlibrary{arrows.meta}
\usepackage{pdfpages}
\usepackage{mathrsfs}
\usepackage{fancyhdr}
\usepackage[bottom=0.5in, top=1in, left=0.5in, right=0.5in]{geometry}
\usepackage{array}   % for \newcolumntype macro
\newcolumntype{L}{>{$}l<{$}}

\begin{document}

\topmargin=-40pt
\rhead{Henry Woodburn}
\lhead{Math 636}
\renewcommand{\headrulewidth}{1pt}
\renewcommand{\headsep}{20pt}
\thispagestyle{fancy}

\section*{Homework 1}

\begin{enumerate}
    \item[13.1] Let $X$ be a topological space and $A \subset X$. Suppose for each $x \in A$ there is an open 
    set $U_x$ with $x \in U_x \subset A$. 

    Then I claim $U := \bigcup_{x \in A} U_x = A$. Since each $x \in A$ is in some $U_x$, clearly $A \subset U$.
    Conversely, each $U_x$ is a subset of $A$, so their union must be. Then $A$ is a union of open sets 
    and is open.

    \item[13.3] Let $X$ be any set. Let $\mathcal{T}_c$ be the collection of subsets $U$ of $X$ such that 
    $X \setminus U$ is either countable or all of $X$. 

    $\mathcal{T}_c$ contains the empty set and $X$ since $X \setminus \emptyset = X$ and $X \setminus X = \emptyset$,
    which is countable. Also, if ${\{U_\alpha\}}_{\alpha \in A}$ is any collection of open sets,
    we have 
    \[
    X \setminus \bigcup_{\alpha \in A}U_\alpha = X \cap \left(\bigcup_{\alpha \in A}U_\alpha\right)^c 
    = X \cap \bigcap_{\alpha \in A}{U_\alpha}^c = \bigcap_{\alpha \in A}X \cap U_\alpha^c = 
    \bigcap_{\alpha \in A}X \setminus U_\alpha
    \]
    by De Morgan's laws, and thus $\bigcup_{\alpha \in A}U_\alpha \in \mathcal{T}_c$, since 
    any intersection of countable sets is countable.

    Similarly, if $i_1, \dots, i_n \in A$, we have 
    \[
    X\setminus \bigcap_1^n U_{i_n} = \bigcup A\setminus U_{i_n},
    \]

    which is a finite union of countable sets and is countable. Then $\bigcap_1^n U_{i_n} \in \mathcal{T}_c$.

    Then we have shown that $\mathcal{T}_c$ is a topology.

    Now let $\mathcal{T}_\infty$ be the subsets $U$ of $X$ such that $X \setminus U$ is infinite, empty,
    or all of $X$. $\mathcal{T}_\infty$ is not a topology. For example, let $X = \mathbb{N}$. Then
    $\{x\}$ is open for $x \in X$, but $V := \bigcup_{x \notin \{0\}}\{x\}$ is a union of open sets 
    that is not open, as $X \setminus V = \{0\}$. 

    \item[13.6] Let $\mathbb{R}_\ell$ be the reals equipped with the lower limit topology, and let $\mathbb{R}_K$ be 
    the reals equipped with the $K$-topology. Here $K := \{1/z: z \in \mathbb{Z}\}$ and the $K$ topology 
    is generated by the collection of open intervals $(a,b)$ along with sets of the form $(a,b) \setminus K$.

    To show that the topologies on $\mathbb{R}_\ell$ and $\mathbb{R}_K$ are not comparable, we must show the existence
    of sets $U, V \in \mathbb{R}$ such that $U$ is open in the $K$-topology and not in the lower limit topology, and
    $V$ is open in the lower limit topology but not the $K$-topology. 

    Let $U = (-2,2) \setminus K$. $U$ is clearly open in the $K$-topology, but it is not open in the lower limit
    topology. For $0 \in U$, suppose there is an interval $\left[a,b\right)$ containing $0$. Then $0 < b$, 
    so there is some $n \in \mathbb{Z}$ such that $0 < 1/n < b$ and thus $\left[a,b\right) \not\subset U$. Then 
    $U$ is not open in the lower limit topology, which is generated by all such half-open intervals. 

    Let $V = \left[2025, 2026\right)$. Then $V$ is not open in the $K$-topology since there is no open interval
    containing $2025$ which is contained in $V$, and the only basic open sets from the $K$-topology contained in 
    $V$ are open intervals. 

    \item[16.1] Let $X$ be a topological space, $Y$ a subspace, and $A$ a subset of $Y$. Let $\mathcal{T}_Y$ be
    the topology $A$ inherets as a subspace of $Y$ and let $\mathcal{T}_X$ be the topology $A$ inherets as a subspace 
    of $X$. 

    If $U$ is in $\mathcal{T}_Y$, then $U = W \cap A$ for some $W$ open in $Y$. We can write $W = K \cap Y$ for some
    $K$ open in $X$ since $Y$ is a subspace of $X$. Then we have $U = K \cap Y \cap A = K \cap A$, showing that $U$
    is open in $\mathcal{T}_X$. 

    Likewise if $V$ is in $\mathcal{T}_X$, Then $V = W \cap A$ for some $W$ open in $X$. Since $A \subset Y$, 
    $V = W \cap Y \cap A = (W \cap Y) \cap A$. Then $W \cap Y$ is open in $Y$ and $V$ is in $\mathcal{T}_Y$.

    \item[16.4] Let $X$ and $Y$ be topological spaces, and give $X \times Y$ the product topology. 
    Let $\pi_1: X \times Y \rightarrow X$ and $\pi_2: X \times Y \rightarrow Y$ be the projection maps.

    Let $U$ be open in $X\times Y$ and let $x \in \pi_1(U)\subset X$. Then choose a point $z \in U$ such that 
    $\pi_1(z) = x$ using the surjectivity of $\pi_1$. Since $U$ is open, there is a basic open set 
    $A \times B$ for open sets $A \subset X$ and $B\subset Y$, such that $z \in A\times B \subset U$.

    Then $x = \pi_1(z) \in \pi_1(A\times B) = A \subset \pi_(U)$, showing that $\pi_1(U)$ is open.

    A similar argument shows that $\pi_2$ is an open map. 

    \item[16.9] Let $\mathcal{T}_d$ be the dictionary order topology on $\mathbb{R}\times\mathbb{R}$ and let $\mathcal{T}_p$ 
    be the product topology on $\mathbb{R}_d\times\mathbb{R}$ where $\mathbb{R}_d$ is $\mathbb{R}$ with the discrete
    topology and $\mathbb{R}$ has the standard topology. Let $\mathcal{B}_d$ be the basis for $\mathcal{T}_d$ consisting
    of intervals of the form $(a\times b, c\times d)$ where either $a < c$ or $a = c$ and $b < d$. 
    Let $\mathcal{B}_p$ be the basis for $\mathcal{T}_d$ consisting of intervals $(a\times b, a\times c)$ for $b < c$,
    possibly $\pm \infty$.
    
    Since $\mathcal{B}_p \subset \mathcal{B}_d$, we must have $\mathcal{T}_p \subset \mathcal{B}_d$.
    
    In the other direction, we can write $(a\times b, c\times d) = (a\times b, a\times \infty)
    \cup \bigcup_{x \in (a,c)} (x\times -\infty, x\times \infty) \cup (c\times -\infty, c\times c)$.
    Then $\mathcal{B}_d \subset \mathcal{T}_p$ and thus $\mathcal{T}_d \subset \mathcal{T}_p$.

    I claim that $\mathcal{T}_d$ is strictly finer than the standard topology on $\mathbb{R}\times\mathbb{R}$.
    All products of intervals $A\times B$ are unions $\cup_{x \in A} x\times B$, but the interval $(a\times b, a\times c)$ 
    cannot contain any product of intervals $A\times B$ since this would imply $B \subset \{a\}$, which is not possible.

\end{enumerate}

\end{document}