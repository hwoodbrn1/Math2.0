\documentclass[11pt, reqno]{article}

\usepackage{amsmath, amsthm, amssymb}
\usepackage{enumitem}
\usepackage{tcolorbox}
\usepackage{hyperref}
\usepackage{tikz}
\usepackage{tikz-cd}
\usepackage{pgfplots}
\pgfplotsset{compat=1.18}
\usetikzlibrary{arrows.meta}
\usepackage{mathrsfs}
\usepackage{fancyhdr}
\usepackage[bottom=0.75in, top=1in, left=0.5in, right=0.5in]{geometry}
\usepackage{array}   % for \newcolumntype macro
\newcolumntype{L}{>{$}l<{$}}

\theoremstyle{plain}
\newtheorem*{theorem}{Theorem}
\newtheorem*{proposition}{Proposition}
\newtheorem{exercise}{Exercise}
\newtheorem*{lemma}{Lemma}
\newtheorem*{corollary}{Corollary}

\theoremstyle{definition}
\newtheorem*{definition}{Definition}
\newtheorem*{example}{Example}

\theoremstyle{remark}
\newtheorem*{remark}{Remark}

\renewcommand{\phi}{\varphi}
\renewcommand{\epsilon}{\varepsilon}
\renewcommand{\emptyset}{\varnothing}

\newcommand{\RR}{\mathbb{R}}
\newcommand{\ZZ}{\mathbb{Z}}
\newcommand{\NN}{\mathbb{N}}
\newcommand{\CC}{\mathbb{C}}
\newcommand{\QQ}{\mathbb{Q}}

\DeclareMathOperator{\ima}{\text{im}}

\begin{document}

\topmargin=-40pt
\rhead{Henry Woodburn}
\lhead{Math 622}
\renewcommand{\headrulewidth}{1pt}
\renewcommand{\headsep}{20pt}
\thispagestyle{fancy}

{\Huge \bfseries \noindent Homework 2}

\begin{enumerate}
    \item[1.] Let $f: \RR^2 \rightarrow \RR$ be a $C^\infty$ map. Then $f$ is not injective.
    
    Consider two cases. First, if $D_{(x,y)}f = (0,0)$ for all $(x,y) \in \RR^2$, then by the 
    fundamental theorem of calculus, $f$ is constant. 

    Otherwise, without loss of generality we can find a point $(x_0, y_0)$ such that $D_{y_0}f \neq 0$.
    Let $z_0 := f(x_0,y_0)$. Then it follows from the implicit function theorem that there
    is a smooth function $g: U \rightarrow V$, where $U$ and $V$ are open neighborhoods in $\RR$ of 
    $x_0$ and $y_0$ respectively, such that we have $f(x, g(x)) = z_0$ for all $x \in U$. In particular,
    this implies $f$ is not injective.

    \item[2.] (a.) Let $\pi$ denote the stereographic projection $S^n \setminus\{N\} \rightarrow \RR^n$,
    given by $\pi(x_1, \dots, x_{n+1}) = \frac{(x_1, \dots, x_n)}{1 - x_{n+1}}$.

    We prove that $u = \pi(x)$ is the point at which the line through $x$ and $N$ intersects the
    linear subspace defined by $x_{n + 1} = 0$. 

    A point $y$ on this line is given by a sum of the form $y = t(x) + (1-t)(N)$ for $t \in \RR$.. We calculate the 
    value of $t$ such that $y_{n + 1} = 0$. We see that
    \[
        y_{n + 1} = t x_{n + 1} + (1-t),
    \]
    and the value $t = \frac{1}{1 - x_{n+1}}$ gives $y_{n + 1} = 0$. Then subsituting this 
    into our equation, we see that the rest of the coordinates of $y$ are given by $y_i = \frac{x_i}{1 - x_{n+1}}$
    for $i = 1, \dots, n$, and of course we can regard $y$ as an $n-tuple$ within the subspace 
    $\RR^{n+1}|_{x_{n+1} = 0} \simeq \RR^n$.

    We also see that the projection with respect to the south pole given by $\tilde{\pi}(x) = -\pi(-x)$
    is the map obtained from first changing coordinates to be in the first situation, projecting, and
    returning to the original coordinates. Then this map has the same property.

    (b.) We can derive the inverse $\pi^{-1}: \RR^n \rightarrow S^n\setminus\{N\}$ in a similar way.
    Namely, for $u \in \RR^n$, consider the line between $(u,0)$ and $N$ given by
    $y = t(u,0) + (1-t)N$ for $t \in \RR$. In this case, we need to solve for $\|y\| = 1$, 
    or equivalently $\|y\|^2 = 1$. Since the vectors $(u,0)$ and $N$ are orthogonal, 
    we simply have
    \[
        \|y\|^2 = t^2 \|u\|^2 + (1-t)^2,
    \]

    which we can use to solve for $t$ by setting this equal to $1$. We obtain $t = \frac{2}{\|u\|^2 + 1}$.
    Define $\phi(u)$ to be the value of the above equation at this given $t$. Then
    \[
        \phi(u) = \frac{(2u, \|u\|^2 - 1)}{\|u\|^2 + 1}.
    \]

    Then it is clear that the maps $\phi$ and $\pi$ are inverses of each other, so $\phi = \pi^{-1}$
    and $\pi$ is bijective.

    (c.) We compute the transition map $\tilde{\pi}\circ\pi^{-1}$ for the charts $(U, \pi)$, $(V, \tilde{\pi})$,
    with $U = S^n \setminus \{N\}$ and $V = S^n \setminus \{S\}$. To show this defines a smooth
    structure, we show $\tilde{\pi}\circ\pi^{-1}: \pi(U \cap V) \rightarrow \tilde{\pi}(U \cap V)$
    is smooth, where $U \cap V = S^n \setminus \{N, S\}$, $\pi(U \cap V) = \tilde{\pi}(U \cap V) = \RR^n \setminus\{0\}$.
    
    Both maps $\pi$ and $\tilde{\pi}$ only involve scalings of the first $n$ coordinates of the input 
    variable, and similarly $\pi^{-1}$, in its first $n$ coordinates, involves only a scaling
    of the input variable. Then we can expect $\tilde{\pi}\circ\pi^{-1}$ to be a function of 
    $\|u\|$ only. So it suffices to consider the special case of $n = 1$. 

    I claim that in general, $\tilde{\pi}\circ\pi^{-1}$ is the map $u \mapsto \frac{u}{\|u\|^2}$. 
    In one dimension we will obtain $\tilde{\pi}\circ\pi^{-1}(u) = \frac{u}{|u|^2}$ and the result 
    will extend to arbitrary dimension by the discussion above. 

    Consider a point $U \in \RR \setminus \{0\}$. Let $UN$ be the line segment from $U$ to $N$. Denote
    point at the intersection of $UN$ with $S^1$ by $C$, and denote by $CS$ the line 
    segment from $C$ to $S$. Let $O$ be the origin. 
    Let $B$ denote the point at the intersection of $CS$ and the $x$-axis.
    By Thale's theorem from basic geometry, the line $CS$ meets $UN$ at a right angle. 
    Thus the traingle $\triangle UCB$ is a right triangle, and clearly so is $\triangle BOS$. 
    Moreover we see that the angles $\angle UBC$ and $\angle OBS$ are equal. Thus $\triangle UCB \sim
    \triangle BOS$, and we have the relation
    \[
        \frac{|UO|}{|NO|} = \frac{|OS|}{|OB|}.
    \]
    In other words, $B = \frac{1}{U}$, now considering these as real numbers. Then indeed the map 
    is given by $U \mapsto \frac{U}{|U|^2}$ and extends to the case $\RR^n$ in the obvious way. 
    This map is smooth $\RR^n \rightarrow \RR^n$. The other transition map is defined in the 
    same way. Then we have verified that these charts define a smooth structure. 

    (d.) Let $(U^{\pm}_i, \phi^\pm_i)$ be the given atlas for $S^n$, and let $(U, \pi), (V, \tilde{\pi})$
    be the atlas from above. To check that these two atlases are smoothly compatible, it will suffice
    to check that the charts $(\phi^\pm_{n+1}, U^\pm_{n + 1})$ are both compatible with 
    $(U, \pi)$, and that charts $(\phi^\pm_{i}, U^\pm_{i})$ are both compatible with $(U, \pi)$ for
    some $i < n+1$. The other cases are proven similarly. 

    First, $U^+_{n+1} \cap U$ is the ``upper'' half ($x_{n+1} > 1$) of the sphere minus the north pole.
    Then $\phi^+_{n + 1} \circ \pi^{-1}$ has domain $\{x \in \RR^n: \|x\| > 1$, and we can calculate
    that 
    \[
        \phi^+_{n + 1}\circ \pi^{-1}(u) =  \frac{2u}{\|u\|^2 + 1},
    \]
    so this map is clearly smooth. Similarly we have
    \[
        \pi \circ {(\phi^+_{n + 1})}^{-1}(u) = \frac{u}{1 - \sqrt{1- \|u\|^2}}
    \]
    which is also smooth, noting that $0 < \|u\| < 1$ in this case.

    The opposite chart $(U^-_{n+1}, \phi^-_{n+1})$ can be seen to be compatible with $(U, \pi)$ as well.

    Now for $0 \leq i < n+1$, we have that $\phi^+_{i}\circ\pi^{-1}$ maps $\{x \in \RR^n: x_i \geq 0\}$
    to $\{x \in \RR^n: \|x\| \leq 1\ \text{and}\ x_i \geq 0\}$, and that the composition
    is smooth is due to the fact that $\pi^{-1}$ is smooth on the given domain, and $\phi$ is smooth
    on the image of this domain under $\pi^{-1}$. The rest of the charts are similar.

    \item[3.] Let $\tilde{\RR}$ be the smooth manifold with a single chart $(\tilde{\RR}, \phi)$
    where $\phi(x) = x^3$. Let $f: \RR \rightarrow \RR$ be smooth in the usual sense. 

    (a.) To show that $f$ is smooth as a map $\RR \rightarrow \tilde{\RR}$, we must show
    that its coordinate representation $\phi \circ f \circ \text{id}^{-1} = \phi \circ f = f^3$ is
    a smooth map $\RR \rightarrow \RR$. It is, since it is a composition of smooth maps $\RR \rightarrow \RR$.
    
    (b.) Let $f: \tilde{\RR} \rightarrow \RR$ be a map. First suppose it is smooth. By taylor's theorem,
    we can write
    \[
        f(x) = f(0) + f'(0)x + f''(0)x^2 + f'''(0)x^3 + r(x)
    \]
    where $r(x) = 0(x^3)$, and thus we have
    \[
        f(x^{1/3}) = f(0) + f'(0)x^{1/3} + f''(0)x^{2/3} + f'''(0)x + r(x^{1/3}).
    \]

    That $f$ is a smooth map between manifolds implies $f(x^{1/3})$ is a smooth map $\RR \rightarrow \RR$.
    But we see by taking one or two derivatives on the right side that unless $f'(0) = f''(0) = 0$, 
    this cannot be true. We may have $f^{(3n)}(0)$ nonzero, but for all other derivatives
    the value at zero must vanish, or else we can derive a similar contradiction.

    conversely suppose all derivatives of $f$ order not a multiple of $3$ vanish at $0$. Again we can 
    write
    \[
        f(x) = f(0) + f'''(0)x^3 + r(x)
    \]
    where $r(x) = o(x^3)$. Then substituting $x^{1/3}$, we have
    \[
        g(x) = f(x^{1/3}) = f(0) + f'''(0)x + r(x^{1/3}),
    \]
    and thus $g$ is a smooth function since $r(x^{1/3}) = o(x)$. Then $f$ is a smooth map $\tilde{\RR} 
    \rightarrow \RR$.

    \item[4.] Let $P = \text{span}\{e_1, \dots, e_k\}$, $Q = \text{span}\{e_{k + 1}, \dots, e_n\}$ 
    be subspaces of $\RR^n$. Let $S$ be a $k$-dimensional subspace which intersects $Q$ trivially. 

    The map $\phi$ assigns to $S$ a $(n-k)\times k$ matrix $B$ such that 
    \[
        S = \{v + B v: v \in P\} = \left\{\begin{pmatrix}
        I_k\\ B
        \end{pmatrix}v: v \in P \right\}.
    \]
    Then since $P = \text{span}\{e_1, \dots, e_k\}$, we have
    \[
        S = \text{span}\left\{\begin{pmatrix}
        I_k\\ B
        \end{pmatrix} e_i\right\}_{i = 1}^k.
    \]
    In other words, the columns of $\begin{pmatrix}
        I_k\\ B
        \end{pmatrix}$ span $S$.

    For uniqueness, suppose there is another such matrix $K$. Then in other words, 
    $S$ is the graph of the linear map $K: P \rightarrow Q$. But by the discussion from the book,
    there is a unique map whose graph is equal to $S$. Then we must have $B = K$, and $B$ is unique. 

    \item[5.] Let $M$ be a smooth $n$ dimensional manifold which is also a group, such that the group operation
    is a smooth map $M \times M \rightarrow M$. Fix some $x \in M$. Choose charts $(U, \phi),
    (U', \phi)$ with $x \in U$ and $x^{-1} \in U'$. Also let $(V, \psi)$ be a chart with $e \in V$.
    
    Let $F: \phi(U) \times \phi'(U') \rightarrow \psi(V)$ be the map $F(s,t) = 
    \psi\left(\phi^{-1}(s)\cdot \phi'^{-1}(t)\right)$. Fix $s_0 = \phi(x)$. Let $F_{s_0}(t) = F(s_0,t)$. 
    Without carefully defining the charts, we can 
    construct a map $G$ which maps a neighborhood of $\psi(e)$ in $\RR^n$ to a neighborhood of $\phi'(x^{-1})$
    such that $F_{s_0}\circ G$ is the identity on its domain.
    Similarly construct a map $H$ which maps a neighborhood of $\phi(s)$ in $\RR^n$ to a neighborhood
    of $\psi(e)$ in $\RR^n$ such that $H \circ F_{s_0}$ is the identity on its domain. 
    Both $G$ and $H$ should map from $\RR^n$ into $M$, multiply by $x^{-1}$, and map back into $\RR^n$. 

    Then by the chain rule, we have 
    \[
        D_{\phi'(x^{-1})}F \circ D_{e}G = D_{e}H \circ D_{\phi'(x^{-1})}F = I
    \]
    which verifies that the map $F$ has nonsingular differential in the second coordinate at the 
    point $(\phi(x), \phi'(x^{-1}))$. By the implicit function theorem, there exists $U \ni \phi(x)$ 
    and $V \ni \phi'(x^{-1})$ and a smooth function $g: U \rightarrow V$. This is the coordinate 
    representation of the map $x \mapsto x^{-1}$, and thus the map $x \mapsto x^{-1}$ is a 
    smooth map $M \rightarrow M$.


    

\end{enumerate}

\end{document}