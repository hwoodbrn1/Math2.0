\documentclass[11pt, reqno]{article}

\usepackage{amsmath, amsthm, amssymb}
\usepackage{enumitem}
\usepackage{tcolorbox}
\usepackage{hyperref}
\usepackage{tikz}
\usepackage{tikz-cd}
\usepackage{pgfplots}
\pgfplotsset{compat=1.18}
\usetikzlibrary{arrows.meta}
\usepackage{mathrsfs}
\usepackage{fancyhdr}
\usepackage[bottom=0.75in, top=1in, left=0.5in, right=0.5in]{geometry}
\usepackage{array}   % for \newcolumntype macro
\newcolumntype{L}{>{$}l<{$}}

\theoremstyle{plain}
\newtheorem*{theorem}{Theorem}
\newtheorem*{proposition}{Proposition}
\newtheorem{exercise}{Exercise}
\newtheorem*{lemma}{Lemma}
\newtheorem*{corollary}{Corollary}

\theoremstyle{definition}
\newtheorem*{definition}{Definition}
\newtheorem*{example}{Example}

\theoremstyle{remark}
\newtheorem*{remark}{Remark}

\renewcommand{\phi}{\varphi}
\renewcommand{\epsilon}{\varepsilon}
\renewcommand{\emptyset}{\varnothing}

\newcommand{\RR}{\mathbb{R}}
\newcommand{\ZZ}{\mathbb{Z}}
\newcommand{\NN}{\mathbb{N}}
\newcommand{\CC}{\mathbb{C}}
\newcommand{\QQ}{\mathbb{Q}}

\DeclareMathOperator{\ima}{\text{im}}

\begin{document}

\topmargin=-40pt
\rhead{Henry Woodburn}
\lhead{Math 622}
\renewcommand{\headrulewidth}{1pt}
\renewcommand{\headsep}{20pt}
\thispagestyle{fancy}

{\Huge \bfseries \noindent Homework 1}

\begin{enumerate}
    \item[1.] Let $F: \text{Mat}_{n\times n}(\RR) \rightarrow \RR$ be given by $F(A) : = \text{det} A$.
    
    (a.) $F$ is a polynomial in the $n^2$ entries of the matrix. It is thus differentiable as a map
    $\RR^{n^2} \rightarrow \RR$. The space $\RR^{n^2}$ is isomorphic to the space of $n\times n$ matrices as a finite
    dimensional normed vector space. 

    (b.) Suppose $A$ is invertible. We are looking for a linear map $D_A F: \text{Mat}_{n \times n}(\RR) \rightarrow
    \text{Mat}_{n \times n}(\RR)$ such that 
    \[  
        F(A + H) = F(A) + D_A F(H) + r(H)
    \]
    where $r(H) = o(\|H\|)$. We have
    \[
        F(A + H) = \det(A + H) = \det(A)\det(I + A^{-1}H),
    \]
    and we can write
    \[
        \det(I + A^{-1}H) = 1 + \text{Tr}(A^{-1}H) + r(H),
    \]
    where $r(H)$ consists of terms in the determinate of at least order $2$ in the components of $A^{-1}H$.
    
    We can bound $|r(H)|$ by a sum of absolute values of monics in elements of $A^{-1}H$ of orders 
    at least $2$. Then use $(A^{-1}H)_{ij} \leq \|A^{-1}H\| \leq \|A^{-1}\|\|H\|$, so that 
    \[
        \lim_{\|H\| \rightarrow 0}\frac{|r(H)|}{\|H\|} = 0.
    \]

    Then we have
    \[
        \det(A + H) = \det(A) + \det(A)\text{Tr}(A^{-1}H) + \det(A)r(H),
    \]
    where of course the last term is still $o(\|H\|)$, and where $D_A F(H) = \det(A)\text{Tr}(A^{-1}H)$
    is a linear map. 

    (c.) Using the formula for the inverse of an invertible matrix $A$, we have 
    \[
        D_A F(H) = \det(A)\text{Tr}(A^{-1}H) = \det(A)\frac{1}{\det(A)}\text{Tr}(\text{adj}(A)H) = \text{Tr}(\text{adj}(A)H).
    \]
    Since the set of invertible matrices is dense in $\text{Mat}_{n \times n}(\RR)$, we can 
    approach a non-invertible matrix $B$ by a sequence of invertible matrices $\{A_j\}_{j = 1}^\infty$
    so that $A_j \rightarrow B$.
    Since $F$ is continuously differentiable, $\lim_{j \rightarrow \infty} D_{A_j} F(H) = D_{B} F(H)$.
    Then $\lim_{j \rightarrow infty} D_{A_j} F(H) = \text{Tr}(\text{adj}(A_j)H) = \text{Tr}(\text{adj}B)$,
    as trace and adjoint are continuous functions. 

    \item[2.] Suppose $\Gamma: \RR \rightarrow GL_n(\RR)$ is a smooth map such that $\Gamma(t)$ is orthogonal
    for every $t \in \RR$. By the product rule, we have
    \[
        \frac{d}{dt}(\Gamma(t)^T\Gamma(t)) = \frac{d}{dt}(\Gamma(t)^T)\Gamma(t) + \Gamma(t)^T \frac{d}{dt}\Gamma(t) = 0.
    \]
    And we also see that 
    \[
        \frac{d}{dt}(\Gamma(t)^T) = \lim_{h \rightarrow 0}\frac{\Gamma(t + h)^T - \Gamma(t)^T}{h}
        = \lim_{h \rightarrow 0}\frac{\left(\Gamma(t + h) - \Gamma(t)\right)^T}{h} = \left(\frac{d}{dt}\Gamma(t)\right)^T.
    \]

    Then 
    \[
        \left(\Gamma(t)^{-1}\frac{d}{dt}\Gamma(t)\right)^T = \left(\Gamma(t)^T\frac{d}{dt}\Gamma(t)\right)^T
        = \left(\frac{d}{dt}\Gamma(t)\right)^T\Gamma(t) = -\Gamma(t)^T \frac{d}{dt}\Gamma(t) = -\Gamma(t)^{-1} \frac{d}{dt}\Gamma(t)
    \]

\end{enumerate}

\end{document}