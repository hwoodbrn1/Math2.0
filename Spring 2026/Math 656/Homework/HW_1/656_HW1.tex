\documentclass[11pt, reqno]{article}

\usepackage{amsmath, amsthm, amssymb}
\usepackage{enumitem}
\usepackage{tcolorbox}
\usepackage{hyperref}
\usepackage{tikz}
\usepackage{tikz-cd}
\usepackage{pgfplots}
\pgfplotsset{compat=1.18}
\usetikzlibrary{arrows.meta}
\usepackage{mathrsfs}
\usepackage{fancyhdr}
\usepackage[bottom=0.75in, top=1in, left=0.5in, right=0.5in]{geometry}
\usepackage{array}   % for \newcolumntype macro
\newcolumntype{L}{>{$}l<{$}}

\theoremstyle{plain}
\newtheorem*{theorem}{Theorem}
\newtheorem*{proposition}{Proposition}
\newtheorem{exercise}{Exercise}
\newtheorem*{lemma}{Lemma}
\newtheorem*{corollary}{Corollary}

\theoremstyle{definition}
\newtheorem*{definition}{Definition}
\newtheorem*{example}{Example}

\theoremstyle{remark}
\newtheorem*{remark}{Remark}

\renewcommand{\phi}{\varphi}
\renewcommand{\epsilon}{\varepsilon}
\renewcommand{\emptyset}{\varnothing}

\newcommand{\RR}{\mathbb{R}}
\newcommand{\ZZ}{\mathbb{Z}}
\newcommand{\NN}{\mathbb{N}}
\newcommand{\CC}{\mathbb{C}}
\newcommand{\QQ}{\mathbb{Q}}

\DeclareMathOperator{\ima}{\text{im}}

\begin{document}

\topmargin=-40pt
\rhead{Henry Woodburn}
\lhead{Math 656}
\renewcommand{\headrulewidth}{1pt}
\renewcommand{\headsep}{20pt}
\thispagestyle{fancy}

{\Huge \bfseries \noindent Homework 1}

\begin{enumerate}
    \item[17.1.] Let $P$ be a nonzero projection, so that $P^2 = P \neq 0$. Then by norm properties,
    we have
    \[
        \|P\| = \|P^2\| \leq \|P\|\cdot\|P\| .
    \]

    Since $\|P\| \neq 0$, we have $\|P\| \geq 1$

    \item[17.2.] Fix some $\epsilon > 0$. Suppose the inequality does not hold. Then for every $N > 0$,
    there exists $n \geq N$ and $\lambda_n$ such that $\lambda_n \in \sigma(M_n)$ and $|\lambda_n|
    \geq |\sigma(M)| + \epsilon$. Obtain an increasing sequence $n_k$ such that $\lambda_{n_k} \in \sigma(M_{n_k})$
    by induction, taking $n_{k + 1}$ to be the next $n$ greater than $n_{k}$ with the desired property. 
    
    Since $M_n$ is a convergent sequence, it is bounded in norm,
    and thus $|\sigma(M_n)|$ is bounded uniformly in $n$. Choose a convergent subsequence of $\lambda_{n_k}$ which converges
    to some $\lambda$, which also must satisfy $|\lambda| \geq |\sigma(M)| + \epsilon$.
    Denote this also by $\lambda_{n_k}$ for simplicity. Then $(M - \lambda_{n_k}) \rightarrow (M - \lambda)$.
    Since $(M - \lambda_{n_k})$ is not invertible, and the set of non-invertible elements is closed,
    we have that $(M - \lambda)$ is not invertible as well, implying $\lambda \in \sigma(M)$. But
    this is impossible as it lies outside the spectral radius of $M$. Then we are done.

    \item[17.4.] We know from complex analysis that we can define a branch of the logarithm 
    on any open set in $\CC$ which does not contain zero. Let $\Omega$ be an open set not containing
    $0$ which contains $\sigma(M)$. This is possible as $\sigma(M)$ is closed. Define 
    a branch of the complex logarithm, which is analytic on $\Omega$. By the hypothesis that $0$ 
    can be connected to $\infty$ by a path in $\rho(M)$, it is possible to enclose $\sigma(M)$
    by a path $\gamma$ contained within $\Omega$ which does not encircle $0$. Using this path,
    we define 
    \[
        \log(M) = \oint_\gamma (\zeta - M)^{-1}f(\zeta)d\zeta.
    \]

    Moreover, $\exp$ is analytic on $\CC$ so that $\exp(\log(M))$ is well defined, and equals the
    identity by theorem 5, since $\exp(\log(\zeta)) = \zeta$ as functions on $\CC$.

    \item[17.5.] Let $\overline{\mathcal{L}_M}$ be the closure of the algebra generated by elements 
    $M$ and $(\lambda - M)^{-1}$ for $\lambda \in \rho(M)$ within some larger Banach algebra $\mathcal{M}$.

    Since $(\lambda - M)^{-1}$ is analytic on $\rho(M)$, it can be expressed as a power series. Then
    it is clear that the elements $M$ and $(\lambda - M)^{-1}$ commute for all $\lambda \in \rho(M)$. 
    Every element of $\mathcal{L}_M$ is a polynomial in $M$ and elements of the form $(\lambda - M)^{-1}$.
    Then it is clear that $\mathcal{L}$ is a commutative algebra. 

    Moreover, take any $A, B \in \overline{\mathcal{L}_M}$ and choose $A_n, B_n \in \mathcal{M}_L$ 
    converging to $A$ and $B$ respectively. Then by the continuity of the multiplication operation
    in an algebra, 
    \[
        A \cdot B = \lim_n A_n\cdot B_n = \lim_n B_n\cdot A_n = B\cdot A,
    \]
    which proves $\overline{\mathcal{M}_L}$ is a commutative subalgebra of $\mathcal{M}$

    \item[19.1.] Let $M$ be a maximal ideal in $C(S)$. If $M$ contains a nonzero function $f$,
    then $\frac{1}{f}\cdot f = 1 \in M$, and thus $M = C(S)$, a contradiction. Then every element
    of $M$ is zero at some point. Moreover, there must exist at least one point at which
    all elements of $M$ vanish. If not, choose a collection of functions $\{f_n\}_{n \in I}$
    which do not uniquely vanish at any $x \in S$. Without loss of generality we can 
    assume each $f_n$ is positive in some neighborhood. By urysohn's lemma, we can assume the
    $f_n$ are positive and supported within this neighborhood. Then by taking a finite subcover 
    and summing over these indices, we obtain an invertible element in $M$, giving a contradiction. 

    Choosing any $x_0$ at which every element of $M$ vanishes, we have that 
    $M \subset \{f \in C(S): f(x_0) = 0\}$ implying $M$ is of the desired form since $M$ is maximal.

\end{enumerate}

\end{document}